\InputIfFileExists{../data/global.src}\relax\relax

\iffull
\title{TODO}
\subtitle{Tutorium sechs}
\date{KW 24}
\addbibresource{references.bib}
\fi
\SetTutoriumNumber{6}

\iffull\begin{document}
\titleframe

\TopicOverview{7}
\fi


\iffull{\SummaryFrame
\begin{frame}[fragile,c]{Kurzwiederholung}
    \begin{itemize}[<+(1)->]
        \itemsep9.25pt
        \item TODO: zusammenfassungsfolien der wdg nochmaleinbringe
    \end{itemize}
\end{frame}
}\fi

% TODO: make sectionlink auto triggerat start of section
\SetNextSectionText[.75\linewidth]{TODO\smallskip\\ TODO}
\section{Präsenzaufgabe}
\begin{frame}[fragile,c]{Präsenzaufgabe}
\begin{aufgabe}{Meine Kleine, du hast Palindrom!}
\end{aufgabe}
\end{frame}

\begin{frame}[c,fragile]{Übungsblatt 9 - Aufgabe 2}
    \onslide<2->{\only<3->{\color{gray}}In dieser Aufgabe sollen Sie die Methode {\color{black}\bjava{public static boolean isPalindrome} \bjava{(String s)}} implementieren, welche überprüfen soll ob es sich bei dem als Parameter übergebenen String \bjava{s} um ein Palindrom handelt. Ein Palindrom ist ein Wort, das vorwärts und rückwärts gelesen identisch ist. Die Überprüfung soll {\color{black}nicht case-sensitive} sein, d.h. das Wort \textit{Kajak} soll zum Beispiel ein gültiges Palindrom sein.
    Verwenden Sie bei ihrer Implementierung {\color{black}keine Schleifen}, sondern {\color{black}rekursive Methoden-Aufrufe!}}\bigskip

    \onslide<4->{\color{black}Idee:}
    \footnotesize\begin{itemize}
        \item<5-> Prüfe für \(i = 0\) bis \(i = \floor{\sfrac{\text{\bjava{s.length}}}{2}}\), ob \say{\bjava{s[i] == s[s.length - i - 1]}}.
        \item<6-> \bjava{String::toLowerCase()} entweder für jeden Vergleich oder einmal per Hilfsmethode.
        \item<7-> Noch einfacher: Anstelle \bjava{i} zu inkrementieren, löschen wir das erste und letzte Zeichen nach dem Vergleich (via \bjava{String::substring(int, int)}---das Ende ist exklusiv)!
    \end{itemize}
\end{frame}

\begin{frame}[fragile]{Übungsblatt 9 - Aufgabe 2}
    \begin{itemize}[<+(1)->]
        \item Die Datei befindet sich hier: \only<2->{\textattachfile{\curpath Palindrome.java}{Palindrome.java}}
    \end{itemize}\vfill
\SetupLstHl\lstfs{10}%
\begin{plainjava}
!*\onslide<3->*!public static boolean isPalindrome(String s) { // die "Hilfsmethode"
!*\onslide<4->*!    return isPalindromeRecursive(s.toLowerCase());
!*\onslide<3->*!}


!*\onslide<5->*!private static boolean isPalindromeRecursive(String s) {
!*\onslide<6->*!    if(s.length() < 2) // Basisfall: weniger als zwei Zeichen (Abrunden)
!*\onslide<6->*!        return true;
!*\onslide<7->*!    else if(s.charAt(0) != s.charAt(s.length() - 1)) // Basisfall
!*\onslide<7->*!        return false;
!*\onslide<8->*!    else // Rekursionsfall
!*\onslide<8->*!        return isPalindromeRecursive(s.substring(1, s.length() - 1));
!*\onslide<5->*!}
\end{plainjava}
\vfill\null
\end{frame}

\iffull
\MakeThePinguExplainIt[text width=7cm]{cap=!hide,glasses=!hide,sunglasses round,eyes shiny,cup=!hide,santa beard,santa hat,right item angle=-142,staff right length=17mm}{Mit \bjava{:lan:ret:c:urn:ran:} haben wir hier die zurückzugebenden anonymen Variablen referenziert. Generell ist hier das \say{Speichern} der Position für den Aufstieg der Rekursion informal dargestellt.}
\begin{frame}[c,fragile]{Übungsblatt 9 - Aufgabe 2\hfill Sie Simulantario Sie!}
\sollockinline\SetupLstHl\lstfs{10}\begin{plainjava}
!*\onslide<2->*!!*\MD4*!private static boolean isPalindrome(String s) { !*\rBS<handout:2-4|4->{s=\dq RegalelaGEr\dq }*!
!*\onslide<2->*!    !*\MD5*!return isPalindrom!*\mb{7,38}\mbg[2-3]{8-42}*!eRecursive(s!*\mb6*!.toLowerCase());!*\ml[4]{43}*! !*\rBS<handout:2-3|6-41>{\dq regalelager\dq }~~\rBS<handout:4|43>{true}*!
!*\onslide<2->*!}
!*\onslide<2->*!!*\MD{8,15,20,25,30,35}*!private static boolean isPalindromeRecursive(String s) { !*\rBS<handout:0|8-14>{s=\dq regalelager\dq }\rBS<handout:0|15-19>{s=\dq egalelage\dq }\rBS<handout:0|20-24>{s=\dq galelag\dq }\rBS<handout:2-3|25-29>{s=\dq alela\dq }\rBS<handout:0|30-34>{s=\dq lel\dq }\rBS<handout:0|35-37>{s=\dq e\dq }*!
!*\onslide<2->*!    !*\MD{9,16,21,26,31,36}*!if(s.length() < 2) return true;!*\ml[3]{37}*!
!*\onslide<2->*!    !*\MD[2]{10,17,22,27,32}*!else if(s.charAt(0) != s.charAt(s.length() - 1)) return false; !*\rBS<handout:0|10-14>{'r'!='r'}\rBS<handout:0|17-19>{'e'!='e'}\rBS<handout:0|22-24>{'g'!='g'}\rBS<handout:2|27-29>{'a'!='a'}\rBS<handout:0|32-34>{'l'!='l'}*!
!*\onslide<2->*!    !*\MD{11,18,23,28,33}*!else return isPalindrom!*\mb{14,19,24,29,34}\mbg[2-3]{15-18,20-23,25-28,30-33,35-41}*!eRecursive(s!*\mb{13}*!.substring(1, s!*\mb{12}*!.length() - 1));!*\ml{38,39,40,41,42}*!
!*\onslide<2->*!}
\end{plainjava}
\only<handout:1|3>{\begin{center}
    \huge\bfseries \say{RegalelaGEr}\\[-2mm]
    {\normalfont\info{Wer sieht auch ein \T{e}, dass die Arme hochwirft?}}
\end{center}}
\begin{onlyenv}<handout:2-4|4-43>\begin{center}\lstfs{6}\lstset{aboveskip=0pt,belowskip=0pt,add to literate={:ll:}{{{\color{lightgray!60!gray}$\ldots$}}}1}
    \begin{tikzpicture}[b/.style={draw=gray,fill=white,text width=4.1cm,minimum height=2.25cm,thick,rounded corners=2pt,inner xsep=1em}]
            \node[b] (a) at(0,0) {%
\begin{plainjava}
!*\MD4*!boolean isPalindrome(String s) {
    !*\MD5*!return isPalindrom!*\mb[1-]{7-42}*!eRecurs!*\mb{6}*!:ll:!*\ml{43-}*!
}
\end{plainjava}\medskip
\centerline{\bjava{s = "RegalelaGEr"}\onslide<43->{ \bjava{:lan:ret:c:urn:ran: = true}}}
            };
\node[above right,gray] at(a.south west) {\(1\)};
\begin{onlyenv}<handout:-3|8-42>\node[b,right=-4cm,yshift=-.1cm] (b) at(a.east) {%
\begin{plainjava}
!*\MD8*!boolean isPalindromeRecursive:ll:
    !*\MD9*!if(s.length() < 2) return:ll:
    !*\MD{10}*!else if(s.charAt(0) != s.c:ll:
    !*\MD{11}*!else return isPalindrom!*\mb[1-]{14-41}*!eRe!*\mb{12-13}*!:ll:!*\ml{42-}*!
}
\end{plainjava}\medskip
\centerline{\bjava{s = "regalelager"}\onslide<42->{ \bjava{:lan:ret:c:urn:ran: = true}}}
        };
\node[above right,gray] at(b.south west) {\(2\)};
    \end{onlyenv}
\begin{onlyenv}<handout:-3|15-41>\node[b,right=-4cm,yshift=-.1cm] (c) at(b.east) {%
\begin{plainjava}
!*\MD{15}*!boolean isPalindromeRecursive:ll:
    !*\MD{16}*!if(s.length() < 2) return:ll:
    !*\MD{17}*!else if(s.charAt(0) != s.c:ll:
    !*\MD{18}*!else return isPalindrom!*\mb[1-]{19-40}*!eRe:ll:!*\ml{41-}*!
}
\end{plainjava}\medskip
\centerline{\bjava{s = "egalelage"}\onslide<41->{ \bjava{:lan:ret:c:urn:ran: = true}}}
        };
\node[above right,gray] at(c.south west) {\(3\)};
    \end{onlyenv}
\begin{onlyenv}<handout:-3|20-40>\node[b,right=-4cm,yshift=-.1cm] (d) at(c.east) {%
\begin{plainjava}
!*\MD{20}*!boolean isPalindromeRecursive:ll:
    !*\MD{21}*!if(s.length() < 2) return:ll:
    !*\MD{22}*!else if(s.charAt(0) != s.c:ll:
    !*\MD{23}*!else return isPalindrom!*\mb[1-]{24-39}*!eRe:ll:!*\ml{40-}*!
}
\end{plainjava}\medskip
\centerline{\bjava{s = "galelag"}\onslide<40->{ \bjava{:lan:ret:c:urn:ran: = true}}}
        };
    \node[above right,gray] at(d.south west) {\(4\)};
    \end{onlyenv}
\begin{onlyenv}<handout:-3|25-39>\node[b,right=-4cm,yshift=-.1cm] (e) at(d.east) {%
\begin{plainjava}
!*\MD{25}*!boolean isPalindromeRecursive:ll:
    !*\MD{26}*!if(s.length() < 2) return:ll:
    !*\MD{27}*!else if(s.charAt(0) != s.c:ll:
    !*\MD{28}*!else return isPalindrom!*\mb[1-]{29-38}*!eRe:ll:!*\ml{39-}*!
}
\end{plainjava}\medskip
\centerline{\bjava{s = "alela"}\onslide<39->{ \bjava{:lan:ret:c:urn:ran: = true}}}
        };
\node[above right,gray] at(e.south west) {\(5\)};
    \end{onlyenv}
\begin{onlyenv}<handout:3|30-38>\node[b,right=-4cm,yshift=-.1cm] (f) at(e.east) {%
\begin{plainjava}
!*\MD{30}*!boolean isPalindromeRecursive:ll:
    !*\MD{31}*!if(s.length() < 2) return:ll:
    !*\MD{32}*!else if(s.charAt(0) != s.c:ll:
    !*\MD{33}*!else return isPalindrom!*\mb[1-]{34-37}*!eRe:ll:!*\ml{38-}*!
}
\end{plainjava}\medskip
\centerline{\bjava{s = "lel"}\onslide<38->{ \bjava{:lan:ret:c:urn:ran: = true}}}
        };
\node[above right,gray] at(f.south west) {\(6\)};
    \end{onlyenv}
\begin{onlyenv}<handout:3|35-37>\node[b,right=-4cm,yshift=-.1cm] (g) at(f.east) {%
\begin{plainjava}
!*\MD{35}*!boolean isPalindromeRecursive:ll:
    !*\MD{36}*!if(s.length() < 2) return:ll:!*\ml[1-]{37-}*!
    else if(s.charAt(0) != s.c:ll:
    else return isPalindromeRe:ll:
}
\end{plainjava}\medskip
\centerline{\bjava{s = "e"}\onslide<37->{ \bjava{:lan:ret:c:urn:ran: = true}}}
        };
\node[above right,gray] at(g.south west) {\(7\)};
    \end{onlyenv}
    \end{tikzpicture}
\end{center}
\end{onlyenv}
\begin{tikzpicture}[remember picture,overlay]
    \onslide<handout:4-|44->{\node[left=-7mm,scale=.8] at(current page.-20) {\usebox\pinguexplainbox};}
\end{tikzpicture}
\end{frame}
\fi

\iffull
\begin{frame}[c,fragile]{Übungsblatt 9 - Aufgabe 2\hfill Iterativer Ansatz}
    \begin{itemize}[<+(1)->]
        \item Die ersten beiden Aufgaben lassen sich \textit{einfacher} Iterativ prüfen.
        \item Für das Palindrom schauen wir für jedes Zeichen \(i\) der \say{linken Hälfte} ob es mit dem gespiegelten \(\text{\T{length}} - i - 1\) der \say{rechten Hälfte} übereinstimmt:
    \end{itemize}
\begin{plainjava}
!*\onslide<4->*!public static boolean isPalindromeIterative(String s) {
!*\onslide<5->*!    s = s.toLowerCase();
!*\onslide<6->*!    for(int i = 0; i < s.length() / 2; i++) {
!*\onslide<7->*!        if(s.charAt(i) != s.charAt(s.length() - i - 1))
!*\onslide<7->*!            return false;
!*\onslide<6->*!    }
!*\onslide<8->*!    return true;
!*\onslide<4->*!}
\end{plainjava}
\end{frame}

\begin{frame}[c,fragile]{Übungsblatt 9 - Aufgabe 2\hfill Iterativer Ansatz, II}
    \lstfs{10}\begin{itemize}[<+(1)->]
        \item Dies können wir als Tail-Rekursion umschreiben (\only<2->{\textattachfile{\curpath PalindromeIterative.java}{PalindromeIterative.java}}):
    \end{itemize}
\begin{plainjava}
!*\onslide<3->*!public static boolean isPalindrome(String s) {
!*\onslide<4->*!    return helper(s.toLowerCase(), 0);
!*\onslide<3->*!}
!*\onslide<3->*!
!*\onslide<5->*!private static boolean helper(String s, int i) {
!*\onslide<6->*!    if (i >= s.length() / 2)
!*\onslide<6->*!        return true;
!*\onslide<7->*!    if (s.charAt(i) != s.charAt(s.length() - i - 1))
!*\onslide<7->*!        return false;
!*\onslide<8->*!    return helper(s, i + 1);
!*\onslide<5->*!}
\end{plainjava}
\begin{tikzpicture}[overlay, remember picture]
\begin{uncoverenv}<9->
    \node[above left=.925cm,text width=9.65cm,draw=gray,thick,rounded corners=2pt,scale=.65,yshift=.25cm] at(current page.south east) {%
\begin{plainjava}[aboveskip=0pt, belowskip=0pt]
s = s.toLowerCase();
for(int i = 0; i < s.length() / 2; i++) {
    if(s.charAt(i) != s.charAt(s.length() - i - 1))
        return false;
}
return true;
\end{plainjava}
    };
\end{uncoverenv}
\end{tikzpicture}
\end{frame}
\fi

\SetNextSectionText[.55\linewidth]{TODO}
\section{Abschließendes}
{\SummaryFrame
\begin{frame}[t]{Zusammenfassend}
\pause \printBibCommand
\vfill\vfill % double fill for more fraction
\begin{itemize}[<+(1)->]
    \itemsep5pt
    \item TODO
\end{itemize}
\end{frame}
}

% TODO: pascal case


\outro{\vskip9mm\centering \onslide<2->{\scalebox{1.35}{\begin{tikzpicture}
       % TODO
    \end{tikzpicture}}}}

\iffull\end{document}\fi
