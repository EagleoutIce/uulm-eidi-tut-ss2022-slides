\InputIfFileExists{../data/global.src}\relax\relax

\iffull
\title{quaerere atque disponere} % \ldots\ sich selbst
\subtitle{Tutorium M\texorpdfstring{\setbox0=\hbox{a}\resizebox*!{\ht0}{\tiny{8}}}{a}cht}
\date{KW 26}
\addbibresource{references.bib}
\fi
\SetTutoriumNumber{8}

\iffull\begin{document}
\titleframe

\TopicOverview{9}
\fi

\iffull{\SummaryFrame
\begin{frame}[c]{Kurzwiederholung}
\begin{itemize}[<+(1)->]
   \itemsep8pt
    \item Die Rekursionsfamilie \info{Methoden die sich selbst aufrufen} ist riesig: \begin{itemize}
        \itemsep3pt
        \item Ruft sich eine Methode maximal einmal selbst auf, ist sie linear rekursiv.
        \item Head-Rekursiv, wenn dieser Aufruf das erste Statement ist \info{alles passiert im Aufstieg}
        \item Tail-Rekursiv, wenn dieser Aufruf das letzte Statement ist \info{alles passiert im Abstieg}
        \item Ruft sie sich auch mehrfach pro Rekursionsfall auf, ist sie verzweigt rekursiv.
    \end{itemize}
    \item Die \say{Big-O-Notation} nutzen wir zum \info{z.B.} Vergleich der Laufzeitkomplexität:\pause
    \begin{itemize}
        \itemsep3pt
        \item \(T(n) \in \O(f(n))\pause{}\iff \exists n_0 \in \mathbb{N}\: c \in \mathbb{R}^+\: \forall n \geq n_0: T(n) \leq c \cdot f(n)\)
        \item Ab einem gewissen Punkt (\(n_0\)) ist \(c \cdot f(n)\) stets größer als \(T(n)\)
    \end{itemize}
\begin{center}%
\resizebox{\linewidth}{!}{%
\setlength{\aboverulesep}{0pt}%
\setlength{\belowrulesep}{0pt}%
\setlength{\extrarowheight}{.85ex}%
\begin{tabular}{c*{8}{c}}
    \toprule
    & {\cellcolor{pingu@green!100!pingu@purple!21} \(\O(1)\)} & {\cellcolor{pingu@green!85!pingu@purple!21}\(\O(\log n)\)} & {\cellcolor{pingu@green!69!pingu@purple!21} \(\O(n)\)} & {\cellcolor{pingu@green!53!pingu@purple!21} \(\O(n\log n)\)} & {\cellcolor{pingu@green!36!pingu@purple!21} \(\O(n^2)\)} & {\cellcolor{pingu@green!19!pingu@purple!21} \(\O(n^3)\)} & {\cellcolor{pingu@green!14!pingu@purple!21} \(\O(2^n)\)} & {\cellcolor{pingu@green!0!pingu@purple!21} \(\O(n!)\)} \\[0.45ex]\midrule
    {\footnotesize Bez:} &{\footnotesize konst.} & {\footnotesize logarithm.} & {\footnotesize linear} &{\footnotesize linear log.} & {\footnotesize quadratisch} & {\footnotesize kubisch} & {\footnotesize exponentiell} & {\footnotesize faktoriell} \\
\bottomrule
\end{tabular}}\bigskip
\end{center}
\end{itemize}
\end{frame}
}\fi

% TODO: make sectionlink auto triggerat start of section
\SetNextSectionText[.6\linewidth]{}
\section{Präsenzaufgabe}
{\taskenum
\begin{frame}[fragile,c]{Präsenzaufgabe}
\begin{aufgabe}{TODO}
\task<2->{TODO}
\end{aufgabe}
\end{frame}
}

\SetNextSectionText[.55\linewidth]{TODO}
\section{Abschließendes}
{\SummaryFrame
\begin{frame}[t]{Zusammenfassend}
\pause \printBibCommand
\vfill\vfill % double fill for more fraction
\begin{itemize}[<+(1)->]
    \itemsep4pt
    \item TODO
\end{itemize}
\end{frame}
}

\outro{\vskip9mm\centering \onslide<2->{\scalebox{1.42}{\begin{tikzpicture}
    \pingu[body type=legacy,name=pingu,right wing wave,eyes wink,left wing grab,height=10mm,blush,feet=sit];
    \draw[line cap=round]([xshift=.15mm]pingu-wing-right) to[bend right] ++(.1mm,1.1mm) -- ++(4.5mm,8.5mm) coordinate(@);
    \scope[rotate around=-18:(@)]
    \fill[pingu@yellow!80!pingu@red,rounded corners=1.5pt] (@) --++(1mm,0mm) --++(-1mm,.33mm) arc(270:630:4.325mm and 4.7mm) -- ++(-1mm,-.33mm) -- cycle;
    \draw[line cap=round,pingu@black,very thin] (@)++(-.35mm,.3mm) to[bend right=14] ++(.7mm,0);
    \endscope
    \draw[line cap=round](pingu-wing-right) to[bend right] ++(-.65mm,1.4mm) -- ++(-2mm,8mm) coordinate(@);
    \scope[rotate around=10:(@)]
    \fill[pingu@green,rounded corners=1.5pt] (@) --++(1mm,0mm) --++(-1mm,.33mm) arc(270:630:4.25mm and 5.133mm) -- ++(-1mm,-.33mm) -- cycle;
    % \fill[pingu@white,fill opacity=.6] (@)++(-3mm,7.15mm) ellipse[x radius=.39mm,y radius=.42mm];
    \draw[line cap=round,pingu@black,very thin] (@)++(-.35mm,.3mm) to[bend right=14] ++(.7mm,0);
    \endscope
    \draw[line cap=round]([xshift=.1mm]pingu-wing-right) to[bend right] ++(-.55mm,1.4mm) -- ++(-.25mm,11mm) coordinate(@);
    \scope[rotate around=4:(@)]
    \fill[pingu@blue,rounded corners=1.5pt] (@) --++(1mm,0mm) --++(-1mm,.33mm) arc(270:630:4.125mm and 4.9mm) -- ++(-1mm,-.33mm) -- cycle;
    % \fill[pingu@white,fill opacity=.6] (@)++(-3mm,6.75mm) ellipse[x radius=.39mm,y radius=.42mm];
    \draw[line cap=round,pingu@black,very thin] (@)++(-.35mm,.3mm) to[bend right=14] ++(.7mm,0);
    \endscope
    \draw[line cap=round]([xshift=.1mm]pingu-wing-right) to[bend right] ++(-.85mm,1.35mm) -- ++(-5mm,4.5mm) coordinate(@);
    \scope[rotate around=30:(@)]
    \fill[pingu@purple,rounded corners=1.5pt] (@) --++(1mm,0mm) --++(-1mm,.33mm) arc(270:630:3.525mm and 4.3mm) -- ++(-1mm,-.33mm) -- cycle;
    \draw[line cap=round,pingu@black,very thin] (@)++(-.35mm,.3mm) to[bend right=14] ++(.7mm,0);
    \endscope
\end{tikzpicture}}}}

\iffull\end{document}\fi
