\InputIfFileExists{../data/global.src}\relax\relax

\iffull
\title{quaerere atque disponere} % \ldots\ sich selbst | suchen und ordnen
\subtitle{Tutorium M\texorpdfstring{\setbox0=\hbox{a}\resizebox*!{\ht0}{\tiny{8}}}{a}cht}
\date{KW 26}
\addbibresource{references.bib}
\fi
\SetTutoriumNumber{8}

\iffull\begin{document}
\titleframe

\TopicOverview{9}
\fi

\iffull{\SummaryFrame
\begin{frame}[c]{Kurzwiederholung}
\begin{itemize}[<+(1)->]
   \itemsep8pt
    \item Die Rekursionsfamilie \info{Methoden die sich selbst aufrufen} ist riesig: \begin{itemize}
        \itemsep3pt
        \item Ruft sich eine Methode maximal einmal selbst auf, ist sie linear rekursiv.
        \item Head-Rekursiv, wenn dieser Aufruf das erste Statement ist \info{alles passiert im Aufstieg}
        \item Tail-Rekursiv, wenn dieser Aufruf das letzte Statement ist \info{alles passiert im Abstieg}
        \item Ruft sie sich auch mehrfach pro Rekursionsfall auf, ist sie verzweigt rekursiv.
    \end{itemize}
    \item Die \say{Big-O-Notation} nutzen wir zum \info{z.B.} Vergleich der Laufzeitkomplexität:\pause
    \begin{itemize}
        \itemsep3pt
        \item \(T(n) \in \O(f(n))\pause{}\iff \exists n_0 \in \mathbb{N}\: c \in \mathbb{R}^+\: \forall n \geq n_0: T(n) \leq c \cdot f(n)\)
        \item Ab einem gewissen Punkt (\(n_0\)) ist \(c \cdot f(n)\) stets größer als \(T(n)\)
    \end{itemize}
\onslide<12->{\begin{center}%
\resizebox{\linewidth}{!}{%
\setlength{\aboverulesep}{0pt}%
\setlength{\belowrulesep}{0pt}%
\setlength{\extrarowheight}{.85ex}%
\begin{tabular}{c*{8}{c}}
    \toprule
    & {\cellcolor{pingu@green!100!pingu@purple!21} \(\O(1)\)} & {\cellcolor{pingu@green!85!pingu@purple!21}\(\O(\log n)\)} & {\cellcolor{pingu@green!69!pingu@purple!21} \(\O(n)\)} & {\cellcolor{pingu@green!53!pingu@purple!21} \(\O(n\log n)\)} & {\cellcolor{pingu@green!36!pingu@purple!21} \(\O(n^2)\)} & {\cellcolor{pingu@green!19!pingu@purple!21} \(\O(n^3)\)} & {\cellcolor{pingu@green!14!pingu@purple!21} \(\O(2^n)\)} & {\cellcolor{pingu@green!0!pingu@purple!21} \(\O(n!)\)} \\[0.45ex]\midrule
    {\footnotesize Bez:} &{\footnotesize konst.} & {\footnotesize logarithm.} & {\footnotesize linear} &{\footnotesize linear log.} & {\footnotesize quadratisch} & {\footnotesize kubisch} & {\footnotesize exponentiell} & {\footnotesize faktoriell} \\
\bottomrule
\end{tabular}}\bigskip
\end{center}}
\end{itemize}
\end{frame}
}\fi

\SetNextSectionText[.6\linewidth]{Teile und Herrsche\\--- König Ludwig~XI von Frankreich}
\section{Präsenzaufgabe}
\begin{frame}[fragile,c]{Präsenzaufgabe}
\begin{aufgabe}{Mit vereinter Kraft!}
\task<2->{Sortieren Sie das folgende Array händisch absteigend mit dem Merge Sort Algorithmus. Geben Sie die Split- und die Mergephase an und begründen Sie, anhand dieser informell die \textit{worst case} Laufzeit von \(\O(n \log n)\) für den Mergesort Algorithmus.}
\onslide<3->{\begin{center}
    \([-1, 0, 9, 4, 1, 2, 3, 5]\)
\end{center}}
\end{aufgabe}
\end{frame}

\begin{frame}[c]{Ein wenig Mergesort}
    % TODO: update array :D
    \resizebox{.975\linewidth}{!}{%
    \begin{tikzpicture}[yscale=0.95,every path/.append style={line cap=round}]
        \onslide<1->{\node (a) at(0,0) {[\(-1, 0, 9, 4, 1, 2, 3, 5\)]};}
        \foreach[count=\i] \l/\x in {{-1,0,9,4}/-4,{1,2,3,5}/4}{
            \onslide<+(1)->{
                \node (b\i) at(\x,-1) {[\(\l\)]};
                \draw[-Kite] (a) -- (b\i);
            }
        }

        \foreach[count=\i] \l/\x/\p in {{-1,0}/-6/1,{9,4}/-2/1,{1,2}/2/2,{3,5}/6/2}{
            \onslide<+(1)->{
                \node (c\i) at(\x,-2) {[\(\l\)]};
                \draw[-Kite] (b\p) -- (c\i);
            }
        }

        \foreach[count=\i] \l/\x/\p in {{-1}/-7/1,{0}/-5/1,{9}/-3/2,{4}/-1/2,%
            {1}/1/3,{2}/3/3,{3}/5/4,{5}/7/4}{
            \onslide<+(1)->{
                \node (d\i) at(\x,-3) {[\(\l\)]};
                \draw[-Kite] (c\p) -- (d\i);
            }
        }

        \onslide<+(1)->{
        \foreach[count=\i] \l/\x/\p in {{-1}/-7/1,{0}/-5/2,{9}/-3/3,{4}/-1/4,%
            {1}/1/5,{2}/3/6,{3}/5/7,{5}/7/8}{
                \node[codeouthl] (e\i) at(\x,-4) {[\(\l\)]};
                \draw[densely dotted,-Kite,codeouthl] (d\p) -- (e\i);
            }
        }

        \foreach[count=\i] \l/\x/\p/\pt in {{-1,0}/-6/1/2,{4,9}/-2/3/4,{1,2}/2/5/6,{3,5}/6/7/8}{
            \onslide<+(1)->{
                \node (f\i) at(\x,-5) {[\(\l\)]};
                \ifnum\p=\pt\draw[densely dotted,gray,-Kite] (e\p) -- (f\i);\else \draw[-Kite] (e\p) -- (f\i);\draw[-Kite] (e\pt) -- (f\i); \fi
            }
        }

        \foreach[count=\i] \l/\x/\p/\pt in {{-1,0,4,9}/-4/1/2,{1,2,3,4,5}/4/3/4}{
            \onslide<+(1)->{
                \node (g\i) at(\x,-6) {[\(\l\)]};
                \ifnum\p=\pt \draw[densely dotted,gray,-Kite] (f\p) -- (g\i);\else \draw[-Kite] (f\p) -- (g\i);\draw[-Kite] (f\pt) -- (g\i); \fi
            }
        }

        \foreach[count=\i] \l/\x/\p/\pt in {{-1,0,1,2,3,4,5,9}/0/1/2}{
            \onslide<+(1)->{
                \node (h\i) at(\x,-7) {[\(\l\)]};
                \draw[-Kite] (g\p) -- (h\i); \draw[-Kite] (g\pt) -- (h\i);
            }
        }
        \onslide<+(1)->{\draw[densely dashed] (-8,-3.5) -- ++(17.25,0) node[above left] {\strut Split} node[below left] {\strut Merge};}
    \end{tikzpicture}
    }
\end{frame}

\begin{frame}{Eigenschaften unter der Lupe}
\begin{itemize}[<+(1)->]
    \itemsep6pt
    \item Mergesort ist \say{easy split, hard join}, bezeichne nun \(n\) die Eingabelänge
    \item Das Aufteilen benötigt immer \(\O(\log n)\) viele Schritte \info{wie bei der binären Suche}\infoblock<0|handout:1>{Genau genommen kommt das auf die Implementation an, nutzen wir Arrays und müssen diese jedes mal kopieren landen wir schon eher bei \(\O(n \log n)\),}
    \item Merge hat auch \(\O(\log n)\) Schritte und muss pro Schritt \(\O(n)\) Elemente vergleichen\pause \infoblock{Bei Mergesort vergleichen wir ja jeweils nur die vordersten Elemente beider Arrays, jeder Vergleich übernimmt ein Element in das Zielarray des Schrittes.}
    \item Alle anderen Operationen (Kopie der Zahl ins Ziel-Array,~\ldots) sind konstant \infoblock<0|handout:1>{Selbst, wenn sie nicht konstant wären, sind Array-Erzeugungen und Kopien alle auch maximal in \(\O(n)\) machbar.}
    \item Wir erhalten: \(\O(\log n) + \O(n) \cdot \O(\log n) \onslide<+(1)->{= \O(\max\{\log n, n \cdot \log n\}) = \O(n \log n)}\)
\end{itemize}
% LAufzeitabschätzung, über DArstelung mit Komma und easy split hard join reden!
% Die Dopplung der mittleren REiheist nicht notwendig, sie dient zur schöneren DArstellung der beiden Phasen (Ein- und Ausgabedaten usw., graue Pfeile)
% TODO: ref in-place mergesort
\end{frame}

\iffull
{\AddonFrame
\immediate\write18{wget https://media.githubusercontent.com/media/EagleoutIce/Episode-Inplace/gh-pages/preview-01.png -O logoInplace-\jobname.png}
\begin{frame}{Mehr Merge! Mehr Sort!}
    \begin{itemize}[<+(1)->]
        \item Es empfiehlt sich irgendwas zwischen die Zahlen zu setzen (wie ein Komma), sonst rutschen diese Erfahrungsgemäß (zu) eng zusammen
        \item Die Verdopplung des Schrittes bei Split und Merge dient der Übersicht, sie ist nicht notwendig.
        \item Merge-Sort gibt es in einigen Flavors. Beispielsweise in-place mergesort:\medskip\pause
\begin{center}
    \scalebox{.75}{\begin{tikzpicture}[align-base]
        \onslide<2->{\draw[thick,darkgray,rounded corners=2.5pt,path picture={\node at(path picture bounding box.center) {\href{https://media.githubusercontent.com/media/EagleoutIce/Episode-Inplace/gh-pages/noanim_inplace-merge.pdf}{\includegraphics[width=8.5cm,height=4.788cm,keepaspectratio]{logoInplace-\jobname.png}}};}] (0,0) rectangle (8.5cm,4.788cm);}
    \end{tikzpicture}}\qquad\fancyqr{https://github.com/EagleoutIce/Episode-Inplace}
\end{center}
    \end{itemize}
\end{frame}}
\fi
\SetNextSectionText{Suchen und Sortieren\\Abgabe: \DTMDate{2022-06-27}}
\section{Übungsblatt 8}
\subsection{Aufgabe 1}
{\taskenum
\begin{frame}{Aufgabe 1: Naive Sortieralgorithmen}
\task<2->{Teil von Übungsblatt~2 war der Entwurf eines naiven Sortieralgorithmus, welcher durch iteratives Vertauschen eine
gegebene Liste aufsteigend sortiert. Hier finden Sie die Algorithmen als Pseudocode nochmals aufgeführt:}
\SetAlgoVlined
\columns[onlytextwidth,c]
\scriptsize
\column{.4\linewidth}
\onslide<3->{\begin{algorithm}[H]
\PreCode
    \KwIn{List \(L = (l_1, \ldots, l_N)\), Indizes \(i\) und~\(j\)}
\StartCode
    make new \(N\) have value \(\text{length}(L)\)\;
    \If{\(N < 2\) \KwOr \(i < 1\) \KwOr \(j < 1\) \KwOr \(i > N\) \KwOr \(j > N\)}{
        print \textit{out} \say{Eingabe ungültig!}\;
        stop program\;
    }
    make new \(t\) have value \(l_i\)\;
    make \(l_i\) have value \(l_j\)\;
    make \(l_j\) have value \(t\)\;
\end{algorithm}}
\column{.6\linewidth}
\onslide<4->{\begin{algorithm}[H]
\PreCode
    \KwIn{List \(L = (l_1, \ldots, l_N)\)}
\StartCode
make new \(N\) have value \(\text{length}(L)\)\;
\lIf{\(N < 2\)}{
    print \textit{out} \say{Eingabe ungültig!} and stop program
}
make new \(i\) have value \(1\)\;
\While{\(i < N\)}{
    make new \(j\) have the value \(i + 1\)\;
    \While{\(j \leq N\)}{
        \lIf{\(l_i > l_j\)}{call \T{vertausche(L, i, j)}}
        Increment \(j\) by \(1\)\;
    }
    Increment \(i\) by \(1\)\;
}
\end{algorithm}}
\endcolumns
\begin{enumerate}
    \item<5-> \task{Implementieren Sie dieses Verfahren nun in Java. Halten Sie sich dabei genau an die Angaben aus den
    gegeben Algorithmen und testen Sie Ihre Implementierung an einem Beispiel.}
    \item<6-> \task{Für die Pseudocodevariante des Sortierverfahrens haben wir bereits eine Laufzeitabschätzung von:\columns[onlytextwidth,c]
    \column{.29\linewidth}\nomathskip
    \begin{equation*}
        T_{max}^{\text{sort}}(n) = 3 + 3n + 12 \cdot \frac{n^2 - n}{3}
    \end{equation*}
    \column{.71\linewidth}
    Elementaroperationen (s. Übungsblatt~2) angegeben. Geben Sie nun hierfür die Laufzeit in der \O-Notation an und begründen Sie ihre Antwort \textit{kurz}.
    \endcolumns}
\end{enumerate}
\end{frame}

\begin{frame}[fragile,c]{Swappsies}
\begin{plainjava}[lineskip=1.5pt]
!*\onslide<3->*!public static void vertausche(!*\Snode{list}*!!*\onslide<5->*!int[] arr!*\Snode{list@}*!,!*\onslide<4->*! int i, int j!*\onslide<3->*!) {
!*\onslide<6->*!    int n = arr.length;
!*\onslide<7->*!    if(n < 2!*\onslide<8->*! || !*\Snode{arr}*!i < 0 || j < 0 || i >= n || j >= n!*\Snode{arr@}*!!*\onslide<7->*!) {
!*\onslide<9->*!        System.err.println("Eingabe ungültig!");
!*\onslide<10->*!        return;!*\Snode{ret@}*!
!*\onslide<7->*!    }
!*\onslide<11->*!    int t = arr[i];
!*\onslide<12->*!    arr[i] = arr[j];
!*\onslide<13->*!    arr[j] = t;
!*\onslide<3->*!}!*\onslide<1->*!
\end{plainjava}
\onslide<2->{\begin{tikzpicture}[@O]
    \node[above left,yshift=\btdmfootheight,text width=6.75cm,font=\scriptsize] at(current page.south east) {\color{codeouthl}\SetAlgoVlined\begin{algorithm}[H]
\PreCode
    {\only<3-5>{\color{black}}\KwIn{List \(L = (l_1, \ldots, l_N)\), Indizes \(i\) und~\(j\)}}
\StartCode
    {\only<6>{\color{black}}make new \(N\) have value \(\text{length}(L)\)\;}
    {\only<7-8>{\color{black}}\If{\(N < 2\) \KwOr \(i < 1\) \KwOr \(j < 1\) \KwOr \(i > N\) \KwOr \(j > N\)}{
        \color{codeouthl}
        {\only<9>{\color{black}}print \textit{out} \say{Eingabe ungültig!}\;}
        {\only<10>{\color{black}}stop program\;}
    }}
    {\only<11>{\color{black}}make new \(t\) have value \(l_i\)\;}
    {\only<12>{\color{black}}make \(l_i\) have value \(l_j\)\;}
    {\only<13>{\color{black}}make \(l_j\) have value \(t\)\;}
\end{algorithm}};
\only<0|handout:1>{
    \def\hlcolor{pingu@green}
    \hlbehindcodeunder{list}{@l>}
    \draw[codeouthl,-Kite] ([xshift=-2mm,yshift=.1mm]@l>_up.north) to[out=80,in=180] ++(.5,.65) node[T,below right,yshift=.7\baselineskip,align=left] {Von Zahlen war nicht die Rede. Von Arrays auch nicht!\\Das sind (willkürlich sinnvolle) Annahmen.};
    \def\hlcolor{pingu@purple}
    \hlbehindcodeunder{arr}{@a>}
    \draw[codeouthl,-Kite] ([xshift=-1.25cm,yshift=.1mm]@a>_up.north) to[out=80,in=180] ++(.5,.25) node[T,right] {Mathematisch: \(\{1, \ldots, n\}\), Java: \(\{0, \ldots, n - 1\}\)};
    \node[above right,xshift=2mm,yshift=\btdmfootheight,T,align=left] at (current page.south west) {Ich habe mir hier die Freiheit genommen,\\\T{L} und \T{N} umzubenennen.};
    \iffull
    \draw[codeouthl,-Kite] (ret@.east) to[out=-5,in=180] ++(1.2,-.25) node[T,below right,yshift=.6\baselineskip,align=left] {Was machen wir aus \say{stop program}?\\\textbullet~\T{System.exit(1)}?\\\textbullet~Werfen einer Exception?};
    \fi
}
\end{tikzpicture}}
\end{frame}

\begin{frame}[fragile,c]{Sort it!}
\lstfs{9}
\begin{plainjava}
!*\onslide<3->*!public static void sortiere(int[] arr) {
!*\onslide<4->*!    int n = arr.length;
!*\onslide<5->*!    if (n < 2) {
!*\onslide<6->*!        System.err.println("Eingabe ungültig!");
!*\onslide<7->*!        return;
!*\onslide<5->*!    }
!*\onslide<8->*!    int i = 0;
!*\onslide<9->*!    while(i < arr.length - 1) {
!*\onslide<10->*!        int j = i + 1;
!*\onslide<11->*!        while(j < arr.length) {
!*\onslide<12->*!            if (arr[i] > arr[j]) {
!*\onslide<13->*!                vertausche(arr, i, j);
!*\onslide<12->*!            }
!*\onslide<14->*!            j++;
!*\onslide<11->*!        }
!*\onslide<15->*!        i++;
!*\onslide<9->*!    }
!*\onslide<3->*!}!*\onslide<1->*!
\end{plainjava}
\onslide<2->{\begin{tikzpicture}[@O]
    \node[above left,yshift=\btdmfootheight,text width=6cm,font=\scriptsize] at(current page.south east) {\color{codeouthl}\SetAlgoVlined\begin{algorithm}[H]
\PreCode
    {\only<3>{\color{black}}\KwIn{List \(L = (l_1, \ldots, l_N)\)}}
\StartCode
{\only<4>{\color{black}}make new \(N\) have value \(\text{length}(L)\)\;}
{\only<5>{\color{black}}\lIf{\(N < 2\)}{
    \color{codeouthl}
    {\only<6-7>{\color{black}}print \textit{out} \say{Eingabe ungültig!}} {\only<7>{\color{black}}and stop program}
}}
{\only<8>{\color{black}}make new \(i\) have value \(1\)\;}
{\only<9>{\color{black}}\While{\(i < N\)}{
    \color{codeouthl}
    {\only<10>{\color{black}}make new \(j\) have the value \(i + 1\)\;}
    {\only<11>{\color{black}}\While{\(j \leq N\)}{
        \color{codeouthl}
        {\only<12>{\color{black}}\lIf{\(l_i > l_j\)}{\color{codeouthl}{\only<13>{\color{black}}call \T{vertausche(L, i, j)}}}}
        {\only<14>{\color{black}}Increment \(j\) by \(1\)\;}
    }}
    {\only<15>{\color{black}}Increment \(i\) by \(1\)\;}
}}
\end{algorithm}};
\end{tikzpicture}}
\end{frame}

\iffull
{
    \AddonFrame
    \MakeThePinguExplainIt[text width=6.65cm,yshift=-1cm]{cap=!hide,construction helmet,glasses=!hide,eyes shiny,heart=shadeA,body type=chubby,right item angle=-52}{Hier sind auch \bjava{for}-Schleifen in Ordnung, da sie für den gezeigten Fall ja nur eine Kurzschreibweise sind.}
    \begin{frame}[fragile,c]{Sort it, for the \sout{children} loops!}
\begin{plainjava}
!*\onslide<2->*!public static void sortiere(int[] arr) {
!*\onslide<3->*!    int n = arr.length;
!*\onslide<3->*!    if (n < 2) {
!*\onslide<3->*!        System.err.println("Eingabe ungültig!");
!*\onslide<3->*!    }
!*\onslide<4->*!    for (int i = 0; i < arr.length; i++) {
!*\onslide<5->*!        for (int j = i + 1; j < arr.length; j++) {
!*\onslide<6->*!            if (arr[i] > arr[j]) {
!*\onslide<6->*!                vertausche(arr, i, j);
!*\onslide<6->*!            }
!*\onslide<5->*!        }
!*\onslide<4->*!    }
!*\onslide<2->*!}!*\onslide<1->*!
\end{plainjava}
\begin{tikzpicture}[@O]
\onslide<7->{\node[above left,xshift=5mm,scale=.8,yshift=\btdmfootheight] at(current page.south east) {\copy\pinguexplainbox};}% copy for animations
\end{tikzpicture}
    \end{frame}
}
\fi

\begin{frame}[fragile,c]{Testen an einem Beispiel}
\SetupLstHl
\begin{plainjava}
!*\CodeFileMarkerAttach<2->{NaiveSort.java}*!
!*\onslide<2->*!public class NaiveSort {
!*\onslide<3->*!   |ihl|public static void vertausche(int[] arr, int i, int j) { ... }|ihl|
!*\onslide<3->*!   |ihl|public static void sortiere(int[] arr) { ... }|ihl|

!*\onslide<3->*!   !*\ShowInTheWeb[.1mm]{https://www.online-java.com/Ry7xg6AO1m}*!
!*\onslide<4->*!   public static void main(String[] args) {
!*\onslide<5->*!      int[] arr = { 5, 8, 1, -1 };
!*\onslide<6->*!      sortiere(arr);
!*\onslide<7->*!      for (int i : arr)
!*\onslide<7->*!        System.out.println(i);
!*\onslide<4->*!   }
!*\onslide<2->*!}
\end{plainjava}
%
\end{frame}

\begin{frame}{Komplexitätsabschätzung}
    \begin{itemize}[<+(1)->]
        \itemsep9pt
        \item Nun fehlt noch die Laufzeit in \(\O\)-Notation für:
        \begin{equation*}
            T_{max}^{\text{sort}}(n) = 3 + 3n + 12 \cdot \frac{n^2 - n}{3}
        \end{equation*}
        \item Vereinfachen wir zunächst:
        \begin{equation*}
            T_{max}^{\text{sort}}(n) = 3 + 3n + 4\cdot n^2 - 4 \cdot n = 3 - 1\cdot n + 4\cdot n^2
        \end{equation*}
        \item Was Wachstumsverhalten des Ausdrucks wird von \(n^2\) dominiert, der Leitkoeffizient entfällt wie der Rest nach den Rechenregeln.
        \item Damit ist: \(T_{max}^{\text{sort}}(n) \in \O(n^2)\)
    \end{itemize}
\end{frame}
}

\subsection{Aufgabe 2}
{\taskenum
\begin{frame}[fragile,c]{Aufgabe 2: Nicht ganz so naive Sortieralgorithmen}
\task<2->{Betrachten Sie folgende Implementierung von Insertion Sort zur aufsteigenden Sortierung:}\bigskip
\lstfs{9}
\columns[onlytextwidth,c]
\column{.6\linewidth}
\begin{plainjava}
!*\onslide<3->*!public static void insertionSort(int[] arr) {
!*\onslide<3->*!    int pos, key;
!*\onslide<3->*!    for (int i = 1; i < arr.length; i++) {
!*\onslide<3->*!        pos = i - 1;
!*\onslide<3->*!        key = arr[pos + 1];
!*\onslide<3->*!        while (pos >= 0 && arr[pos] > key) {
!*\onslide<3->*!            arr[pos + 1] = arr[pos];
!*\onslide<3->*!            pos--;
!*\onslide<3->*!        }
!*\onslide<3->*!        arr[pos + 1] = key;
!*\onslide<3->*!    }
!*\onslide<3->*!}!*\onslide<1->*!
\end{plainjava}
\column{.4\linewidth}
\begin{enumerate}
    \itemsep6pt
    \item<4-> \task{Unter welchen Bedingungen tritt im obigen Sortieralgorithmus der {\sbseries best case} ein? Geben Sie die Laufzeit hierfür in \(\O\)-Notation an.}
    \item<5-> \task{Unter welchen Bedingungen tritt im obigen Sortieralgorithmus der {\sbseries worst case} ein? Geben Sie die Laufzeit hierfür in \(\O\)-Notation an.}
\end{enumerate}
\endcolumns
\end{frame}

\begin{frame}{Case-Analyse}
\begin{enumerate}
    \itemsep12pt
    \item \task{Unter welchen Bedingungen tritt im obigen Sortieralgorithmus der {\sbseries best case} ein? Geben Sie die Laufzeit hierfür in \(\O\)-Notation an.}\pause
    Ist das Array bereits \textit{aufsteigend} sortiert, läuft die äußere Schleife \(n - 1\)-mal und die innere nie, da nie \say{\bjava{arr[pos] > arr[pos + 1]}} eintritt.\medskip\pause

    Dies ist der best case mit \(\O(n)\) (der Rest ist \(\O(1)\)).
    \item \task{Unter welchen Bedingungen tritt im obigen Sortieralgorithmus der {\sbseries worst case} ein? Geben Sie die Laufzeit hierfür in \(\O\)-Notation an.}\pause
    Ist das Array \textit{absteigend} sortiert, läuft die äußere Schleife \(n - 1\) mal, und die innere über den kleinen Gauß insgesamt \(\O(n^2)\) mal, da stets \say{\bjava{arr[pos] > arr[pos + 1]}} gilt.\medskip

    Somit haben wir einen worst case mit einer Laufzeitkomplexität von \(\O(n^2)\).
\end{enumerate}
\end{frame}
}
\subsection{Aufgabe 3}
{\taskenum
\begin{frame}[c]{Aufgabe 3: Rekursives Suchen}
\task<2->{In dieser Aufgabe sollen Sie den iterativen Binary Search aus der Vorlesung in eine rekursive Variante umprogrammieren. Beantworten Sie die Fragen als Kommentar in der entsprechenden Java Datei.}
\begin{enumerate}
    \itemsep4pt
    \item<3-> \label{8.3itsk:a}\task{Implementieren Sie eine rekursive Version des Binary Search Algorithmus. Für den Fall, dass der gesuchte Wer nicht gefunden werden konnte, geben Sie eine Meldung auf die Konsole aus sowie den Index -1 zurück. Testen Sie Ihre Implementierung an einem Beispiel. Nehmen Sie an, dass die Eingaben immer sortiert sind.}
    \item<4-> \task{Ist Ihre Implementierung aus Teilaufgabe~\link{8.3itsk:a}{a)} Kopf- oder End-rekursiv? Begründen Sie Ihre Antwort kurz.}
    \item<5-> \task{Geben Sie an, unter welchen Bedingungen für Ihre Implementierung der worst-case bzgl. der Laufzeit eintritt. Begründen Sie Ihre Antwort kurz.}
    \item<6-> \task{Geben Sie die bestmögliche Laufzeitabschätzung für den worst-case in \(\O\)-Notation an. Begründen Sie Ihre Antwort kurz.}
\end{enumerate}
\end{frame}

\begin{frame}[fragile,c]{Save me, Sonic!}
\begin{plainjava}
!*\onslide<2->*!public static int binSearchRec(int[] arr, int key, int left, int right) {
!*\onslide<3->*!    if (left >= right) {
!*\onslide<4->*!        System.out.println("Wert konnte nicht gefunden werden!");
!*\onslide<5->*!        return -1;
!*\onslide<3->*!    }
!*\onslide<6->*!    int mid = (left + right) / 2;
!*\onslide<7->*!    if (arr[mid] == key) {
!*\onslide<8->*!        return mid;
!*\onslide<7->*!    } else if (arr[mid] > key) {
!*\onslide<9->*!        return binSearchRec(arr, key, left, mid - 1);
!*\onslide<8->*!    } else {
!*\onslide<10->*!        return binSearchRec(arr, key, mid + 1, right);
!*\onslide<8->*!    }
!*\onslide<2->*!}!*\onslide<1->*!
\end{plainjava}
\begin{tikzpicture}[@O]
    \onslide<11->{\node[above left,yshift=\btdmfootheight,T,xshift=-1mm] at(current page.south east) {Eine wunderschöne Tail(s)-Rekursion!};}
\end{tikzpicture}
\end{frame}
\begin{frame}[fragile,c]{Kunckles! Echidnas want to test that!}
\SetupLstHl
\begin{plainjava}
!*\CodeFileMarkerAttach<2->{BinarySearchRecursive.java}*!
!*\onslide<2->*!public class BinarySearchRecursive {
!*\onslide<3->*!   |ihl|public static int binSearchRec(int[] arr, int k, int l, int r) { ... } |ihl|
!*\onslide<2->*!
!*\onslide<4->*!   public static void main(String[] ars) {
!*\onslide<5->*!      int[] arr = { 1, 2, 4, 8, 16, 32, 64, 128, 256 };
!*\onslide<6->*!      System.out.printf("Pos: " + binSearchRec(arr, 7, 0, arr.length - 1));
!*\onslide<4->*!   }
!*\onslide<2->*!}
\end{plainjava}
\end{frame}

\begin{frame}[c]{Kopflose Analysen}
\begin{enumerate}[<+(1)->]
    \setcounter{enumi}{1}
    \itemsep12pt
    \item \task{Ist Ihre Implementierung aus Teilaufgabe~\link{8.3itsk:a}{a)} Kopf- oder End-rekursiv? Begründen Sie Ihre Antwort kurz.}\pause
    Wie bereits angemerkt liegt eine Endrekursion vor: \pause alle Berechnungen passieren im Abstieg, die rekursiven Aufrufe sind für alle rekursiven Ausführungspfade das letzte Statement.
    \item \task{Geben Sie an, unter welchen Bedingungen für Ihre Implementierung der worst-case bzgl. der Laufzeit eintritt. Begründen Sie Ihre Antwort kurz.}\pause
    Wenn der gesuchte Wert nicht im Array enthalten ist, müssen wir die meisten Vergleiche durchführen.\pause \infoblock{Nachdem wir das letzte Element betrachtet haben, gehen wir noch einmal in die Rekursion um dann festzustellen, dass es nichts mehr zu durchsuchen gibt~--- das Suchfenster ist leer.}
\end{enumerate}
\end{frame}

\begin{frame}[c]{Komplexity? Analyse!}
\begin{enumerate}[<+(1)->]
    \setcounter{enumi}{3}
    \item \task{Geben Sie die bestmögliche Laufzeitabschätzung für den worst-case in \(\O\)-Notation an. Begründen Sie Ihre Antwort kurz.} \pause
    Die Analyse bleibt zur iterativen Variante weitgehend unverändert. \pause Wir können das Array \(\log_2 n\)-mal teilen und haben somit \(\O(\log_2 n)\) rekursive Aufrufe. Da alle anderen Operationen konstant sind, landen wir somit bei: \begin{equation*}
        T_{max}^{\text{rec. bin-search}} \in \O(\log n)
    \end{equation*}
\end{enumerate}
\end{frame}
}


\iffull
{\AddonFrame
\begin{frame}{Ein tieferer Blick}
    \begin{itemize}[<+(1)->]
        \itemsep12pt
        \item Im Folgenden ein leicht modifizierter Ansatz an die binäre Suche\pause\infoblock{Aus dem letzten Semester}
        \item Wenn ein Element mehrfach enthalten ist, möchten wir nun den kleinsten Index
        \item Sonst folgen ein paar ganz brave Pingus\ldots
    \end{itemize}
\end{frame}
\newsavebox\pinguBSa \savebox\pinguBSa{\PinguBanner{-14}{strawhat}}
\newsavebox\pinguBSaa \savebox\pinguBSaa{\PinguBanner{-14}{cloak=pingu@black}}
\newsavebox\pinguBSb \savebox\pinguBSb{\PinguBanner{24}{hat,hat position={1:(0cm,-.09cm){1.33}}}}
\newsavebox\pinguBSbb \savebox\pinguBSbb{\PinguBanner{24}{cake-hat}}
\newsavebox\pinguBSc \savebox\pinguBSc{\PinguBanner{42}{halo}}
\newsavebox\pinguBSd \savebox\pinguBSd{\PinguBanner{109}{headband,pants}}
\newsavebox\pinguBSe \savebox\pinguBSe{\PinguBanner{113}{vr-headset}}
\newcommand<>\PinguBS[2]{$\underset{\onslide#3{\text{\huge\strut\T{#1}}}}{\expandafter\usebox\csname pinguBS#2\endcsname}$}
\begin{frame}[fragile]{Modifizierte Binäre Suche\hfill Statische Simulation\llap{\smash{\raisebox{-.3\baselineskip}{\scalebox{.7}{\itshape\tiny \say{Statische},\;weil\;ich\;keinen\;Platz\;mehr\;für\;static\;hatte\;:D}}}}}
\SetupLstHl\lstfs{8}\vspace*{-.5\baselineskip}\begin{columns}[c,onlytextwidth]
\column{.495\linewidth}
\begin{plainjava}
!*\onslide<2->*!!*\omd[1]{7,13}*!int search(int[] arr, int l, int r, int e) {
!*\onslide<2->*!  !*\omd{8,14}*!if(r < l) return -1; !*\rBS<handout:2|8-12>{r=6, l=0}\rBS<handout:3|14-17>{r=2, l=0}*!
!*\onslide<2->*!  !*\omd{9,15}*!int mid = l + (r - l) / 2; !*\rBS<handout:2|9-12>{mid=3}\rBS<handout:3|15-17>{mid=1}*!
!*\onslide<2->*!  !*\omd[3]{10,16}*!if (arr[mid] == e) return m!*\mb{17}*!id; !*\rBS<handout:2|10-12>{e=-14}\rBS<handout:3|16-17>{e=-14, mid=1}*!
!*\onslide<2->*!  !*\omd{11}*!else if (arr[mid] > e) !*\rBS<handout:2|11-12>{24 > -14}*!
!*\onslide<2->*!      !*\omd[2]{12}\mg[3]{13-17}*!return search(arr, l, mid - 1, e);!*\ml{18}*!
!*\onslide<2->*!  else
!*\onslide<2->*!      return search(arr, mid + 1, r, e);
!*\onslide<2->*!}!*\ml{19}*!

!*\onslide<3->*!!*\omd5*!int find(int[] arr, int e) {
!*\onslide<3->*!  !*\omd{6}\mg[1-3]{7-19}*!int idx = search(arr, 0, arr.length - 1, e);!*\ml{20}*!
!*\onslide<3->*!  !*\omd[4]{21}*!if(idx == -1) return -1; !*\rBS<handout:4|21-25>{idx=1}*!
!*\onslide<3->*!  !*\omd{22,24}*!for(int i = 1; i <= idx; i++) { !*\rBS<handout:0|22-23>{i=1, 1<=1}\rBS<handout:0|24-25>{i=2, 2<=1}*!
!*\onslide<3->*!      !*\omd{23}*!if(arr[idx - i] != e) !*\rBS<handout:0|23>{arr[0]=-14, e=-14}*!
!*\onslide<3->*!          return idx - i + 1;
!*\onslide<3->*!  }
!*\onslide<3->*!  !*\omd{25}*!return 0; !*\onslide<handout:5|25->*!//Wir sind ganz links
!*\onslide<3->*!}
\end{plainjava}
\column{.505\linewidth}
\centering\onslide<handout:1-|4->{\begin{tikzpicture}
\only<handout:3-|13-19>{\scope[transparency group,opacity=.6]}
\only<handout:4-|20->{\scope[transparency group,opacity=.3]}
    \node at (0,0) {\downsize\linewidth{\PinguBS<handout:2-|7->{left}{a}~\PinguBS{}{aa}~\PinguBS{}b~\PinguBS<handout:2-|9->{mid}{bb}~\PinguBS{}c~\PinguBS{}d~\PinguBS<handout:2-|7->{right}e}};
\only<handout:4-|20->{\endscope}
\only<handout:3-|13-19>{\endscope}
\end{tikzpicture}}\bigskip\par
\only<handout:3-|13->{\begin{tikzpicture}
    \only<handout:4-|20->{\scope[transparency group,opacity=.6]}
    \node at (0,0) {\downsize\linewidth{\PinguBS<13->{left}{a}~\PinguBS<15->{mid}{aa}~\PinguBS<13->{right}b~\PinguBS{}{bb}~\PinguBS{}c~\PinguBS{}d~\PinguBS<7->{}e}};
    \only<handout:4-|20->{\endscope}
\end{tikzpicture}}\bigskip\par
\only<handout:4-|20->{\begin{tikzpicture}
    \node at (0,0) {\downsize\linewidth{\PinguBS{}{a}~\PinguBS<20->{idx}{aa}~\PinguBS{}b~\PinguBS{}{bb}~\PinguBS{}c~\PinguBS{}d~\PinguBS<7->{}e}};
\end{tikzpicture}}
\end{columns}\vspace*{-1.15\baselineskip}
\begin{center}
    \footnotesize\onslide<4->{\mbox{\omd[0]{4}\mg[1-3]{5-25}\bjava{find(|ihl|new int[]\{-14, -14, 24, 24, 42, 109, 113\}|ihl|, -14);}}}\ml[5]{26}\smash{~~~~\onslide<handout:5|26->{~\bjava{// :yields: 0}}}
\end{center}
\end{frame}

\newsavebox\pinguTalkHSa \savebox\pinguTalkHSa{\tikz\pingu[eyes=wink, left wing wave, right wing grab,bow tie=paletteA];}
\newsavebox\pinguTalkHSb \savebox\pinguTalkHSb{\tikz\pingu[eyes=wink, left wing wave, right wing grab,tie=paletteB,shirt=pingu@red!60!pingu@black!60!pingu@white,second shirt=pingu@black,shirt above,princess crown];}
\begin{frame}[fragile]{Modifizierte Binäre Suche\hfill Stackische Heap-Simulation}
\vspace*{-.5\baselineskip}\begin{columns}[c,onlytextwidth]
\column{.495\linewidth}\SetupLstHl\lstfs{8}
\begin{plainjava}
!*\onslide<2->*!!*\omd[1]{7,13}*!int search(int[] arr, int l, int r, int e) {
!*\onslide<2->*!  !*\omd{8,14}*!if(r < l) return -1; !*\rBS<handout:2|8-12>{r=6, l=0}\rBS<handout:3|14-17>{r=2, l=0}*!
!*\onslide<2->*!  !*\omd{9,15}*!int mid = l + (r - l) / 2; !*\rBS<handout:2|9-12>{mid=3}\rBS<handout:3|15-17>{mid=1}*!
!*\onslide<2->*!  !*\omd[3]{10,16}*!if (arr[mid] == e) return m!*\mb{17}*!id; !*\rBS<handout:2|10-12>{e=-14}\rBS<handout:3|16-17>{e=-14, mid=1}*!
!*\onslide<2->*!  !*\omd{11}*!else if (arr[mid] > e) !*\rBS<handout:2|11-12>{24 > -14}*!
!*\onslide<2->*!      !*\omd[2]{12}\mg[3]{13-17}*!return search(arr, l, mid - 1, e);!*\ml{18}*!
!*\onslide<2->*!  else
!*\onslide<2->*!      return search(arr, mid + 1, r, e);
!*\onslide<2->*!}!*\ml{19}*!

!*\onslide<3->*!!*\omd5*!int find(int[] arr, int e) {
!*\onslide<3->*!  !*\omd{6}\mg[1-3]{7-19}*!int idx = search(arr, 0, arr.length - 1, e);!*\ml{20}*!
!*\onslide<3->*!  !*\omd[4]{21}*!if(idx == -1) return -1; !*\rBS<handout:4|21-25>{idx=1}*!
!*\onslide<3->*!  !*\omd{22,24}*!for(int i = 1; i <= idx; i++) { !*\rBS<handout:0|22-23>{i=1, 1<=1}\rBS<handout:0|24-25>{i=2, 2<=1}*!
!*\onslide<3->*!      !*\omd{23}*!if(arr[idx - i] != e) !*\rBS<handout:0|23>{arr[0]=-14, e=-14}*!
!*\onslide<3->*!          return idx - i + 1;
!*\onslide<3->*!  }
!*\onslide<3->*!  !*\omd{25}*!return 0; !*\onslide<handout:5|25->*!//Wir sind ganz links
!*\onslide<3->*!}
\end{plainjava}
\column{.505\linewidth}
\raggedleft\makeatletter
\lhns@elemwidth4.5cm
\lhns@minborderheight11cm
\makeatother
\onslide<5->{\raisebox{.55cm}{\downsizeBoth{5.85cm}\linewidth{\begin{tikzpicture}[lhns@basestyle/.append style={execute at begin node=\strut,font=\ttfamily},lhns@blockstyle/.append style={draw=gray,thick}]
\begin{heap-n-stack}{}
\heap{\bjava{\{-14, :ldots:, 114\}}}
\renderheap

\stack{\textbf{main}}
\stack{$\ldots$}
\only<handout:5-|26->{\stack{\say{\bjava{find = 0}}}}
\only<handout:1-4|-25>{\stack{\textbf{find}}
\stack{arr}
\stack{\bjava{e = -14}}
\only<handout:1-3|6-19>{\stack{\bjava{idx =}~\textcolor{gray}{?}}}
\only<handout:4|20->{\stack{\bjava{idx = 1}}}
\only<handout:0|22-23>{\stack{\bjava{i = 1}}}
\only<handout:0|24>{\stack{\bjava{i = 2}}}
\only<handout:1-3|7-18>{\stack{\textbf{search}}
\only<handout:1-2|-12>{\stack{arr}
\stack{\bjava{l = 0}}
\stack{\bjava{r = 6}}
\stack{\bjava{e = -14}}}
}
\only<handout:2|9-12>{\stack{\bjava{mid = 3}}}
\only<handout:3|13-17>{\stack{$\ldots$}
\stack{\textbf{search}}
\stack{arr}
\stack{\bjava{l = 0}}
\stack{\bjava{r = 2}}
\stack{\bjava{e = -14}}
}
\only<handout:3|15-17>{\stack{\bjava{mid = 1}}}
\only<handout:5|18>{\stack{$\ldots$}\stack{\say{\bjava{search = 1}}}}
\only<handout:5|19>{\stack{\say{\bjava{search = 1}}}}}
\renderstack
\only<-25>{
\only<handout:1-4>{\draw[lhns] (stack-3.east) to[out=20,in=220] (heap-0.west);}
\only<handout:1-3|7-18>{
    \draw[lhns] (stack-7.east) to[out=35,in=220] (heap-0.west);
}
\only<handout:3|13-17>{
    \draw[lhns] (stack-9.east) to[out=37,in=218] (heap-0.west);
}}
\end{heap-n-stack}
\end{tikzpicture}}}}~~
\end{columns}\vspace*{-1\baselineskip}
\begin{center}\SetupLstHl\lstfs{8}
    \footnotesize\onslide<4->{\mbox{\omd[0]{4}\mg[1-3]{5-25}\bjava{find(|ihl|new int[]\{-14, -14, 24, 24, 42, 109, 113\}|ihl|, -14);}}}\ml[5]{26}\smash{~~~~\onslide<handout:5|26->{~\bjava{// :yields: 0}}}
\end{center}
\begin{tikzpicture}[overlay,remember picture]
    \onslide<handout:1|6>{%
        \node[left=-.95cm,scale=.6] (pingu) at ([yshift=-1.5cm]current page.east) {\rotatebox[origin=c]{70}{\usebox\pinguTalkHSa}};
        \node[above left,scale=.6,text width=4.425cm,align=right,xshift=-1cm,yshift=-.325cm] at (pingu.north) {Dabei muss der \bjava{find(int[], int)}-\\Aufruf nicht \hbox{direkt} in \bjava{main} sein.};
    }
    \onslide<handout:3|13>{%
        \node[left=-.95cm,scale=.6] (pingu) at ([yshift=-1.5cm]current page.east) {\rotatebox[origin=c]{70}{\usebox\pinguTalkHSb}};
        \node[above left,scale=.6,text width=4.425cm,align=right,xshift=-.665cm,yshift=-.325cm] at (pingu.north) {Genau genommen wird da (in \bjava{...}) noch mehr abgelegt. Zum Beispiel, wo es in der aufrufenden Methode weitergeht)\ldots};
    }
\end{tikzpicture}%
\end{frame}
}
\fi

\iffull
\SetNextSectionText{Suchen und Sortieren~II\\Abgabe: \DTMDate{2022-07-04}}
\section{Aussicht: Übungsblatt 9}
\begin{frame}{Nicht viel spannendes}
    \begin{itemize}[<+(1)->]
        \itemsep5pt
        \item Weitere Hilfsmethoden sind erlaubt.
        \item Die Aufgabe mag erschlagend wirken, es genügt aber einfach nur Schritt für Schritt dem Pseudocode zu Folgen.
        \item Wer mag kann gerne Visualisierungen der Fibonacci-Suche zurate ziehen \infoblock{Oder auch gerne selbst erstellen!}
        \item Bei der dritten Aufgabe ist Code-Verständnis gefragt!
    \end{itemize}
\end{frame}

\fi

\SetNextSectionText[.55\linewidth]{Thus all sorts of sophisticated order-systems become possible, which keep successively modifying themselves and hence also the computational processes that are likewise under their control.\\--- John v. Neumann~\cite{yon1958computer}} % merge sort | https://learning.oreilly.com/library/view/art-of-computer/9780321635792/ch05e.html
\section{Abschließendes}
{\SummaryFrame
\begin{frame}[t]{Zusammenfassend}
\pause \printBibCommand
\vfill\vfill % double fill for more fraction
\begin{itemize}[<+(1)->]
    \itemsep4pt
    \item Such- und Sortierverfahren sind mit die \say{Musterkinder} in der Informatik
    \begin{itemize}
        \item Suchverfahren können durch eine vorhergehende Sortierung beschleunigt werden
        \item Verschiedene Verfahren haben verschiedene Eigenschaften
        \item Selectionsort kann beispielsweise jederzeit abgebrochen werden und liefert immer noch eine Teil-Sortierung \info{Mergesort beispielsweise nicht!}
    \end{itemize}
    \item Es ist wichtig, ein Grundverständnis für alle Such- und Sortierverfahren zu entwickeln
    \item Rekursion wird uns nicht nur beim Suchen und Sortieren begegnen\infoblock{Uuuuh, dynamische-Datenstruktur mich! Weitere Verfahren folgen nächste Woche.}
    % TODO: Tafelnummer?
\end{itemize}
\end{frame}
}

\outro{\vskip9mm\centering \onslide<2->{\begin{tikzpicture}
    \pingu[body=pingu@black,body front=pingu@black,eyes wink,bill color=pingu@black,eyes color=pingu@black,feet color=pingu@black,tie=pingu@white]
\end{tikzpicture}}\bigskip\par
\onslide<3->{You expected\ldots\ a recursion gag? Read this line again!}\par\onslide<4->{\itshape\color{gray}\footnotesize Still funny, after all those semesters\ldots}}

\iffull\end{document}\fi
