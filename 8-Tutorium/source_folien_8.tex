\InputIfFileExists{../data/global.src}\relax\relax

\iffull
\title{quaerere atque disponere} % \ldots\ sich selbst
\subtitle{Tutorium M\texorpdfstring{\setbox0=\hbox{a}\resizebox*!{\ht0}{\tiny{8}}}{a}cht}
\date{KW 26}
\addbibresource{references.bib}
\fi
\SetTutoriumNumber{8}

\iffull\begin{document}
\titleframe

\TopicOverview{9}
\fi

\iffull{\SummaryFrame
\begin{frame}[c]{Kurzwiederholung}
\begin{itemize}[<+(1)->]
   \itemsep8pt
    \item Die Rekursionsfamilie \info{Methoden die sich selbst aufrufen} ist riesig: \begin{itemize}
        \itemsep3pt
        \item Ruft sich eine Methode maximal einmal selbst auf, ist sie linear rekursiv.
        \item Head-Rekursiv, wenn dieser Aufruf das erste Statement ist \info{alles passiert im Aufstieg}
        \item Tail-Rekursiv, wenn dieser Aufruf das letzte Statement ist \info{alles passiert im Abstieg}
        \item Ruft sie sich auch mehrfach pro Rekursionsfall auf, ist sie verzweigt rekursiv.
    \end{itemize}
    \item Die \say{Big-O-Notation} nutzen wir zum \info{z.B.} Vergleich der Laufzeitkomplexität:\pause
    \begin{itemize}
        \itemsep3pt
        \item \(T(n) \in \O(f(n))\pause{}\iff \exists n_0 \in \mathbb{N}\: c \in \mathbb{R}^+\: \forall n \geq n_0: T(n) \leq c \cdot f(n)\)
        \item Ab einem gewissen Punkt (\(n_0\)) ist \(c \cdot f(n)\) stets größer als \(T(n)\)
    \end{itemize}
\begin{center}%
\resizebox{\linewidth}{!}{%
\setlength{\aboverulesep}{0pt}%
\setlength{\belowrulesep}{0pt}%
\setlength{\extrarowheight}{.85ex}%
\begin{tabular}{c*{8}{c}}
    \toprule
    & {\cellcolor{pingu@green!100!pingu@purple!21} \(\O(1)\)} & {\cellcolor{pingu@green!85!pingu@purple!21}\(\O(\log n)\)} & {\cellcolor{pingu@green!69!pingu@purple!21} \(\O(n)\)} & {\cellcolor{pingu@green!53!pingu@purple!21} \(\O(n\log n)\)} & {\cellcolor{pingu@green!36!pingu@purple!21} \(\O(n^2)\)} & {\cellcolor{pingu@green!19!pingu@purple!21} \(\O(n^3)\)} & {\cellcolor{pingu@green!14!pingu@purple!21} \(\O(2^n)\)} & {\cellcolor{pingu@green!0!pingu@purple!21} \(\O(n!)\)} \\[0.45ex]\midrule
    {\footnotesize Bez:} &{\footnotesize konst.} & {\footnotesize logarithm.} & {\footnotesize linear} &{\footnotesize linear log.} & {\footnotesize quadratisch} & {\footnotesize kubisch} & {\footnotesize exponentiell} & {\footnotesize faktoriell} \\
\bottomrule
\end{tabular}}\bigskip
\end{center}
\end{itemize}
\end{frame}
}\fi

% TODO: make sectionlink auto triggerat start of section
\SetNextSectionText[.6\linewidth]{}
\section{Präsenzaufgabe}
\begin{frame}[fragile,c]{Präsenzaufgabe}
\begin{aufgabe}{Mit vereinter Kraft!}
\task<2->{Sortieren Sie das folgende Array händisch absteigend mit dem Merge Sort Algorithmus. Geben Sie die Split- und die Mergephase an und begründen Sie, anhand dieser informell die \textit{worst case} Laufzeit von \(\O(n \log n)\) für den Mergesort Algorithmus.}
\onslide<3->{\begin{center}
    \([-1, 0, 9, 4, 1, 2, 3, 5]\)
\end{center}}
\end{aufgabe}
\end{frame}

\begin{frame}[c]{Ein wenig Mergesort}
    % TODO: update array :D
    \resizebox{.975\linewidth}{!}{%
    \begin{tikzpicture}[yscale=0.95,every path/.append style={line cap=round}]
        \onslide<1->{\node (a) at(0,0) {[\(8, 6, 3, 5, 1, 2, 7, 4\)]};}
        \foreach[count=\i] \l/\x in {{8,6,3,5}/-4,{1,2,7,4}/4}{
            \onslide<+(1)->{
                \node (b\i) at(\x,-1) {[\(\l\)]};
                \draw[-Kite] (a) -- (b\i);
            }
        }

        \foreach[count=\i] \l/\x/\p in {{8,6}/-6/1,{3,5}/-2/1,{1,2}/2/2,{7,4}/6/2}{
            \onslide<+(1)->{
                \node (c\i) at(\x,-2) {[\(\l\)]};
                \draw[-Kite] (b\p) -- (c\i);
            }
        }

        \foreach[count=\i] \l/\x/\p in {{8}/-7/1,{6}/-5/1,{3}/-3/2,{5}/-1/2,%
            {1}/1/3,{2}/3/3,{7}/5/4,{4}/7/4}{
            \onslide<+(1)->{
                \node (d\i) at(\x,-3) {[\(\l\)]};
                \draw[-Kite] (c\p) -- (d\i);
            }
        }

        \onslide<+(1)->{
        \foreach[count=\i] \l/\x/\p in {{8}/-7/1,{6}/-5/2,{3}/-3/3,{5}/-1/4,%
            {1}/1/5,{2}/3/6,{7}/5/7,{4}/7/8}{
                \node (e\i) at(\x,-4) {[\(\l\)]};
                \draw[densely dotted,-Kite,gray] (d\p) -- (e\i);
            }
        }

        \foreach[count=\i] \l/\x/\p/\pt in {{6,8}/-6/1/2,{3,5}/-2/3/4,{1,2}/2/5/6,{4,7}/6/7/8}{
            \onslide<+(1)->{
                \node (f\i) at(\x,-5) {[\(\l\)]};
                \ifnum\p=\pt\draw[densely dotted,gray,-Kite] (e\p) -- (f\i);\else \draw[-Kite] (e\p) -- (f\i);\draw[-Kite] (e\pt) -- (f\i); \fi
            }
        }

        \foreach[count=\i] \l/\x/\p/\pt in {{3,5,6,8}/-4/1/2,{1,2,4,7}/4/3/4}{
            \onslide<+(1)->{
                \node (g\i) at(\x,-6) {[\(\l\)]};
                \ifnum\p=\pt \draw[densely dotted,gray,-Kite] (f\p) -- (g\i);\else \draw[-Kite] (f\p) -- (g\i);\draw[-Kite] (f\pt) -- (g\i); \fi
            }
        }

        \foreach[count=\i] \l/\x/\p/\pt in {{1,2,3,4,5,6,7,8}/0/1/2}{
            \onslide<+(1)->{
                \node (h\i) at(\x,-7) {[\(\l\)]};
                \draw[-Kite] (g\p) -- (h\i); \draw[-Kite] (g\pt) -- (h\i);
            }
        }
        \onslide<+(1)->{\draw[densely dashed] (-8,-3.5) -- ++(17.25,0) node[above left] {\strut Divide} node[below left] {\strut Merge};}
    \end{tikzpicture}
    }
\end{frame}

\begin{frame}{Eigenschaften unter der Lupe}
LAufzeitabschätzung, über DArstelung mit Komma und easy split hard join reden!
Die Dopplung der mittleren REiheist nicht notwendig, sie dient zur schöneren DArstellung der beiden Phasen (Ein- und Ausgabedaten usw., graue Pfeile)
\end{frame}

\SetNextSectionText[.55\linewidth]{TODO}
\section{Abschließendes}
{\SummaryFrame
\begin{frame}[t]{Zusammenfassend}
\pause \printBibCommand
\vfill\vfill % double fill for more fraction
\begin{itemize}[<+(1)->]
    \itemsep4pt
    \item TODO
\end{itemize}
\end{frame}
}

\outro{\vskip9mm\centering \onslide<2->{\scalebox{1.42}{\begin{tikzpicture}
    % TODO
\end{tikzpicture}}}}

\iffull\end{document}\fi
