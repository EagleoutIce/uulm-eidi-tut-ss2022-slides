\InputIfFileExists{../data/global.src}\relax\relax

\iffull
\title{Integral refs in higher dimensions}
\subtitle{Tutorium vier}
\date{KW 21}
\addbibresource{references.bib}
\fi
\SetTutoriumNumber{4}

\iffull\begin{document}
\titleframe

\TopicOverview{4}
\fi

\iffull{\SummaryFrame
\begin{frame}[fragile,c]{Kurzwiederholung}
    \begin{itemize}[<+(1)->]
        \itemsep9.25pt
        \item Wir haben uns Methoden angesehen \begin{itemize}
            \item Methoden mit einem Rückgabetyp \say{$\neq$~\bjava{void}} müssen für jeden Pfad diesen Typ liefern
            \item Methoden haben Vor- \info{Parameter} und Nachbedingungen \info{Rückgabetyp}.
            \item Parameter und lokale Variablen sind nur in ihrer Methode sichtbar.
        \end{itemize}
        \item Für Arrays haben wir for-each
\columns[onlytextwidth,c]
\column{.45\linewidth}
\begin{plainjava}
!*\onslide<4->*!int[] arr = { 1, 9, -2, 3 };
!*\onslide<5->*!for(int elem : arr)
!*\onslide<6->*!    System.out.println(elem);!*\onslide<1->*!
\end{plainjava}
\column{.55\linewidth}
\begin{plainjava}
!*\onslide<7->*!int[] arr = new int[4];
!*\onslide<8->*!for(int i = 0; i < arr.length; i++)
!*\onslide<9->*!    System.out.println(arr[i]);
\end{plainjava}
\endcolumns
        \item<10-> Java verwaltet Daten auf dem Stack \info{Primitiv} und Heap \info{Komplex} \begin{itemize}
            \item Java kopiert, vergleicht, schützt\ldots\ Stackelemente
            \item Bei der Ablage auf dem Heap wird auf dem Stack eine Referenz\\zu diesem Element abgelegt \info{\bjava{null} für die \say{leere} Referenz}
        \end{itemize}
    \end{itemize}
    \onslide<1->
\begin{tikzpicture}[overlay,remember picture]
    \node[above left, yshift=\btdmfootheight] at (current page.south east) {TODO: STACK};
\end{tikzpicture}
\end{frame}
}\fi
\SetNextSectionText[.55\linewidth]{}
\section{Präsenzaufgabe}
\begin{frame}[fragile,c]{Präsenzaufgabe}
\begin{aufgabe}{TODO}
\onslide<1->\task<2->{Implementieren Sie die folgenden zwei Methoden, welche beide ein char-Array als Parameter übernehmen und alle
darin enthaltenen Kleinbuchstaben zu Großbuchstaben, und alle Großbuchstaben zu Kleinbuchstaben machen.} \task<3->{Sie finden die Datei auf unserer Moodle Seite \info{auf Papier genügt es, nur den benötigten Code zu schreiben}.}
\begin{plainjava}
x
\end{plainjava}
\end{aufgabe}
\end{frame}

\subsection{Kopie-Me}
\begin{frame}

\end{frame}

\outro{\vskip6mm\centering X}


\iffull\end{document}\fi
