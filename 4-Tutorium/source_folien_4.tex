\InputIfFileExists{../data/global.src}\relax\relax

\iffull
\title{Integral refs in higher dimensions}
\subtitle{Tutorium vier}
\date{KW 21}
\addbibresource{references.bib}
\fi
\SetTutoriumNumber{4}

\iffull\begin{document}
\titleframe

\TopicOverview{4}
\fi

\iffull{\SummaryFrame
\setbox\pinguA=\hbox{\scalebox{.5}{\tikzpicture
\fill[lgray,rounded corners] (0,1.65) rectangle ++(3,3.85);
\fill[lgray,rounded corners] (3.5,1.65) rectangle ++(3,3.85);
\node[ldesc,above] at (1.5,5.5) {Stack};
\node[ldesc,above] at (5,5.5) {Heap};
\node[lrel] at (1.5,5) {\ldots};
\node[lrel] at (5,5) {\ldots};
\node[lrel] at (1.5,5-.85) {\bjava{byte b = 7}};
\node[lrel] at (1.5,5-.85*2) {\bjava{char c = '?'}};
\node[lrel] (fg) at (1.5,5-.85*3) {\bjava{int[] i =\ }\raisebox{-1pt}\faGenderless};
\node[lrel] (ft) at (5,5-.85-.425) {\bjava{\{1, 9, 0\}}};
\draw[lightgray,very thick,-Kite]  (fg.east) to[out=0,in=180] (ft.west);
\endtikzpicture}}
\begin{frame}[fragile,c]{Kurzwiederholung}
    \begin{itemize}[<+(1)->]
        \itemsep9.25pt
        \item Wir haben uns Methoden angesehen \begin{itemize}
            \item Methoden mit einem Rückgabetyp \say{$\neq$~\bjava{void}} müssen für jeden Pfad diesen Typ liefern
            \item Methoden haben Vor- \info{Parameter} und Nachbedingungen \info{Rückgabetyp}
            \item Parameter und lokale Variablen sind nur in ihrer Methode sichtbar
        \end{itemize}
        \item<6-> Für Arrays haben wir for-each\vspace*{-\medskipamount}
\columns[onlytextwidth,c]
\column{.45\linewidth}
\begin{plainjava}
!*\onslide<7->*!int[] arr = { 1, 9, -2, 3 };
!*\onslide<8->*!for(int elem : arr)
!*\onslide<9->*!    System.out.println(elem);!*\onslide<1->*!
\end{plainjava}
\column{.55\linewidth}
\begin{plainjava}
!*\onslide<10->*!int[] arr = new !*\tikzmarknode{@}{\sbasic{\skB{int}[\snum{5}]}}*!;
!*\onslide<11->*!for(int i = 0; i < arr.length; i++)
!*\onslide<12->*!    System.out.println(arr[i]);!*\onslide<1->*!
\end{plainjava}
\endcolumns\medskip
        \item<14-> Java verwaltet Daten auf dem Stack \info{Primitiv} und Heap \info{Komplex} \begin{itemize}
            \item<15-> Java kopiert, vergleicht, schützt, verwirft,\ldots\ Stackelemente
            \item<16-> Bei der Ablage auf dem Heap wird auf dem Stack eine Referenz\\zu diesem Element abgelegt \info{\bjava{null} für die \say{leere} Referenz}
        \end{itemize}
    \end{itemize}
    \onslide<1->
\begin{tikzpicture}[overlay,remember picture]
    \onslide<13->{\draw[Kite-,gray] (@)++(.2,.3) to[out=50,in=230] ++(.45,.3) node[above,T] {Initialisiert automatisch mit \T{\{0, 0, 0, 0, 0\}}};}
    \onslide<17->{\node[above left, yshift=\btdmfootheight] at (current page.south east) {\copy\pinguA};}
\end{tikzpicture}
\end{frame}
}\fi
\SetNextSectionText[.55\linewidth]{}
\section{Präsenzaufgabe}
{\setbox\pinguA=\hbox{\tikz{\pingu[devil horns,eyes angry,pants=btdm@primary,wings raise]}}
\setbox\pinguB=\hbox{\tikz{\pingu[devil horns,eyes sad,pants=btdm@primary]}}
\begin{frame}[fragile,c]{Präsenzaufgabe}
\begin{aufgabe}{Flip-em till 'ya hit-em}
\only<2-10|handout:0>{Implementieren Sie die folgenden zwei Methoden, welche beide ein \bjava{char}-Array als Parameter übernehmen und alle
darin enthaltenen Kleinbuchstaben zu Großbuchstaben, und alle Großbuchstaben zu Kleinbuchstaben machen.} \only<2-10|handout:0>{Sie finden die Datei auf unserer Moodle Seite \info{auf Papier genügt es, nur den bei \say{\T{// TODO}} benötigten Code zu schreiben}.}
\begin{onlyenv}<4-10|handout:0>
\only<5>{\lstfs{9}}\only<6>{\lstfs{5}\lstset{multicols=2,lineskip=2pt}}
\only<7->{\lstfs{4}\lstset{multicols=2,lineskip=-1pt}}
\begin{plainjava}
public class Praesenzaufgabe {
    public static void flipInPlace(char[] flipBuchstaben) {
        // TODO
    }

    public static char[] flipInCopy(char[] flipBuchstaben) {
        // TODO
    }

    public static void main(String[] args) {
        String s = "DieSeR SatZ HaT EinE merKwüRdige Groß- und KleinschreibunG.";
        System.out.println(s);

        char[] flipBuchstaben = s.toCharArray();

        String inCopy = String.valueOf(flipInCopy(flipBuchstaben));
        System.out.println(inCopy);

        flipInPlace(flipBuchstaben);
        String inPlace = String.valueOf(flipBuchstaben);
        System.out.println(inPlace);
    }
}
\end{plainjava}
\only<8->{Die erste Methode soll diese Änderungen direkt im übergebenen Array vornehmen, die zweite Methode soll auf einer
Kopie des Arrays arbeiten und diese Kopie dann zurückgeben. Überlegen Sie sich, welche Vor- und Nachteile die
jeweiligen Implementierungen haben.
[TODO: Hinweise, aber das sieht hier ja niemand heheheee]}
\end{onlyenv}
\begin{onlyenv}<11->
    \begin{columns}[onlytextwidth,c]
\column{.5\linewidth}
\small \onslide<12->{Implementieren Sie beide Methoden, welche  alle im übergebenen \bjava{char}-Array enthaltenen Klein- zu Großbuchstaben, und alle Groß- zu Kleinbuchstaben machen.}\medskip

\onslide<13->{\T{flipInPlace} soll diese Änderungen direkt im übergebenen Array vornehmen, \T{flipInCopy} soll auf einer
Kopie des Arrays arbeiten und diese zurückgeben.~}\onslide<14->{Überlegen Sie sich, welche Vor- und Nachteile die
jeweiligen Implementierungen haben.}
\column{.5\linewidth}
\end{columns}
\end{onlyenv}
% TODO: Rest, TODO: devil pingu
\end{aufgabe}
\begin{tikzpicture}[overlay,remember picture]
    \only<9>{\node[yshift=-2.5cm,above,scale=1.33] at (current page.center) {\copy\pinguA};}
    \only<10>{\node[yshift=-2.5cm,above,scale=1.33] at (current page.center) {\copy\pinguB};}
\end{tikzpicture}
\end{frame}
}
\subsection{Kopie-Me}
\begin{frame}
    TODO: letztes Semester Blatt 5 im join P7
TODO: P aufgabe zahlen zu chars
TODO: clone escapade

TODO: Animationsdurchläufe Heap Stack für Referenzvariablen
\end{frame}

\outro{\vskip6mm\centering X}


\iffull\end{document}\fi
