\InputIfFileExists{../data/global.src}\relax\relax

\iffull
\title[Erstes Tutorium -- Übungsblatt 1]{I've built this. With my \textit{bare} hands!}
\subtitle{Tutorium Eins}
\date{KW 18}
\addbibresource{references.bib}
\fi
\SetTutoriumNumber{1}

\iffull\begin{document}
\titleframe

\TopicOverview{1}
\fi


\SetNextSectionText{Practice yourself, for heaven's sake in little things, and then proceed to greater.\\--- Epictetus~\cite[adapted, chp.~16]{epictetusdiscourses}}
\section{Präsenzaufgabe}
\begin{frame}[fragile]{Präsenzaufgabe}
\begin{aufgabe}{It's true, isn't it?}
\columns[c,onlytextwidth]
\column{.65\linewidth}
\begin{uncoverenv}<2->
\lstfs{10}\begin{plainjava}[aboveskip=-.65\baselineskip,belowskip=0pt]
class BoolescheAusdruecke {
   public static void main(String[] args) {

   }
}
\end{plainjava}
\end{uncoverenv}
\column{.35\linewidth}
    \onslide<3->{Erweitern Sie die Datei so, dass die benötigten Variablen initialisiert mit \say{beliebigen} Werten werden.}
    \medskip
\endcolumns
\onslide<4->{Konstruieren Sie boolesche Ausdrücke, die folgendes abprüfen:\vspace*{-.5\topsep}}
\begin{itemize}
    \item<5-> Die Zimmertemperatur (\T{temperatur}) beträgt höchstens \num{22.5} Grad Celsius.
    \item<6-> Eine Person (\T{alter}) ist nicht zwischen 13 und 18 Jahre alt.\vspace*{-.75\topsep}
\end{itemize}
\onslide<7->{Geben Sie die einzelnen Ergebnisse über \bjava{System.out.println(:lan:...:ran:)} aus.}
    \onslide<1->
\end{aufgabe}
\end{frame}


\MakeThePinguExplainIt[text width=5.5cm]{cap=!hide,vr-headset,body type=legacy,cup=!hide,lollipop left,right item angle=27}{Einen generell korrekten Datentyp gibt es selten. Diverse Bedingungen hängen vom Kontext ab.\medskip\\Die Zahlen sind erstmal beliebig.}
\begin{frame}[fragile,c]{Die Datentypen finden}
\DoAnimations
\begin{center}
    \task<2->{Zimmertemperatur höchstens \num{22.5} Grad Celsius, Alter nicht zwischen 13 und 18 Jahren.}
\end{center}
\begin{plainjava}
$2->$class BoolescheAusdruecke {
$2->$    public static void main(String[] args) {
$2->$        $8->$float $3->$temperatur = $9->$18.0f$3->$;
$2->$        $15->$short $10->$alter = $16->$15$10->$;
$18->$        // ...
$2->$    }
$2->$}$1->$
\end{plainjava}
\begin{tikzpicture}[overlay,remember picture,every path/.append style={line cap=round}]
    \onslide<4->{\node[above left,xshift=-1.75cm,yshift=1.5cm,yshift=\btdmfootheight] (t) at(current page.south east) {\strut Temperatur};}
    \onslide<5->{\node[below left=1mm,yshift=-1mm] (g) at (t.south) {\only<7->{\color{gray}}Ganzzahl}; \draw[thick,rounded corners=2pt] (t.south) -- ++(0,-.5mm) -| (g.north);}
    \onslide<6->{\node[below right=1mm,yshift=-1mm] (f) at (t.south) {Fließzahl}; \draw[thick,rounded corners=2pt] (t.south) -- ++(0,-.5mm) -| (f.north);}
    \onslide<7->{\node[below left=1mm,yshift=-1mm] (j1) at (f.south) {\solGet{keywordB}{float}};
        \node[below right=1mm,yshift=-1mm] (j2) at (f.south) {\only<8->{\color{gray}\ttfamily}\only<-7|handout:0>{\expandafter\solGet{keywordB}}{double}};
        \draw[thick,rounded corners=2pt] (f.south) -- ++(0,-.5mm) -| (j1.north);
        \draw[thick,rounded corners=2pt] (f.south) -- ++(0,-.5mm) -| (j2.north);
    }

    \onslide<11->{\node[above left,xshift=-6.1cm,yshift=1.5cm,yshift=\btdmfootheight] (t) at(current page.south east) {\strut Alter};}
    \onslide<12->{\node[below left=1mm,yshift=-1mm] (g) at (t.south) {Ganzzahl}; \draw[thick,rounded corners=2pt] (t.south) -- ++(0,-.5mm) -| (g.north);}
    \onslide<13->{\node[below right=1mm,yshift=-1mm] (f) at (t.south) {\only<14->{\color{gray}}Fließzahl}; \draw[thick,rounded corners=2pt] (t.south) -- ++(0,-.5mm) -| (f.north);}
    \onslide<14->{\node[below left=1mm,xshift=-1.15cm,yshift=-1mm] (j1) at (g.south) {\only<15->{\color{gray}\ttfamily}\only<-14|handout:0>{\expandafter\solGet{keywordB}}{byte}};
        \node[below left=1mm,xshift=2mm,yshift=-1mm] (j2) at (g.south) {\solGet{keywordB}{short}};
        \node[below right=1mm,yshift=-1mm] (j3) at (g.south) {\only<15->{\color{gray}\ttfamily}\only<-14|handout:0>{\expandafter\solGet{keywordB}}{int}};
        \node[below right=1mm,xshift=1.15cm,yshift=-1mm] (j4) at (g.south) {\only<15->{\color{gray}\ttfamily}\only<-14|handout:0>{\expandafter\solGet{keywordB}}{long}};
        \draw[thick,rounded corners=2pt] (g.south) -- ++(0,-.5mm) -| (j1.north);
        \draw[thick,rounded corners=2pt] (g.south) -- ++(0,-.5mm) -| (j2.north);
        \draw[thick,rounded corners=2pt] (g.south) -- ++(0,-.5mm) -| (j3.north);
        \draw[thick,rounded corners=2pt] (g.south) -- ++(0,-.5mm) -| (j4.north);
    }
    \onslide<17->{\node[left=-4mm,xshift=5mm,scale=.8,yshift=\btdmfootheight] at(current page.-10) {\copy\pinguexplainbox};}% copy for animations
\end{tikzpicture}
\end{frame}

\begin{frame}[fragile,c]{Boolesche Ausdrücke konstruieren}
\DoAnimations
\begin{center}
    \task<2->{Zimmertemperatur \only<5|handout:0>{\textbf}{höchstens} \num{22.5} Grad Celsius, Alter \only<11|handout:0>{\textbf}{nicht} \only<10-11|handout:0>{\textbf}{zwischen 13 und 18 Jahren}.}
\end{center}
\SetupLstHl
% star uses only
% TODO: gray it out
\def\donumber{\solGet{numbers}}
\def\dokey{\solGet{keywordB}}
\def\dokeyC{\solGet{keywordC}}
\setbox0=\hbox{\solGet{basicsstyle}{\only<8-|handout:0>{\color{codeouthl}}temperatur~\onslide<5->{<=}~\only<-7>{\expandafter\donumber}{\slshape22.5}}}%
\setbox1=\hbox{\solGet{basicsstyle}{\onslide<11->{!(}\only<-11>{\expandafter\donumber}{\slshape13} \onslide<10->{<=} alter \onslide<10->{\&\&} alter \onslide<10->{<=} \only<-11>{\expandafter\donumber}{\slshape18}\onslide<11->{)}}}%
\begin{plainjava}
$2->$class BoolescheAusdruecke {
$2->$    public static void main(String[] args) {
$2->$        $2->{\text{\solGet{basicstyle}{\only<8-13|handout:0>{\color{codeouthl}}\only<-7,14->{\expandafter\dokey}{float} temperatur = \only<-7,14->{\expandafter\donumber}{\slshape18.0}f;}}}$
        $2->{\text{\solGet{basicstyle}{\only<3-7|handout:0>{\color{codeouthl}}\only<-2,8->{\expandafter\dokey}{short} alter = \only<-2,8->{\expandafter\donumber}{\slshape15};}}}$
        $6->{\text{\solGet{basicsstyle}{\only<8-13|handout:0>{\color{codeouthl}}\only<7->{{\only<-7,14->{\expandafter\dokeyC}{System}.out.println(}}\tikzmarknode{SE-EQUALS-MAY-RISE}{\solGet{basicsstyle}{\only<8-13|handout:0>{\color{codeouthl}}temperatur <= \only<-7,14->{\expandafter\donumber}{\slshape22.5}f}}\only<7->{);}}}}$
        $12->{\text{\solGet{basicsstyle}{\only<13->{{\dokeyC{System}.out.println(}}\tikzmarknode{BRAVE-NEW-WORLD}{\solGet{basicsstyle}{!(\donumber{13} <= alter \&\& alter <= \donumber{18})}}\only<13->{);}}}}$
$2->$    }
$2->$}$1->$
\end{plainjava}
\begin{tikzpicture}[overlay,remember picture,every path/.append style={line cap=round}]
    \onslide<4-11|handout:0>{\node[above left=.5cm+\btdmfootheight] (@) at(current page.south east) {\box0};}
    \onslide<9-13|handout:0>{\node[above left=.5cm+\btdmfootheight,xshift=-5cm] (@c) at(current page.south east) {\box1};}
    \onslide<6-11|handout:0>{
        \scope
        \only<8-|handout:0>{\color{codeouthl}}
        \draw[decoration={brace,amplitude=3pt},decorate] (SE-EQUALS-MAY-RISE.south east) to  coordinate[pos=.5,yshift=-2mm] (@b) (SE-EQUALS-MAY-RISE.south west);
    \draw[-Kite] (@.north) to[out=140,in=320] (@b);
    \endscope }
    \onslide<12-13|handout:0>{
        \draw[decoration={brace,amplitude=3pt},decorate] (BRAVE-NEW-WORLD.south east) to  coordinate[pos=.5,yshift=-2mm] (@d) (BRAVE-NEW-WORLD.south west);
        \only<12|handout:0>{
            \draw[-Kite] (@c.north) to[out=100,in=280] (@d);
        }
        \only<13->{
            \draw[-Kite] (@c.north) to[out=80,in=260] (@d);
        }
    }
    \only<15->{\node[above left,yshift=\btdmfootheight] at (current page.south east) {\textattachfile{BoolescheAusdruecke.java}{BoolescheAusdruecke.java}\;};}
\end{tikzpicture}\relax
\end{frame}

\SetNextSectionText{Algorithmenentwurf und -analyse\\Abgabe: \DTMDate{2022-05-02}}
\section{Übungsblatt 1}
\subsection{Aufgabe 1}
{
\setbeamertemplate{enumerate item}{\task{\alph{enumi})}}
\begin{frame}[t]{Aufgabe 1: Aufwandsanalyse}
    \task<2->{\task{\onslide<2->{In der Vorlesung haben wir uns mit der Aufwandsanalyse von Algorithmen beschäftigt und uns angeschaut, wie viele
    Operationen im schlechtesten Fall notwendig sind, um den Algorithmus durchzuführen. Im Folgenden wollen wir uns
    nun anschauen, wie sich solche Aussagen (vereinfacht) auf Maschinenoperationen übetragen lassen.}

    \onslide<3->{Bei einer Firma fällt im Produktionsprozess täglich ein Optimierungsproblem an, für dessen Lösung eine Stunde zur
    Verfügung steht. Der Algorithmus zur Lösung des Optimierungsproblems muss bei Eingabe eines Problems der Größe \(n\) insgesamt \(100n^2\) viele Operationen durchführen.
    Bisher verwendet die Firma zur Lösung einen Mikro-Rechner mit einem Prozessortakt von \(\qty{800}{\mega\hertz}\). Dieser kann in
    jeder Sekunde \(\num{40000}\) der oben genannten Operationen berechnen.}
    \begin{enumerate}
            \item<4-> \task{Welche Problemgröße \(n\) dürfen die Probleme maximal haben, damit die Berechnung innerhalb der Frist
            durchgeführt werden kann?}
            \item<5-> \task{Nun soll auf einen schnelleren Rechner mit \(\qty{3200}{\mega\hertz}\) Taktfrequenz umgestiegen werden. Nehmen Sie an, die
            höhere Taktfrequenz übersetze sich direkt in eine entsprechend höhere Rechengeschwindigkeit. Wie groß können
            nun die Optimierungsprobleme sein?}
            \item<6-> \task{Anstelle einer CPU mit einem, soll eine CPU mit 16 Kernen (jeweils mit \(\qty{3200}{\mega\hertz}\)) eingesetzt werden.
            Glücklicherweise lässt sich das Optimierungsproblem leicht parallelisieren, so dass sich die Geschwindigkeit im
            Vergleich zu einem Kern verneunfacht. Wie groß können nun die Optimierungsprobleme sein?}
    \end{enumerate}}}
\end{frame}

\begin{frame}{Aufwandsobergrenze}
\only<handout:0>{\task<2->{Tägliches Optimierungsproblem für dessen Lösung eine Stunde zur
Verfügung steht. Der Algorithmus zur Lösung benötigt bei Eingabe eines Problems der Größe \(n\) insgesamt \(100n^2\) Operationen.
Der Mikro-Rechner mit einem Prozessortakt von \(\qty{800}{\mega\hertz}\) kann in
jeder Sekunde \(\num{40000}\) obiger Operationen berechnen.}}
\begin{enumerate}
    \item<3-> \task{Was ist die maximale Problemgröße \(n\) um in der Frist zu bleiben?}
    \onslide<4->{In der Stunde haben wir \(60 \cdot 60\) Sekunden und damit \(60 \cdot 60 \cdot \num{40000}\) Operationen. Bei Problemgröße \(n\) benötigen wir \(100n^2\) Operationen:} \shortmathskip\begin{align*}
        \onslide<5->{100n^2} &\onslide<5->{\leq 60 \cdot 60 \cdot \num{40000}}\\
        \onslide<6->{100n^2} &\onslide<6->{\leq \num{144000000}}\\
        \onslide<7->{n}      &\onslide<7->{\leq \sqrt{\num{1440000}}}\\
        \onslide<8->{n}      &\onslide<8->{\leq 1200}
     \end{align*}
     \onslide<9->{\faAngleRight~Die Problemgröße \(n\) darf \(1200\) nicht übersteigen.}
\end{enumerate}
\end{frame}

\begin{frame}{Die Macht der Taktfrequenz}
\only<handout:0>{\task<2->{Tägliches Optimierungsproblem für dessen Lösung eine Stunde zur
Verfügung steht. Der Algorithmus zur Lösung benötigt bei Eingabe eines Problems der Größe \(n\) insgesamt \(100n^2\) Operationen.
Der Mikro-Rechner mit einem Prozessortakt von \(\qty{800}{\mega\hertz}\) kann in
jeder Sekunde \(\num{40000}\) obiger Operationen berechnen.}}
\begin{enumerate}
    \setcounter{enumi}{1}
    \item<3-> \task{Angenommen, die
    höhere Taktfrequenz übersetze sich direkt in eine entsprechend höhere Rechengeschwindigkeit. Wie groß können
    die Optimierungsprobleme bei \(\qty{3200}{\mega\hertz}\) Taktfrequenz sein?}
    \onslide<4->{Bisher schaffen wir \(\num{40000}\) Operationen, nun schaffen wir \(\num{40000} \cdot \sfrac{\qty{3200}{\mega\hertz}}{\qty{800}{\mega\hertz}}\):}
    \shortmathskip\begin{align*}
        \onslide<5->{100n^2} &\onslide<5->{\leq 60 \cdot 60 \cdot \num{40000} \cdot \frac{\num{3200}}{800}}\\
        \onslide<6->{100n^2} &\onslide<6->{\leq \num{576000000}}\\
        \onslide<7->{n}      &\onslide<7->{\leq \sqrt{\num{5760000}}}\\
        \onslide<8->{n}      &\onslide<8->{\leq 2400}
     \end{align*}
     \onslide<9->{\faAngleRight~Die Problemgröße \(n\) darf nun \(2400\) nicht übersteigen.}
     \onslide<10->{\infoblock{Ein vervierfachen der Leistung verdoppelt die mögliche Problemgröße aufgrund des quadratischen Wachstums.}}
\end{enumerate}
\end{frame}

\begin{frame}{Parallelisierungskosten}
    \only<handout:0>{\task<2->{Tägliches Optimierungsproblem für dessen Lösung eine Stunde zur
    Verfügung steht. Der Algorithmus zur Lösung benötigt bei Eingabe eines Problems der Größe \(n\) insgesamt \(100n^2\) Operationen.
    Der Mikro-Rechner mit einem Prozessortakt von \(\qty{800}{\mega\hertz}\) kann in
    jeder Sekunde \(\num{40000}\) obiger Operationen berechnen.}}
    \begin{enumerate}
        \setcounter{enumi}{2}
        \item<3-> \task{Anstelle einer CPU mit einem, soll eine CPU mit 16 Kernen (jeweils mit \(\qty{3200}{\mega\hertz}\)) eingesetzt werden.
        Glücklicherweise lässt sich das Optimierungsproblem leicht parallelisieren, so dass sich die Geschwindigkeit im
        Vergleich zu einem Kern verneunfacht. Wie groß können nun die Optimierungsprobleme sein?}
        \onslide<4->{Die Anzahl der Operationen verneunfacht sich direkt:}
        \shortmathskip\begin{align*}
            \onslide<5->{100n^2} &\onslide<5->{\leq 60 \cdot 60 \cdot \num{40000} \cdot \frac{\num{3200}}{800} \cdot 9} \\
            \onslide<6->{100n^2} &\onslide<6->{\leq \num{5184000000}}\\
            \onslide<7->{n}      &\onslide<7->{\leq \sqrt{\num{51840000}}}\\
            \onslide<8->{n}      &\onslide<8->{\leq 7200}
            \end{align*}
            \onslide<9->{\faAngleRight~Die Problemgröße \(n\) darf nun \(7200\) nicht übersteigen.}
    \end{enumerate}
\end{frame}

\iffull
{\AddonFrame
\begin{frame}{Laufzeitabschätzungen}
    \begin{itemize}[<+(1)->]
        \itemsep16pt
        \item Wir bekommen in der Regel \info{leider} nichts geschenkt.
        \begin{itemize}
            \itemsep3.5pt
            \item Parallelisierung ist beispielsweise mit Kosten verbunden.
            \item So müssen wir die parallelen Abläufe beispielsweise koordinieren und abgleichen.
            \item Je nach Optimierung und Problem kann mehr Parallelisierung sogar schädlich sein!
            \item Auch kann man nie allen Code parallelisieren (Initialisierungen, Ausgaben,~\ldots).
            \item Eine pessimistische Abschätzung liefert das \say{Amdahlsche Gesetz}~\info{\cite{amdahl1967amdahl}}.
        \end{itemize}
        \item Für die Skalierung hat \(n^2\) einen viel größeren Einfluss, als der Faktor \(100\) \begin{itemize}
            \itemsep3.5pt
            \item So vervielfachen 16 CPUs mit je vierfacher Leistung die Performanz um lediglich \(6\)!
            \item Meist interessiert uns nur das Wachstumsverhalten (polynomiell, exponentiell,~\ldots)
        \end{itemize}
    \end{itemize}
\end{frame}}
\fi
}

\subsection{Aufgabe 2}
\begin{frame}{Aufgabe 2: Korrektheit von Algorithmen}
\parallelcontent[c]{
\task<2->{Betrachten Sie die Anweisungsfolge. Handelt es sich hierbei um einen Algorithmus? Prüfen Sie \emph{alle} notwendigen Voraussetzungen.}
}{% fake the design :)
\task<3->{\begin{tabular}{r<{:}ll}
    1 & Setze & \(x = 1\) \\
    2 & \textbf{solange} & \(x \geq 0\) \textbf{wiederhole}\\
    3 & & Verdopple \(x\) \\
    4 & & Verringere \(x\) um \(1\) \\
    5 & \textbf{ende} & \\
\end{tabular}}
}
TODO: NÖ, aber alle zeigen!
\end{frame}

\SetNextSectionText{TODO: quote}
\section{Abschließendes}
{\SummaryFrame
\begin{frame}[t]{Zusammenfassend}
\pause \printBibCommand
\vfill\vfill % double fill for more fraction
\begin{itemize}[<+(1)->]
    \itemsep14pt
    \item TODO
\end{itemize}
\end{frame}
}

\outro{\vskip6mm\centering\begin{tikzpicture}[scale=2.5]
    \only<2->{TODO}
\end{tikzpicture}}

\iffull\end{document}\fi
