\InputIfFileExists{../data/global.src}\relax\relax

\iffull
\title[Erstes Tutorium -- Übungsblatt 1]{I've built this. With my bare hands!}
\subtitle{Tutorium Eins}
\date{KW 18}
\addbibresource{references.bib}
\fi
\SetTutoriumNumber{1}

\iffull\begin{document}
\titleframe

\TopicOverview{1}
\fi


\SetNextSectionText{Practice yourself, for heaven's sake in little things, and then proceed to greater.\\--- Epictetus~\cite[adapted, chp.~16]{epictetusdiscourses}}
\section{Präsenzaufgabe}
\begin{frame}[c]{Präsenzaufgabe}
\begin{aufgabe}{Whats that number?}

    \onslide<1->
\end{aufgabe}
\end{frame}



\SetNextSectionText{He who chooses the beginning of a road chooses the place it leads to. It is the means that determine the end.\\--- Harry Emerson Fosdick, \cite[p.~111]{fosdick1941living}}
\section{Abschließendes}
{\SummaryFrame
\begin{frame}[t]{Zusammenfassend}
\pause \printBibCommand
\vfill\vfill % double fill for more fraction
\begin{itemize}[<+(1)->]
    \itemsep14pt
    \item Für Pseudocode wünschen wir uns \begin{itemize}
        \itemsep1pt
        \item eine konsistente und menschenlesbare Notation,
        \item mathematische Definitionen,
        \item aber nichts sprachspezifisches (\bjava{int}) oder zu allgemeines (\say{löst Problem}).
    \end{itemize}
    \item Algorithmen und deren Konstruktion: \begin{itemize}
        \itemsep1pt
        \item Eine eindeutige, endliche Beschreibung wohldefinierter Elementaroperationen, deren schrittweise Ausführung durch einen Prozessor möglich und endlich ist.
        \item \def\t{~~\faAngleRight~~}Spezifikation\t Abstraktion\t Entwurf\t Verifikation\t Aufwandsanalyse
        % \item Als Frage für die Woche: Warum und wann sind all diese Schritte von Bedeutung?
    \end{itemize}
\end{itemize}
\end{frame}
}

\outro{\vskip6mm\centering\begin{tikzpicture}[scale=2.5]
    \only<2->{\pingu[right wing grab,headband=purple!90!green,cup=purple!90!green,name=saphira, left eye wink,body type=chubby]}
\end{tikzpicture}}

\iffull\end{document}\fi
