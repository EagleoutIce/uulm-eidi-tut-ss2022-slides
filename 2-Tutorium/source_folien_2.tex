\InputIfFileExists{../data/global.src}\relax\relax

\iffull
\titlesuffix{Jeah}
\title[Erstes Tutorium -- Übungsblatt 2]{\texorpdfstring{Diese Worte \tikz[baseline=(x.base)]{\node[font=\usebeamerfont{title},inner xsep=0pt] (x) {getauscht}} ich habe}{Diese Worte getauscht ich habe}}
\subtitle{Tutorium Zwei}
\date{KW 19}
\addbibresource{references.bib}
\fi
\SetTutoriumNumber{2}

\iffull\begin{document}
\titleframe

\TopicOverview{2}
\fi


% \SetNextSectionText{Practice yourself, for heaven's sake in little things, and then proceed to greater.\\--- Epictetus~\cite[adapted, chp.~16]{epictetusdiscourses}}
\section{Präsenzaufgabe}
\begin{frame}[fragile]{Präsenzaufgabe}
\begin{aufgabe}{It's true, isn't it?}
\end{aufgabe}
\end{frame}

\section{Abschließendes}
{\SummaryFrame
\begin{frame}[t]{Zusammenfassend}
\pause \printBibCommand
\vfill\vfill % double fill for more fraction
\begin{itemize}[<+(1)->]
    \itemsep14pt
    \item Wir haben die Geburt von Variablen sowie ihre Wachstumsphase kennengelernt: \begin{itemize}
        \item Die \textit{Deklaration} (\bjava{int x}) reserviert einen Namen samt Charakteristika (Typ,~\ldots)
        \item Die \textit{Initialisierung} (\bjava{int x; x = 5}) beschreibt die erste Wertzuweisung
        \item Eine \textit{(Wert-)Zuweisung} (\bjava{x = 3}) ist jede weitere Änderung des gespeicherten Wertes
    \end{itemize}
    \item In die Wahl des \say{richtigen\texttrademark} Datentyps fließen viele Informationen mit ein.
\end{itemize}
\end{frame}
}

\outro{\vskip6mm\centering\begin{tikzpicture}[scale=2.5]
    \only<2->{\pingu[monocle left,right eye wink,left eye vertical,laptop right,cane left,tie,body type=legacy,headphones=pingu@green!80!pingu@black]}
\end{tikzpicture}}


\iffull\end{document}\fi
