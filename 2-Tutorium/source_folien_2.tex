\InputIfFileExists{../data/global.src}\relax\relax

\iffull
\definecolor{@joda}{RGB}{204, 186, 157}
\definecolor{@joda@}{RGB}{78, 68, 66}
\setbox\pinguA=\hbox{\tikz\pingu[body=pingu@green!52!pingu@black,body front=pingu@green!52!pingu@black!5!pingu@white,cloak=@joda,cloak cap=@joda,lightsaber right=pingu@green,lightsaber right length=1.4cm,shirt=@joda@,shirt no buttons,small,eyes wink,cane left=pingu@bronze!80!pingu@black,cane left length=11mm,cane left raise=1mm,right item angle=-30];}
\titlesuffix{\tikzpicture[overlay,remember picture]
    \node[above left,yshift=\btdmfootheight-1mm] at(current page.south east) {\copy\pinguA};
\endtikzpicture}
\def\packtitlenode#1#2{\node[font=\usebeamerfont{title},inner xsep=0pt,opacity=#1,outer ysep=2pt] (#2) {getauscht};}
\title[Erstes Tutorium -- Übungsblatt 2]{\texorpdfstring{Diese Worte \tikz[baseline=(k.base)]{%
\packtitlenode0k
\scope[yshift=4pt,xshift=-3.75pt]\packtitlenode0x
\clip (x.south west)--(x.300)--(x.120)-| cycle;
\packtitlenode1x
\endscope
\scope[yshift=-4pt,xshift=3.75pt]
\packtitlenode0y
\clip (y.south east)--(y.300)--(y.120)-| cycle;
\packtitlenode1y
\endscope
% cutline
\filldraw[paletteA,rounded corners=1pt] (y.300)--++(-3.75pt,0)--(x.120)--++(3.75pt,0)--cycle;
} ich habe}{Diese Worte getauscht ich habe}}
\subtitle{Tutorium Zwei}
\date{KW 19}
\addbibresource{references.bib}
\fi
\SetTutoriumNumber{2}

\iffull\begin{document}
\titleframe

\TopicOverview{2}
\fi


% \SetNextSectionText{Practice yourself, for heaven's sake in little things, and then proceed to greater.\\--- Epictetus~\cite[adapted, chp.~16]{epictetusdiscourses}}
\section{Präsenzaufgabe}
\begin{frame}[fragile]{Präsenzaufgabe}
\begin{aufgabe}{It's true, isn't it?}
\end{aufgabe}
\end{frame}

\section{Abschließendes}
{\SummaryFrame
\begin{frame}[t]{Zusammenfassend}
\pause \printBibCommand
\vfill\vfill % double fill for more fraction
\begin{itemize}[<+(1)->]
    \itemsep14pt
    \item Wir haben die Geburt von Variablen sowie ihre Wachstumsphase kennengelernt: \begin{itemize}
        \item Die \textit{Deklaration} (\bjava{int x}) reserviert einen Namen samt Charakteristika (Typ,~\ldots)
        \item Die \textit{Initialisierung} (\bjava{int x; x = 5}) beschreibt die erste Wertzuweisung
        \item Eine \textit{(Wert-)Zuweisung} (\bjava{x = 3}) ist jede weitere Änderung des gespeicherten Wertes
    \end{itemize}
    \item In die Wahl des \say{richtigen\texttrademark} Datentyps fließen viele Informationen mit ein.
\end{itemize}
\end{frame}
}

\outro{\vskip6mm\centering\begin{tikzpicture}[scale=2.5]
    \only<2->{\pingu[monocle left,right eye wink,left eye vertical,laptop right,cane left,tie,body type=legacy,headphones=pingu@green!80!pingu@black]}
\end{tikzpicture}}


\iffull\end{document}\fi
