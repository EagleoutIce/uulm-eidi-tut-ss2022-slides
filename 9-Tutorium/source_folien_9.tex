\InputIfFileExists{../data/global.tex}\relax\relax

\iffull
\title{Fifo Heaps} % \ldots\ sich selbst | suchen und ordnen
\subtitle{Tutorium Neun}
\date{KW 27}
\addbibresource{references.bib}
\fi
\SetTutoriumNumber{9}

\iffull\begin{document}
\titleframe

\TopicOverview{6}
\fi

\iffull{\SummaryFrame
\begin{frame}[c]{Kurzwiederholung}
\begin{itemize}[<+(1)->]
   \itemsep12pt
    \item Wir kennen eine ganze Bandbreite an Sortierverfahren: \begin{description}[Selectionsort]
        \itemsep3pt
        \item[Bubblesort] Vertausche benachbarte Elemente, solange nicht in Sortierreihenfolge.
        \item[Insertionsort] Sortiere erstes unsortiertes Element in sortierten Teil ein.
        \item[Selectionsort] Setze kleinsten unsortierten Elements ans Ende des sortierten Teils.
        \item[Mergesort] Aufteilen bis einelementig, wiederholtes mergen sortierter Teillisten.
        \item[Quicksort] Pivot ans Ende, \(l\) solange \(<\), \(r\) solange \(\geq\). Bei treffen, tausche Pivot.
        \item[Heapsort] Baue Heap (heapify, \(\text{Eltern} \leq \text Kinder\)), entferne Wurzel und heapify.
    \end{description}
    \item Zusätzlich zwei Suchverfahren: \begin{description}[Lineare Suche]
        \itemsep3pt
        \item[Lineare Suche] Betrachte alle Elemente und vergleiche mit Suchschlüssel.
        \item[Binäre Suche] Sortierte Liste, wenn Mitte \(\neq\), prüfe wiederholt linke/rechte Hälfte.
    \end{description}
\end{itemize}
\end{frame}
}\fi

% \SetNextSectionText[.6\linewidth]{TODO}
\section{Präsenzaufgabe}
{\tikzset{W/.style={draw, rounded corners=2pt},K/.style={W, rectangle split, rectangle split parts=2,rectangle split horizontal}}
\begin{frame}[fragile,c]{Präsenzaufgabe}
\begin{aufgabe}{FIFO-Freuden}
\only<-3|handout:0>{\only<3->{\footnotesize}\onslide<2->{In dieser Aufgabe sollen Sie Ihre eigene dynamische Datenstruktur implementieren. Hierbei soll es sich um eine Queue
handeln, welche nach dem FIFO-Prinzip (First In~--- First Out) arbeitet, und \bjava{int} Werte speichert. Intern benötigen
Sie hier zusätzlich zur Klasse Queue eine (innere) Klasse um die gespeicherten Elemente repräsentieren zu können.
Hierfür können Sie die Vorlage \T{Element.java} auf Moodle verwenden, sowie die Vorlage \T{Queue.java}, die bereits das Grundgerüst enthält. Implementieren Sie nun zusätzlich die folgenden Methoden der Klasse \T{Queue}:\vspace*{-2mm} \begin{itemize}
    \itemsep-3pt
    \item Die Methode \bjava{public void enqueue(int value)} soll ein neues Element mit dem übergebenen wert am Ende der Queue einreihen.
    \item Die Methode \bjava{public boolean dequeue()} soll das vorderste Element aus der Queue entfernen.\vspace*{-2mm}
\end{itemize}
    Sie finden Beispiele für die Operation in Abbildung. Verwenden Sie für Ihre Implementierung keine Arrays oder vorgefertigte dynamische Datenstrukturen, sondern implementieren Sie die Queue selbst als (einfach) verkettete Liste.}}%
\only<4->{\vspace*{-\baselineskip}\par Implementieren Sie eine Queue, die \bjava{int} Werte speichert. Verwenden Sie dazu \T{\FileMarkerAttach<4->{Element.java}} und \T{\FileMarkerAttach<4->{Queue.java}} als Grundgerüst und implementieren Sie:\vspace*{-\smallskipamount} \begin{itemize}
    \item<5-> \bjava{void enqueue(int value)}, welche den Wert ans Ende der Queue einreiht\smallskip\par
    \onslide<6->{\begin{tikzpicture}[align-half-base]
        \node[K] (a) at (0,0) {4\nodepart{two}};
        \node[K,right=2mm] (b) at (a.east) {5\nodepart{two}};
        \node[K,right=2mm] (c) at (b.east) {1\nodepart{two}};
        \node[K,right=2mm] (d) at (c.east) {3\nodepart{two}};
        \node[W,right=2mm] (e) at (d.east) {\phantom{2}};
        \pgfinterruptboundingbox
        \node[below,T] at (a.south) {head};
        \node[below,T] at (d.south) {tail};
        \endpgfinterruptboundingbox
        \draw[shorten <= .4mm,shorten >= .4mm,] (e.south west) -- (e.north east);
        \draw[Circle-Kite] ([xshift=-1.65ex]a.east) -- (b.west);
        \draw[Circle-Kite] ([xshift=-1.65ex]b.east) -- (c.west);
        \draw[Circle-Kite] ([xshift=-1.65ex]c.east) -- (d.west);
        \draw[Circle-Kite] ([xshift=-1.65ex]d.east) -- (e.west);
    \end{tikzpicture}}~\onslide<7->{\(\overset{\T{enqueue(2)}}{\longrightarrow}\) \begin{tikzpicture}[align-half-base]
        \node[K] (a) at (0,0) {4\nodepart{two}};
        \node[K,right=2mm] (b) at (a.east) {5\nodepart{two}};
        \node[K,right=2mm] (c) at (b.east) {1\nodepart{two}};
        \node[K,right=2mm] (d) at (c.east) {3\nodepart{two}};
        \node[K,right=2mm] (e) at (d.east) {2\nodepart{two}};
        \node[W,right=2mm] (f) at (e.east) {\phantom{2}};
        \pgfinterruptboundingbox
        \node[below,T] at (a.south) {head};
        \node[below,T] at (e.south) {tail};
        \endpgfinterruptboundingbox
        \draw[shorten <= .4mm,shorten >= .4mm,] (f.south west) -- (f.north east);
        \draw[Circle-Kite] ([xshift=-1.65ex]a.east) -- (b.west);
        \draw[Circle-Kite] ([xshift=-1.65ex]b.east) -- (c.west);
        \draw[Circle-Kite] ([xshift=-1.65ex]c.east) -- (d.west);
        \draw[Circle-Kite] ([xshift=-1.65ex]d.east) -- (e.west);
        \draw[Circle-Kite] ([xshift=-1.65ex]e.east) -- (f.west);
    \end{tikzpicture}}\bigskip
    \item<8-> \bjava{boolean dequeue()}, welche den vordersten Wert entfernt % (\bjava{false} wenn keiner)
    \smallskip\par
    \onslide<9->{\begin{tikzpicture}[align-half-base]
        \node[K] (a) at (0,0) {4\nodepart{two}};
        \node[K,right=2mm] (b) at (a.east) {5\nodepart{two}};
        \node[K,right=2mm] (c) at (b.east) {1\nodepart{two}};
        \node[K,right=2mm] (d) at (c.east) {3\nodepart{two}};
        \node[K,right=2mm] (e) at (d.east) {2\nodepart{two}};
        \node[W,right=2mm] (f) at (e.east) {\phantom{2}};
        \pgfinterruptboundingbox
        \node[below,T] at (a.south) {head};
        \node[below,T] at (e.south) {tail};
        \endpgfinterruptboundingbox
        \draw[shorten <= .4mm,shorten >= .4mm,] (f.south west) -- (f.north east);
        \draw[Circle-Kite] ([xshift=-1.65ex]a.east) -- (b.west);
        \draw[Circle-Kite] ([xshift=-1.65ex]b.east) -- (c.west);
        \draw[Circle-Kite] ([xshift=-1.65ex]c.east) -- (d.west);
        \draw[Circle-Kite] ([xshift=-1.65ex]d.east) -- (e.west);
        \draw[Circle-Kite] ([xshift=-1.65ex]e.east) -- (f.west);
    \end{tikzpicture}}~\onslide<10->{\(\overset{\T{\setbox0=\hbox{enqueue(2)}\rlap{\kern.5\wd0\clap{\T{dequeue()}}}\phantom{\copy0}}}{\longrightarrow}\) \begin{tikzpicture}[align-half-base]
        \node[K,right=2mm] (b) at (0,0) {5\nodepart{two}};
        \node[K,right=2mm] (c) at (b.east) {1\nodepart{two}};
        \node[K,right=2mm] (d) at (c.east) {3\nodepart{two}};
        \node[K,right=2mm] (e) at (d.east) {2\nodepart{two}};
        \node[W,right=2mm] (f) at (e.east) {\phantom{2}};
        \pgfinterruptboundingbox
        \node[below,T] at (b.south) {head};
        \node[below,T] at (e.south) {tail};
        \endpgfinterruptboundingbox
        \draw[shorten <= .4mm,shorten >= .4mm,] (f.south west) -- (f.north east);
        \draw[Circle-Kite] ([xshift=-1.65ex]b.east) -- (c.west);
        \draw[Circle-Kite] ([xshift=-1.65ex]c.east) -- (d.west);
        \draw[Circle-Kite] ([xshift=-1.65ex]d.east) -- (e.west);
        \draw[Circle-Kite] ([xshift=-1.65ex]e.east) -- (f.west);
    \end{tikzpicture}}\medskip
\end{itemize}\onslide<11->{Verwenden Sie keine Arrays oder vorgefertigte dynamische Datenstrukturen, sondern eine eigene, (einfach) verkettete Liste.\vspace*{-\medskipamount}}}
\end{aufgabe}
\end{frame}
}

\begin{frame}[c,fragile]{Adding something to a Queue}
\SetupLstHl
\begin{onlyenv}<2|handout:0>
\begin{plainjava}
public class Queue {
    private Element first;
    private Element last;
    private int length;
    // ...

    public void enqueue(int value) {
        // TODO
    }
}
\end{plainjava}
\end{onlyenv}
\begin{onlyenv}<3->
    \begin{plainjava}
|ihl|public class Queue {|ihl|
    |ihl|private Element first;|ihl|
    |ihl|private Element last;|ihl|
    |ihl|private int length;|ihl|

    public void enqueue(int value) {
!*\onslide<4->*!        Element next = new Element(value);
!*\onslide<5->*!        if(length == 0)!*\Snode{check-empty}*!
!*\onslide<7->*!            this.first = next;
!*\onslide<5->*!        else
!*\onslide<8->*!            this.last.setNextElement(next);
!*\onslide<9->*!        this.last = next;!*\Snode{shift-tail}*!
!*\onslide<11->*!        length++;!*\onslide<1->*!
    }
|ihl|}|ihl|
\end{plainjava}
\end{onlyenv}
\begin{tikzpicture}[@O]
    \onslide<6->{\node[T,right=3mm] at(check-empty.east) {Spezialfall beim Einfügen: Isse leer?};}
    \onslide<10->{\node[T,right=3mm] at(shift-tail.east) {In jedem Fall: Verschiebe den Tail};}
\end{tikzpicture}
\end{frame}

\begin{frame}[c,fragile]{Moving out}
\SetupLstHl
\begin{plainjava}
!*\CodeFileMarkerAttach<9->[p-solution/]{Queue.java}*!
|ihl|public class Queue {|ihl|
    |ihl|private Element first;|ihl|
    |ihl|private Element last;|ihl|
    |ihl|private int length;|ihl|
    |ihl|public void enqueue(int value) { ... }|ihl|

!*\onslide<2->*!    public boolean dequeue() {
!*\onslide<3->*!        if(length > 0) {
!*\onslide<4->*!            this.first = this.first.getNextElement();
!*\onslide<5->*!            length--;
!*\onslide<6->*!            return true;!*\Snode{do-true}*!
!*\onslide<3->*!        }
!*\onslide<8->*!        return false;
!*\onslide<2->*!    }!*\onslide<1->*!
|ihl|}|ihl|
\end{plainjava}
\begin{tikzpicture}[@O]
    \onslide<7->{\node[T,right=3mm] at(do-true.east) {Es gab etwas zu Entfernen!};}
\end{tikzpicture}
\end{frame}

\SetNextSectionText{Suchen und Sortieren~II\\Abgabe: \DTMDate{2022-07-04}}
\section{Übungsblatt 9}
\subsection{Aufgabe 1}
{\taskenum
\begin{frame}[c]{Aufgabe 1: Heap Sort}
    \taskblock<2->In dieser Aufgabe sollen Sie das folgende Array händisch aufsteigend mit dem Heap Sort Algorithmus sortieren: \begin{center}
    \([0, -1, 9, 4]\)
\end{center}
\begin{enumerate}
    \item Phase 1: Geben Sie den schrittweise den entsprechenden Heap an und markieren Sie jeweils wo die Heap-
    Eigenschaft verletzt ist. Orientieren Sie sich dabei an der Darstellung aus der Vorlesung.
    \item Phase 2: Wandeln Sie den Heap aus Phase 1 schrittweise in ein sortiertes Array um. Markieren Sie jeweils wo nach dem Herausnehmen des \textit{Head} die Heap-Eigenschaft verletzt wurde und das entsprechende Resultat der \textit{Heapify} Operation.
\end{enumerate}
\endtaskblock
\end{frame}
{
\MakeThePinguExplainIt[text width=6.65cm,yshift=-1.15cm]{cap=!hide,hat,hat coronal=paletteA,hat ribbon=paletteA,glasses=!hide,eyes wink,cup=!hide,heart=shadeA,right item angle=-50}{Hier haben wir einen Min-Heap, ein Max-Heap würde genau so funktionieren!}
\begin{frame}{Der Aufbau eines Heaps}
    \begin{enumerate}
        \item<2-> \task{Geben Sie den schrittweise den Heap an und markieren Sie jeweils wo die Heap-Eigenschaft verletzt ist.
        \begin{center}
            \([0, -1, 9, 4]\)
        \end{center}}
    \end{enumerate}\vfill
\tikzset{O/.style={circle,draw,minimum width=2em},every node/.append style={O}}
% no forest for align
\centering\downsize\linewidth{\onslide<3->{\begin{tikzpicture}[align-half-base,xscale=0.75]
    \node (0) at (0,0) {0};
    \node[minimum size=.25mm] (1) at (-1,-1) {};
    \draw (0) -- (1);

    \node[draw=none] at(0,-2) {};
\end{tikzpicture}}\onslide<4->{~~\(\longrightarrow\)~~%
\begin{tikzpicture}[align-half-base,xscale=0.75]
    \node (0) at (0,0) {0};
    \node (1) at (-1,-1) {-1};
    \node[minimum size=.25mm] (2) at ( 1,-1) {};
    \draw[thick] (0) -- (1) coordinate[pos=.5] (@);
    \pgfinterruptboundingbox
    \draw[rotate=42,thick] (@) ellipse[x radius=1.5cm, y radius=7.5mm];
    \endpgfinterruptboundingbox
    \draw (0) -- (2);

    \node[draw=none] at(0,-2) {};
\end{tikzpicture}}\onslide<5->{~~\(\longrightarrow\)~~%
\begin{tikzpicture}[align-half-base,xscale=0.75]
    \node (0) at (0,0) {-1};
    \node (1) at (-1,-1) {0};
    \node[minimum size=.25mm] (2) at ( 1,-1) {};
    \draw (0) -- (1);
    \draw (0) -- (2);

    \node[draw=none] at(0,-2) {};
\end{tikzpicture}}\onslide<6->{~~\(\longrightarrow\)~~%
\begin{tikzpicture}[align-half-base,xscale=0.75]
    \node (0) at (0,0) {-1};
    \node (1) at (-1,-1) {0};
    \node (2) at ( 1,-1) {9};
    \node[minimum size=.25mm] (3) at (-2,-2) {};
    \draw (0) -- (1);
    \draw (0) -- (2);
    \draw (1) -- (3);

    \node[draw=none] at(0,-2) {};
\end{tikzpicture}}\onslide<7->{~~\(\longrightarrow\)~~%
\begin{tikzpicture}[align-half-base,xscale=0.75]
    \node (0) at (0,0) {-1};
    \node (1) at (-1,-1) {0};
    \node (2) at ( 1,-1) {9};
    \node (3) at (-2,-2) {4};
    \node[minimum size=.25mm] (4) at (0,-2) {};
    \draw (0) -- (1) -- (3);
    \draw (1) -- (4);
    \draw (0) -- (2);

    \node[draw=none] at(0,-2) {};
\end{tikzpicture}}}\vfill
\begin{tikzpicture}[@O]
    \onslide<8->{\node[above left,draw=none,rectangle,xshift=5mm,scale=.8,yshift=-\btdmfootheight] at(current page.south east) {\copy\pinguexplainbox};}% copy for animations
\end{tikzpicture}
\end{frame}

\begin{frame}{Der Abbau eines Miep}
\begin{enumerate}
    \setcounter{enumi}{1}
\item<2-> \task{Wandeln Sie den Heap aus Phase 1 schrittweise in ein sortiertes Array um.}
\end{enumerate}\vfill
\tikzset{O/.style={circle,draw,minimum width=2em},every node/.append style={O}}
\centering\downsize\linewidth{\onslide<3->{\begin{tikzpicture}[align-half-base,xscale=0.75]
    \node[codeouthl] (0) at (0,0) {-1};
    \node (1) at (-1,-1) {0};
    \node (2) at ( 1,-1) {9};
    \node (3) at (-2,-2) {4};
    \draw (0) -- (1) -- (3);
    \draw (0) -- (2);
    \onslide<4->{\draw[-Kite] (3) to[bend right] (0);}

    \node[draw=none] at(0,-2) {};
    \onslide<5->{\node[below,draw=none,rectangle] at (current bounding box.south) {\([-1]\)};}
\end{tikzpicture}}\onslide<6->{~~\(\longrightarrow\)~~%
\begin{tikzpicture}[align-half-base,xscale=0.75]
    \node (0) at (0,0) {4};
    \node (1) at (-1,-1) {0};
    \node (2) at ( 1,-1) {9};
    \draw (0) -- (1) coordinate[pos=.5] (@);
    \draw (0) -- (2);
    \node[draw=none] at(0,-2) {};
    \pgfinterruptboundingbox
    \onslide<7->{\draw[rotate=42,thick] (@) ellipse[x radius=1.5cm, y radius=7.5mm];
    \endpgfinterruptboundingbox

    \node[below,draw=none,rectangle] at (current bounding box.south) {\([-1]\)};}
\end{tikzpicture}}\onslide<8->{~~\(\longrightarrow\)~~%
\begin{tikzpicture}[align-half-base,xscale=0.75]
    \node[codeouthl] (0) at (0,0) {0};
    \node (1) at (-1,-1) {4};
    \node (2) at ( 1,-1) {9};
    \draw (0) -- (1) coordinate[pos=.5] (@);
    \draw (0) -- (2);
    \node[draw=none] at(0,-2) {};
    \onslide<9->{\draw[-Kite] (2) to[bend left] (0);}

    \onslide<10->{\node[below,draw=none,rectangle] at (current bounding box.south) {\([-1, 0]\)};}
\end{tikzpicture}}\onslide<11->{~~\(\longrightarrow\)~~%
\begin{tikzpicture}[align-half-base,xscale=0.75]
    \node (0) at (0,0) {9};
    \node (1) at (-1,-1) {4};
    \draw (0) -- (1) coordinate[pos=.5] (@);
    \node[draw=none] at(0,-2) {};
    % \onslide<9->{\draw[-Kite] (2) to[bend left] (0);}

    \pgfinterruptboundingbox
    \onslide<12->{\draw[rotate=42,thick] (@) ellipse[x radius=1.5cm, y radius=7.5mm];
    \endpgfinterruptboundingbox

    \node[below,draw=none,rectangle] at (current bounding box.south) {\([-1, 0]\)};}
\end{tikzpicture}}\onslide<13->{~~\(\longrightarrow\)~~%
\begin{tikzpicture}[align-half-base,xscale=0.75]
    \node[codeouthl] (0) at (0,0) {4};
    \node (1) at (-1,-1) {9};
    \draw (0) -- (1);
    \node[draw=none] at(0,-2) {};
    \onslide<14->{\draw[-Kite] (1) to[bend right] (0);}

    \onslide<15->{\node[below,draw=none,rectangle] at (current bounding box.south) {\([-1, 0, 4]\)};}
\end{tikzpicture}}}\medskip\\
\onslide<16->{~~\(\longrightarrow\)~~%
\begin{tikzpicture}[align-half-base,xscale=0.75]
    \node[codeouthl] (0) at (0,0) {9};
    \onslide<17->{\node[below=2mm,draw=none,rectangle] at (current bounding box.south) {\([-1, 0, 4, 9]\)};}
\end{tikzpicture}}
\end{frame}
}
}
% TODO. darauf hinweisen: rekursive datenstruktur

\subsection{Aufgabe 2}
\begin{frame}[c]{Aufgabe 2: Rekursive Sortieralgorithmen: Shaker Sort}
    \task<2->{In dieser Aufgabe sollen Sie einen rekursiven Sortieralgorithmus selbst implementieren. Wir betrachten eine Variante des Bubble-Sort Algorithmus, bei dem das Array in jedem Sortierschritt alternierend von vorne und hinten durchlaufen wird. Nach jedem Durchlaufen werden die bereits sortieren Elemente vorne und hinten nicht mehr betrachtet werden. Implementieren Sie Shaker Sort in Java und testen Sie Ihre Lösung an mindestens einem Beispiel.}
\end{frame}

\begin{frame}[c,fragile]{Helping Swap}
\begin{plainjava}
!*\onslide<2->*!public class ShakerSort {
!*\onslide<3->*!    private static void swap(int[] arr, int i, int j) {
!*\onslide<4->*!        int tmp = arr[i];
!*\onslide<4->*!        arr[i] = arr[j];
!*\onslide<4->*!        arr[j] = tmp;
!*\onslide<3->*!     }
!*\onslide<2->*!}!*\onslide<1->*!
\end{plainjava}
\end{frame}

{
\MakeThePinguExplainIt[text width=6.65cm,yshift=-1.15cm]{cap=!hide,hair=pingu@purple,headband,glasses=!hide,sunglasses round,eyes shiny,cup=pingu@yellow,right item angle=-50}{Was müsste man anpassen um immer mit \bjava{i + 1} zu vergleichen?}
\begin{frame}[c,fragile]{Recursive Bubble-Up}
\SetupLstHl
\begin{plainjava}
|ihl|public class ShakerSort {|ihl|
    |ihl|private static void swap(int[] arr, int i, int j) { ... }|ihl|

!*\onslide<2->*!    static boolean moveUp(int[] arr, int i, int end, boolean swapped) {
!*\onslide<3->*!        if (i >= end)
!*\onslide<3->*!            return swapped;
!*\onslide<4->*!        if (arr[i] < arr[i - 1]) {
!*\onslide<5->*!            swap(arr, i, i - 1);
!*\onslide<6->*!            swapped = true;
!*\onslide<4->*!        }
!*\onslide<7->*!        return moveUp(arr, i+1, end, swapped);
!*\onslide<2->*!    }!*\onslide<1->*!
|ihl|}|ihl|
\end{plainjava}
\begin{tikzpicture}[@O]
    \onslide<8->{\node[above left,draw=none,rectangle,xshift=5mm,scale=.8] at(current page.south east) {\copy\pinguexplainbox};}% copy for animations
\end{tikzpicture}
\end{frame}
}

\begin{frame}[c,fragile]{Recursive Bubble-Down}
\SetupLstHl
\begin{onlyenv}<-2|handout:0>
\begin{plainjava}
|ihl|public class ShakerSort {|ihl|
    |ihl|private static void swap(int[] arr, int i, int j) { ... }|ihl|
    |ihl|static boolean moveUp(int[] arr, int i, int end, boolean swapped) |ihl|
        |ihl|{ ... }|ihl|

!*\onslide<2->*!    static boolean moveDown(int[] arr, int i, int end, boolean swapped) {
!*\onslide<2->*!        if (i <= end)
!*\onslide<2->*!            return swapped;
!*\onslide<2->*!        if (arr[i] < arr[i - 1]) {
!*\onslide<2->*!            swap(arr, i, i - 1);
!*\onslide<2->*!            swapped = true;
!*\onslide<2->*!        }
!*\onslide<2->*!        return moveDown(arr, i-1, end, swapped);
!*\onslide<2->*!    }!*\onslide<1->*!
|ihl|}|ihl|
\end{plainjava}
\end{onlyenv}
\begin{onlyenv}<3->
\begin{plainjava}
|ihl|public class ShakerSort {|ihl|
    |ihl|private static void swap(int[] arr, int i, int j) { ... }|ihl|
    |ihl|static boolean moveUp(int[] arr, int i, int end, boolean swapped) |ihl|
        |ihl|{ ... }|ihl|

    |ihl|static boolean moveDown(int[] arr, int i, int end, boolean swapped) {|ihl|
        |ihl|if (i |ihl|<=|ihl| end)|ihl|
            |ihl|return swapped;|ihl|
        |ihl|if (arr[i] < arr[i - 1]) {|ihl|
        |ihl|    swap(arr, i, i - 1);|ihl|
        |ihl|    swapped = true;|ihl|
        |ihl|}|ihl|
        |ihl|return|ihl| moveDown|ihl|(arr,|ihl| i-1|ihl|, end, swapped);|ihl|
    |ihl|}|ihl|
|ihl|}|ihl|
\end{plainjava}
\end{onlyenv}
\end{frame}

% TODO: alternative mit start und end
\begin{frame}[c,fragile]{I shake it up, I shake it down!}
\SetupLstHl
\begin{plainjava}
|ihl|public class ShakerSort {|ihl|
    |ihl|private static void swap(int[] arr, int i, int j) { ... }|ihl|
    |ihl|static boolean moveUp(int[] arr, int i, int end, boolean s) { ... }|ihl|
    |ihl|static boolean moveDown(int[] arr, int i, int end, boolean s) { ... }|ihl|

    !*\onslide<2->*!public static void shakerSort(int[] array, int n) {
    !*\onslide<3->*!    int offset = array.length - n;
    !*\onslide<4->*!    if (n <= array.length / 2)
    !*\onslide<4->*!        return;
    !*\onslide<5->*!    else if (!moveUp(array, offset + 1, n, false))
    !*\onslide<6->*!        return;
    !*\onslide<5->*!    else if (!moveDown(array, n - 1, offset, false))
    !*\onslide<6->*!        return;
    !*\onslide<7->*!    shakerSort(array, n - 1);
    !*\onslide<2->*!}
|ihl|}|ihl|
\end{plainjava}
\end{frame}

\begin{frame}[c,fragile]{Some sample Code}
\SetupLstHl
\begin{plainjava}
!*\CodeFileMarkerAttach{ShakerSort.java}*!
|ihl|public class ShakerSort {|ihl|
    |ihl|public static void shakerSort(int[] array, int n) { ... }|ihl|

!*\onslide<2->*!    public static void main(String[] args) {
!*\onslide<3->*!        Random random = new Random();
!*\onslide<4->*!        int length = random.nextInt(11) + 10;
!*\onslide<5->*!        int[] array = new int[length];!*\smallskip*!
!*\onslide<6->*!        for (int i = 0; i < length; i++)
!*\onslide<6->*!           array[i] = random.nextInt(100);!*\smallskip*!
!*\onslide<7->*!        System.out.println("Unsortiert: " + Arrays.toString(array));
!*\onslide<8->*!        shakerSort(array, length);
!*\onslide<9->*!        System.out.println("Sortiert: " + Arrays.toString(array));
!*\onslide<2->*!    }!*\onslide<1->*!
|ihl|}|ihl|
\end{plainjava}
\end{frame}

\subsection{Aufgabe 3}
\begin{frame}[c]{Aufgabe 3: Laplacescher Entwicklungssatz}
\taskblock<2->
    In dieser Aufgabe sollen Sie nochmal die Implementierung rekursiver Funktionen üben. Dazu betrachten wir die
    rekursive Berechnung der Determinante einer quadratischen \(n \times n\) Matrix \(M\). Die Determinante gibt an, wie sich
    das Volumen bei der durch die Matrix beschriebenen linearen Abbildung ändert, und ist ein Hilfsmittel bei der
    Lösung linearer Gleichungssysteme. Sie kann für kleine Matrizen (\(1 \leq n \leq 10\)) rekursiv über den \link{https://de.wikipedia.org/wiki/Determinante\#Laplacescher\_Entwicklungssatz}{Laplaceschen
    Entwicklungssatz} nach der ersten Spalte berechnet werden:
    \begin{equation*}
        \det(M) = \begin{cases}
            m_{1,1} & \text{ falls } n = 1\\
            \sum_{i = 1}^n (-1)^{i + 1} m_{i, 1} \det(M_{-i 1}) & \text{ für } n \geq 2
        \end{cases}
    \end{equation*}
    wobei \(n \times n\) die Dimensionalität der Matrix \(M\), \(M_{-i, 1}\) die Matrix, die aus \(M\) entsteht, wenn die erste Spalte und die \(i\)-te Zeile entfernt werden (samt anschließendem Zusammenschieben) und \(m_{i, j}\) das element in der \(i\)-ten Zeile und \(j\)-ten Spalte seien.

    In den Materialien zu diesem Übungsblatt finden Sie die Datei \T{Determinante.java}, die bereits alle notwendigen
    Hilfsmethoden enthält. Sie sollen die vorgegebene Methode \T{int det(int[][] mat)} implementieren.
\endtaskblock
\end{frame}
% note hier nichts zur Verfügung weil rechtlich an prof gebunden?

\begin{frame}[fragile,c]{Translating a formula\ldots\space 1:1 (caring for zeros)}
\begin{plainjava}
!*\onslide<2->*!private static int det(int[][] mat) {
!*\onslide<3->*!    if(mat.length == 1) {
!*\onslide<4->*!        return mat[0][0];
!*\onslide<3->*!    } else {
!*\onslide<5->*!        int result = 0;
!*\onslide<6->*!        for(int i = 0; i < mat.length; i++) {
!*\onslide<7->*!            result += (int)Math.pow(-1, i+1) * mat[i][0]
!*\onslide<7->*!                        * det(reduceMatrix(mat, i));
!*\onslide<6->*!        }
!*\onslide<8->*!        return result;
!*\onslide<3->*!    }
!*\onslide<2->*!}!*\onslide<1->*!
\end{plainjava}
\begin{tikzpicture}[@O]
    \onslide<1->{\node[below left,text width=7cm,yshift=-1.25cm,gray] at (current page.north east) {\begin{equation*}
        \det(M) = \begin{cases}
            m_{1,1} & \text{ falls } n = 1\\
            \sum_{i = 1}^n (-1)^{i + 1} m_{i, 1} \det(M_{-i 1}) & \text{ für } n \geq 2
        \end{cases}
    \end{equation*}};}
\end{tikzpicture}
\end{frame}


\iffull
\SetNextSectionText{Dynamische Datenstrukturen\\Abgabe: \DTMDate{2022-07-11}}
\section{Aussicht: Übungsblatt 10}
\subsection{Aufgabe 1}
\begin{frame}{Aufgabe 1: Bäume}
\begin{itemize}[<+(1)->]
    \item Wir können Bäume als rekursive Datenstruktur beschreiben \begin{itemize}
        \item Für Binärbäume haben wir einen Knoten als Wurzel eines Teilbaums
        \item Einen linken Teilbaum und einen rechten Teilbaum
    \end{itemize}
\end{itemize}\bigskip
\begin{columns}[onlytextwidth,c]
    \column{.1\linewidth}
    \column{.3\linewidth}
    \begin{uncoverenv}<5->
    \begin{forest}
        for tree={circle,draw}
        [7[4[9[1][~,phantom]][0[2][5]]][3]]
    \end{forest}
    \end{uncoverenv}
    \column{.6\linewidth}
    \begin{itemize}
        \item<6-> Breitendurchlauf:\\
                  \onslide<7->{7, 4, 3, 9, 0, 1, 2, 5}
        \item<8-> Prä-Order: (Knoten, links, rechts)\\
                  \onslide<9->{7, 4, 9, 1, 0, 2, 5, 3}
        \item<10-> In-Order: (links, Knoten, rechts)\\
                  \onslide<11->{1, 9, 4, 2, 0, 5, 7, 3}
        \item<12-> Post-Order: (links, rechts, Knoten)\\
                  \onslide<13->{1, 9, 2, 5, 0, 4, 3, 7}
    \end{itemize}
\end{columns}
\end{frame}

{\AddonFrame
\begin{frame}[c]{Exkurs: Ein wenig fantastische Heaps gefällig?}
\onslide<2->{\centering\heap{%
    [42,name=root[19,name=inner1[7,name=inner2[3,name=leaf1][-2,name=leaf2]][8,name=leaf3]][-5,name=inner3[-9,name=leaf4]]]%
    \onslide<3->{\fill[paletteA,rounded corners=.4pt] (root.north)++(-.75mm,2mm-.75mm) rectangle ++(1.5mm,1.5mm) [radius=2.15pt] node[left=1.65mm+.75mm,paletteA,font=\bfseries] (dr) {Wurzel};
    \node[below=-4pt,gray,font=\scriptsize\itshape] at (dr.south) {root};}
    \pgfinterruptboundingbox
    \onslide<4->{\fill[paletteB] (inner1.north)++(0,2mm) circle [radius=2.15pt] node[left=1.65mm,paletteB,font=\bfseries] (di) {Innerer Knoten};
    \node[below=-4pt,gray,font=\scriptsize\itshape] at (di.south) {inner node};
    \foreach\i in {2,3} {\fill[paletteB] (inner\i.north)++(0,2mm) circle [radius=2.15pt];}}
    \onslide<5->{\fill[paletteD] (leaf1.west)++(-2mm,0) circle [radius=2.15pt] node[left=1.65mm,paletteD,font=\bfseries] (dl) {Blatt};
    \node[below=-4pt,gray,font=\scriptsize\itshape] at (dl.south) {leaf};
    \foreach\i in {2,3,4} {\fill[paletteD] (leaf\i.west)++(-2mm,0) circle [radius=2.15pt];}}
    \endpgfinterruptboundingbox
    \onslide<6->{
        \node[right=1.25cm,gray] (niveau) at (dr-|inner3) {\textbf{Ebene}};
    }
    \pgfonlayer{background}
        \foreach[count=\i] \x in {root,inner1,leaf3,leaf1} {
            \onslide<\the\numexpr\i+6\relax->{
                \node[gray,font=\bfseries] (dn\i) at(\x-|niveau) {\i};
                \draw[lgray!90!white,densely dashed, very thick] (dn\i.west) -- (dl.west|-dn\i.west);
            }
        }
    \endpgfonlayer
}}
\end{frame}
}
\subsection{Aufgabe 2}
\begin{frame}{Aufgabe 2: Self-Overriding Ring Buffer}
    \begin{itemize}[<+(1)->]
        \itemsep10pt
        \item Bei einem Ring Buffer, zeigt ein Element nie auf \bjava{null}
        \item Komplizierter wird es, wenn nicht werte in Elementen überschrieben sondern neue Elemente eingesetzt werden müssen
        \item Versucht kategorisch, Skizzen zur Hilfestellung zu nehmen \begin{itemize}
            \item Insbesondere wenn mehrere Zeiger im Spiel sind, ist das hilfreich
            \item So können auch Randfälle identifiziert werden
        \end{itemize}
        \item Gebt euch Mühe, sauberen Code abzugeben
        \item Kommentare werden gern gesehen
    \end{itemize}
\end{frame}
\fi


% \SetNextSectionText[.55\linewidth]{TODO}
\section{Abschließendes}
{\SummaryFrame
\begin{frame}[fragile,t]{Zusammenfassend}
% \printBibCommand
\lstfs{9}
\vfill\vfill % double fill for more fraction
\begin{itemize}[<+(1)->]
    \itemsep8pt
    \item Rekursion wird uns so schnell nicht verlassen \begin{itemize}
        \item Viele Datenstrukturen sind rekursiv, was wir ausnutzen können
        \item Für die verschiedenen Tiefen-Traversierungen beispielsweise:
\begin{columns}[onlytextwidth,c]
\column{.33\linewidth}
\begin{plainjava}
!*\onslide<5->*!void preorder(Node n) {
!*\onslide<5->*!    if(n == null) return;
!*\onslide<5->*!    // process n
!*\onslide<5->*!    preorder(n.left);
!*\onslide<5->*!    preorder(n.right);
!*\onslide<5->*!}!*\onslide<1->*!
\end{plainjava}
\column{.33\linewidth}
\begin{plainjava}
!*\onslide<6->*!void inorder(Node n) {
!*\onslide<6->*!    if(n == null) return;
!*\onslide<6->*!    inorder(n.left);
!*\onslide<6->*!    // process n
!*\onslide<6->*!    inorder(n.right);
!*\onslide<6->*!}!*\onslide<1->*!
\end{plainjava}
\column{.33\linewidth}
\begin{plainjava}
!*\onslide<7->*!void postorder(Node n) {
!*\onslide<7->*!    if(n == null) return;
!*\onslide<7->*!    postorder(n.left);
!*\onslide<7->*!    postorder(n.right);
!*\onslide<7->*!    // process n
!*\onslide<7->*!}!*\onslide<1->*!
\end{plainjava}
\end{columns}
    \item Auch lassen sich Regeln für Heaps so beschreiben
    \end{itemize}
    \item Listen (als spezielle Bäume) und Bäume sind vielseitig und elementar \begin{itemize}
        \item Auch bei ihrer Handhabung hilft Übung!
    \end{itemize}
\end{itemize}
\end{frame}
}

\outro{\vskip9mm\centering \onslide<2->{\scalebox{1.35}{\begin{tikzpicture}
    \node[align=center,scale=0.5] (peter) at(-3,0) {\tikz{\duck[body=yellow!50!red!20!white,recedinghair=gray!50!white,eyebrow,tshirt=white!93!black,jacket=red!50!black,glasses=brown!70!lightgray,scale=0.75];}\\Peter};
    \node[align=center,scale=0.5] (erika) at(-1,0) {\tikz{\duck[body=yellow!50!red!20!white,longhair=gray!50!white,eyebrow,tshirt=white!93!black,jacket=blue!50!black,squareglasses=brown!70!lightgray,scale=0.75];}\\Erika};
    \node[align=center,scale=0.5] (wolfgang) at(1,0) {\tikz{\duck[body=yellow!50!red!20!white,recedinghair=gray!50!white,eyebrow,tshirt=white!93!black,jacket=yellow!70!black,glasses=brown!70!lightgray,scale=0.75];}\\Wolfgang};
    \node[align=center,scale=0.5] (rosalinde) at(3,0) {\tikz{\duck[body=yellow!50!red!20!white,longhair=gray!50!white,eyebrow,tshirt=white!93!black,jacket=lime!50!black,squareglasses=brown!70!lightgray,scale=0.75];}\\Rosalinde};

    \node[align=center,scale=0.5] (dieter) at(-2,-1.75) {\tikz{\duck[body=yellow!50!red!20!white,parting=gray!80!brown,eyebrow,jacket=gray,lapel=black,buttons,scale=0.75];}\\Dieter};
    \node[align=center,scale=0.5] (jasmin) at(2,-1.75) {\tikz{\duck[body=yellow!50!red!20!white,shorthair=brown!50!black,eyebrow,tshirt=white!93!black,jacket=purple!90!black,squareglasses=brown!70!lightgray,scale=0.75];}\\Jasmin};

    \node[align=center,scale=0.5] (flo) at(0,-3.5) {\tikz{\duck[body=yellow!50!red!20!white,crazyhair=brown!90!black,eyebrow,jacket=pingu@purple,lapel=pingu@purple!50!black,buttons,scale=0.75,tie=pingu@purple!50!black,tshirt=white,glasses,laughing];}\\Florian};


    \begin{scope}[every path/.style=rounded corners]
        \draw(peter.south) -- ++(0,-0.25) -| (dieter);
        \draw(erika.south) -- ++(0,-0.25) -| (dieter);

        \draw(wolfgang.south) -- ++(0,-0.25) -| (jasmin);
        \draw(rosalinde.south) -- ++(0,-0.25) -| (jasmin);

        \draw(dieter.south) -- ++(0,-0.25) -| (flo);
        \draw(jasmin.south) -- ++(0,-0.25) -| (flo);
    \end{scope}
\end{tikzpicture}}}}

\iffull\end{document}\fi
