\InputIfFileExists{../data/global.tex}\relax\relax

\iffull
\title{Fifo Heaps} % \ldots\ sich selbst | suchen und ordnen
\subtitle{Tutorium Neun}
\date{KW 27}
\addbibresource{references.bib}
\fi
\SetTutoriumNumber{9}

\iffull\begin{document}
\titleframe

\TopicOverview{10}
\fi

\iffull{\SummaryFrame
\begin{frame}[c]{Kurzwiederholung}
\begin{itemize}[<+(1)->]
   \itemsep8pt
    \item
\end{itemize}
\end{frame}
}\fi

\SetNextSectionText[.6\linewidth]{TODO}
\section{Präsenzaufgabe}
\begin{frame}[fragile,c]{Präsenzaufgabe}
\begin{aufgabe}{FIFO-Freuden}
XX
\end{aufgabe}
\end{frame}


\SetNextSectionText{Suchen und Sortieren~II\\Abgabe: \DTMDate{2022-07-04}}
\section{Übungsblatt 9}
\subsection{Aufgabe 1}
{\taskenum
\begin{frame}{Aufgabe 1: }
\task<2->{}
\end{frame}

\iffull
\SetNextSectionText{Dynamische Datenstrukturen\\Abgabe: \DTMDate{2022-07-11}}
\section{Aussicht: Übungsblatt 10}

\begin{frame}{Aufgabe 1: TODO}
\begin{itemize}[<+(1)->]
    \item
\end{itemize}
\end{frame}

\outro{\vskip9mm\centering \onslide<2->{\begin{tikzpicture}
    % \pingu[body=pingu@black,body front=pingu@black,eyes wink,bill color=pingu@black,eyes color=pingu@black,feet color=pingu@black,tie=pingu@white]
\end{tikzpicture}}}

\iffull\end{document}\fi
