\makeatletter
\def\DoCustomTheme{\def\btdmtheme{%
\usetheme[libs,nofootfade,centerfoot]{dividing-lines}%
\RequirePackage{xcolor}\colorlet{btdm@primary}{green}\colorlet{btdm@background}{white}\colorlet{btdm@border@down}{green}\colorlet{btdm@text}{btdl@color@text}%
\let\highlight\hl
\let\hl\relax% work with soul
\def\IfBtdmDarkmode####1####2{####2}\relax
\def\setfootmarker####1{\relax}%
}}
\@ifundefined{iffull}{\DoCustomTheme}{\iffull\DoCustomTheme\fi}
\InputIfFileExists{../data/global.src}\relax\relax
\iffull
\DefinePalette{Recap}{Hellblau,hellbläulich: RGB(21, 110, 130)}
{Rot,rötlich: RGB(137,64,75)}{Blau,bläulich: RGB(51, 93, 126)}{Grün,grünlich: RGB(21, 150, 90)}
\UsePalette{Recap}
\SetColorProfile*{paletteA}{paletteC}{paletteD}
\def\PostTitlepage{\begin{tikzpicture}[@O]
   \node[above right,xshift=7.5mm,yshift=1.25cm,scale=.8] at(current page.south west) {\LoadOverview{1}{}{2}~~\LoadOverview{2}{}{3\,\&\,4}~~\LoadOverview{3}{}{5}~~\LoadOverview{4}{}{6}~~\LoadOverview{5}{}{7}};
\end{tikzpicture}}

\def\defaultitemizeblock{\faAngleRight}
\updateitemize{\paletteA{\defaultitemizeblock}}
\updateitemizesub{\textcolor{gray}{\faAngleRight}}

\newcounter{slidetasks}
\setcounter{slidetasks}{0}
\def\SidebarTask#1{\stepcounter{slidetasks}\SidebarRaw{\Large\rotatebox{90}{\def\rmdefault{AlphaSlabOne-TLF}\color{paletteA!62!pingu@black!36!pingu@white}\rmfamily\selectfont Aufgabe \theslidetasks}\hfill\null}\pdfbookmark[4]{A\theslidetasks)~#1}{lecture@task@@id@\thepage}}
\def\SidebarSolution#1{\SidebarRaw{\hfill\Large\rotatebox{90}{\def\rmdefault{AlphaSlabOne-TLF}\color{paletteA!62!pingu@black!36!pingu@white}\rmfamily\selectfont #1}}}
\def\textsb#1{{\sbfamily#1}}
\def\LargeSide{\begin{tikzpicture}[overlay, remember picture]
  \fill[paletteA] (current page.north west) rectangle ([xshift=4.2mm]current page.south west);
\end{tikzpicture}}
\def\ShortSide{\begin{tikzpicture}[overlay, remember picture]
  \fill[paletteA!62!pingu@black!36!pingu@white] (current page.north west) rectangle ([xshift=2.75mm]current page.south west);
\end{tikzpicture}}
\def\rhead#1{\hfill\textcolor{shadeB}{\sbfamily#1}}
\fi

\iffull
% guard against makeuppercase
\def\titlecolorset{\color{lightgray!15!white}}
\title{\texorpdfstring{\setbox0=\hbox{Zwischen}\hskip\wd0\clap{\smash{\protect\titlecolorset\raisebox{-3.5pt}{\scalebox{1.33}{\faRepeat}}}}\llap{\box0}}{Zwischen}wiederholung}
\subtitle{Ein kleiner Blick zurück}
\date{KW 23}
\addbibresource{references.bib}
\else
\SetTutoriumNumber{Wdh}
\fi

\iffull\begin{document}
\fi

\iffull
{\btfootfalse
{\setbeamercolor{background canvas}{bg=paletteA}
\def\ImpT#1{\textit{\color{pingu@white}#1}}
\begin{frame}[c,plain]{~}
\begin{layout-full}
\begin{center}
   \bfseries\color{paletteA!68!white}
      Dieses \ImpT{Recap} liefert \ImpT{keine Garantie auf Vollständigkeit}. Ich konzentriere mich bewusst auf einige wenige aber wichtige \ImpT{Kernthemen}, die in der Hinsicht~--- zumindest auf den Folien~--- auch leicht simplifiziert dargestellt sind.\bigskip\par
      \ImpT{Penguins will be happy!}\\Flo
\end{center}
\end{layout-full}
\end{frame}
}

\begin{frame}[c]{}
\def\g{\only<4->{\color{gray}}}%
\only<5-|handout:0>{\global\btfoottrue}%
\only<2|handout:0>{\vspace*{2.25em}\footnotesize%
I. Einführung\\
II. Aspekte der Algorithmenkonstruktion\\
III. Programmierung im Kleinen --- Namen und Dinge\\
IV. Programmierung im Kleinen --- Steuerung des Programmablaufs\\
V. Zeigervariable, Arrays und Iterationen\\
VI. Programmieren im Großen --- Strukturierter Entwurf und Unterprogramme\\
VII. Einführung in die objektorientierte Programmierung (OOP)\\
VIII. Rekursive Algorithmen\\
IX. Weiterführende Konzepte der objektorientierten Programmierung\\
X. Dynamische Datenstrukturen\\
XI. Algorithmen und Zeitkomplexität\\
XII. Suchen und Sortieren}%
\updateitemize{10.}%
\only<3->{\vspace*{2.25em}\begin{enumerate}

   \item<3-> \hyperlink{btdl@section.1}{Algorithmenkonstruktion}
   \item<3-> \hyperlink{btdl@section.2}{Programmkonstrukte (Namen \& Programmfluss)}
   \item<3-> \hyperlink{btdl@section.3}{Arrays und Iterationen}
   \item<3-> \hyperlink{btdl@section.4}{Unterprogramme}
   \item<3-> \hyperlink{btdl@section.5}{Objektorientierte Programmierung}\bigskip
   \item<3->[\g 6.] \g Rekursion
   \item<3->[\g 7.] \g Weiterführende Konzepte der OOP
   \item<3->[\g 8.] \g Dynamische Datenstrukturen
   \item<3->[\g 9.] \g Laufzeitkomplexität
   \item<3->[\g10.] \g Suchen und Sortieren
\end{enumerate}}%
\end{frame}
}
\fi

\section{Algorithmen}

\subsection{Definition}
\begin{frame}[c]{Was \textit{ist} ein Algorithmus?}
   \begin{center}
      \onslide<2->{Eine eindeutige Handlungsvorschrift zur Lösung eines Problems}
   \end{center}\vfill
\begin{itemize}
      \itemsep8pt
      \item<3-> Schrittweise ausführ- und reproduzierbar.
      \item<4-> Endlich viele, wohldefinierte Elementaroperationen.
      \item<5-> Stoppt für jede Eingabe in endlich vielen Schritten.\bigskip
      \item<6->[\color{gray}\defaultitemizeblock] {\color{gray}\itshape Fordert gemeinsames Sprachverständnis.}
\end{itemize}
\end{frame}

{\def\longa#1{\multicolumn{2}{l}{#1}}
\SidebarTask{Terminate the Algorithm}
\begin{frame}[c]{Terminate the Algorithm}
\onslide<2->{Antworten Sie jeweils für A, B und C mit \say{Ja}, wenn diese Terminieren oder Begründen Sie kurz.}\vfill
\columns[c,onlytextwidth]
\column{.25\linewidth}
\centering\onslide<3->{\begin{tabular}{ll}
   Given & \(i \in \Z\) \\
   Set & \(j \gets 0\) \\
   \longa{\textbf{while} \(i \geq 0\) \textbf{do}}\\
   \longa{\kern1em\(i \gets i - 1\)} \\
   \longa{\kern1em\(j \gets j + i\)} \\
   \textbf{stop} & \\
\end{tabular}\medskip\par
A}
\column{.29\linewidth}
\centering\onslide<3->{\begin{tabular}{ll}
   \strut & \\
   Given & \(p = (x, y) \in \R^2\) \\
   \longa{\textbf{if} \(x > y\) \textbf{then}}\\
   \longa{\kern1em\(p = (y,x)\)} \\
   \textbf{stop} & \\
\end{tabular}\medskip\par
B}
\column{.46\linewidth}
\centering\onslide<3->{\begin{tabular}{ll}
   \strut & \\
   Given & \(z = (z_1,\ldots, z_n)\), \(z_i \in \N\) \\
   Set & \(j \gets 0\) \\
   \longa{\textbf{while} not \(z_1 \leq \ldots \leq z_n\) \textbf{do}}\\
   \longa{\kern1em shuffle \(z\) randomly} \\
   \textbf{stop} & \\
\end{tabular}\medskip\par
C}
\endcolumns
\onslide<1->{\ShortSide}%
\end{frame}
\SidebarReset
}

\subsection{Eigenschaften}
\begin{frame}{What can we say about an Algorithm?}
   \hfill{\onslide<2->{\Large\bfseries\only<3->{\color{gray}\sbfamily} Totale Korrektheit}}\hfill\null\par
   \vfill\vfill
   \begin{enumerate}
      \itemsep10pt
      \item<3-> {\sbfamily Termination}\\
         Der Algorithmus endet nach endlich vielen Schritten für jede Eingabe.
      \item<4-> {\sbfamily Partielle Korrektheit}\\
         Wenn der Algorithmus terminiert, ist er korrekt.
   \end{enumerate}
   Determinismus und Determiniertheit
   \vfill\par
\end{frame}

\subsection{Algorithmus-Diskussion}
{\def\comm#1{\hfill\textcolor{gray}{\footnotesize#1}}
\begin{frame}[c]{Discussing an Algorithm}
\centering\begin{tikzpicture}
   \onslide<2->{\node (p) at (0,0) {\strut Problem};}
   \onslide<3->{\node (l) at (7,0) {\strut Lösung};}
   \onslide<4->{\draw[-Kite] (p) -- (l) node[pos=.5,fill=white] {\null~?~\null};}
\end{tikzpicture}
\vfill
\begin{itemize}
   \itemsep7.5pt
   \item<5-> Problemspezifikation\comm{Was meinen Sie mit \say{schnell}?}
   \item<6-> Problemabstraktion\comm{Was ist gegeben, was ist gesucht?}
   \item<7-> Algorithmenentwurf\comm{Wie kommen wir von gegeben zu gesucht?}
   \item<8-> Korrektheitsnachweis\comm{Löst unser Ansatz das Problem?}
   \item<9-> Aufwandsanalyse\comm{Wie verhält er sich?}
\end{itemize}
\end{frame}}


\section{Konstrukte}
\begin{frame}

\end{frame}
\section{Arrays \& Iteration}
\begin{frame}

\end{frame}

\section{Unterprogramme}
\begin{frame}

\end{frame}

\section{OOP}
\begin{frame}

\end{frame}

\iffull
\else
\outro{\vskip9mm\centering \onslide<2->{x}}
\fi
\iffull\end{document}\fi
