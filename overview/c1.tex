\input overview.tex
% \let\mrgray\mrssteele

\begin{document}
\dotopic\topicA
% \def\xs{2cm}\relax
% \begin{tikzpicture}
%     % 2. Algorithmenkonstruktion
%     \node (a) at (0,0) {};
%     % 3.-4. Programmieren im Kleinen
%     \node[right=\xs] (b) at(a.east) {\copy\topicB};
%     % 5. Zeiger, Iterationen und Arrays
%     \node[right=\xs] (c) at(b.east) {\copy\topicC};
%     % 6. Strukturierter Entwurf und Unterprogramme
%     \node[right=\xs] (d) at(c.east) {\copy\topicD};
%     % TODO: Kreise drum rum und etwas hübscher machen, dann thmen text daneben und einen sat kurzzusammenfassung als themnnetz vorne einbauen, gesamtüebrscht über die Wochen,Update der Folien präfereiren
%     % 7. & 9. Objektorientierung
%     \node[below=\xs] (e) at(a.south) {\copy\topicE};
%     % 8. Dynamische Datenstrukturen
%     \node[right=\xs] (f) at(e.east) {\copy\topicF};
%     % 10. Rekursive Algorithmen
%     \node[right=\xs] (g) at(f.east) {\copy\topicG};
%     % 11. Laufzeitkomplexität
%     \node[right=\xs] (h) at(g.east) {\copy\topicH};
%     % 12. Suchen und Sortieren
%     \node[below=\xs] (i) at(e.south) {\copy\topicI};
% \end{tikzpicture}
\end{document}
