\InputIfFileExists{../data/global.src}\relax\relax

\iffull
\title{Klasse globale, potente Entwürfe}
\subtitle{Tutorium fünf}
\date{KW 22}
\addbibresource{references.bib}
\fi
\SetTutoriumNumber{5}

\iffull\begin{document}
\titleframe

\TopicOverview{5}
\fi

\iffull{\SummaryFrame
\begin{frame}[fragile,c]{Kurzwiederholung}
    \begin{itemize}[<+(1)->]
        \itemsep9.25pt
        \item Bei Arrays \textit{muss} Java \info{\say{die erste Ebene}} initialisieren! \begin{itemize}
            \itemsep3pt
            \item Alles wird \say{Null} (\bjava{0}, \bjava{0.0}, \bjava{'\\0'}, \bjava{false}, \bjava{null})
            \item \bjava{new float[2][]}~\textcolor{gray}{\(\to\)} \bjava{\{null, null\}},~~\bjava{new int[2][1]}~\textcolor{gray}{\(\to\)} \bjava{\{\{0\}, \{0\}\}},~\ldots
        \end{itemize}
        \item Methoden werden in Java durch ihre Signatur unterschieden \begin{itemize}
            \itemsep3pt
            \item Dies ist der Name und die Parametertypliste
            \item Im Beispiel \onslide<9->{ist dies \bjava{mega(int, char[])}:}
\AnimateCode{onslide={*\Line{1}\Others{3}\NoLocation\Forever},first slide=8}%
\begin{plainjava}[aboveskip=0pt,belowskip=0pt]
!*\onslide<7->*!public static double mega(int i, char[] r) {
!*\onslide<7->*!    return (double) i + r.length;
!*\onslide<7->*!}!*\onslide<1->*!
\end{plainjava}
\endAnimateCode
        \end{itemize}
    \item<10-> Unterprogramme sind ein wichtiger Abstraktions-Mechanismus
    \item<11-> Seiteneffekte sind ein Problem und sollten \info{wo möglich} vermieden werden!
    \end{itemize}
\end{frame}
}\fi

\SetNextSectionText[.55\linewidth]{TODO.\\---}
\section{Präsenzaufgabe}
\begin{frame}[fragile,c]{Präsenzaufgabe}
\begin{aufgabe}{Show me ya' potency}
\vspace*{-\baselineskip}\relax
\begin{enumerate}
    \item<2-> Implementieren Sie eine Klasse für Potenzen. Die Klasse soll zwei private Attribute \T{basis} und \T{potenz} besitzen,
    sowie einen Konstruktor mit zwei Argumenten definieren, der den Attributen Anfangswerte zuweist. \onslide<3->{Zusätzlich
    soll es getter und setter für die Attribute geben. Die Klasse soll eine öffentliche Methode besitzen, die die
    Attribute der Instanz auf die Konsole ausgibt.}
    \item<4-> Legen Sie eine zweite Java Datei namens \T{PotenzenMain.java} an, die als Programmeinstiegspunkt dienen soll, d.h. hier ist die main Methode implementiert. \onslide<5->{Instanziieren Sie innerhalb der \T{main} Methode ein Objekt der
    Klasse \T{Potenz} und lassen Sie sich die Attribute des Objekts anzeigen.}
\end{enumerate}
\end{aufgabe}
\end{frame}

{\newcommand<>\Mark[1]{{\only#2{\color{black}}#1}}
\begin{frame}[fragile,c]{Ich bin Herbert Klassenzüchter!}
\columns[onlytextwidth,c]
\column{.29\linewidth}
\task<2->{Implementieren Sie eine \Mark<3->{Klasse für Potenzen}. Die Klasse soll \Mark<5->{zwei private Attribute \T{basis} und \T{potenz}} besitzen,
sowie einen \Mark<8->{Konstruktor mit zwei Argumenten} definieren, der den \Mark<10->{Attributen Anfangswerte zuweist}.
Zusätzlich soll es \Mark<12->{getter} und \Mark<14->{setter} für die Attribute geben. Die Klasse soll eine \Mark<16->{öffentliche Methode} besitzen, die die
Attribute der Instanz auf die Konsole ausgibt.}
\column{.64\linewidth}
\lstfs{9}
\begin{plainjava}[lineskip=.5pt]
!*\onslide<4->*!public class Potenz {
!*\onslide<6->*!   private !*\onslide<7->*!double !*\onslide<6->*!basis;
!*\onslide<6->*!   private !*\onslide<7->*!int !*\onslide<6->*!potenz;
!*\onslide<4->*!
!*\onslide<9->*!   public Potenz(double b, int e) {
!*\onslide<11->*!      this.basis = b;
!*\onslide<11->*!      this.potenz = e;
!*\onslide<9->*!   }
!*\onslide<13->*!   public double getBasis() { return this.basis; }
!*\onslide<13->*!   public int getPotenz() { return this.potenz; }
!*\onslide<15->*!   public void setBasis(double b) { this.basis = b; }
!*\onslide<15->*!   public void setPotenz(int e) { this.potenz = e; }

!*\onslide<17->*!   public void print() {
!*\onslide<17->*!      System.out.println(basis + "^(" + potenz + ")");
!*\onslide<17->*!   }
!*\onslide<4->*!}
\end{plainjava}
\endcolumns
\end{frame}
}

\begin{frame}{TODO:}
    TODO: toString machen

    Bauplananalogie etc.
    TODO: davor auf Dateien und selber Ordner eingehen
    TODO: auf this eingehen TODO: start überlagern  und soo
    TOPDO: heap-darstellung mit mehr chaos für Chlassen

    TODO: private hilfsmethoden ansprechen
    TODO: Ausblick mit Gültigkeits- und Sichtbarkeitsbereichen
    TODO: Signatur nochmal wiederholen usw
    TODO: überschatten überlagern

    TODO: Klassenentwurf vom letztenmal :D
    TODO: Begriffe fundieren

    TODO: godbolt als Java-Beispiele?
\end{frame}
\SetNextSectionText[.55\linewidth]{TODO}
\section{Abschließendes}
{\SummaryFrame
\begin{frame}[t]{Zusammenfassend}
\pause \printBibCommand
\vfill\vfill % double fill for more fraction
\begin{itemize}[<+(1)->]
    \itemsep6.5pt
    \item TODO
\end{itemize}
\end{frame}
}


\outro{\vskip9mm\centering \onslide<2->{\begin{tikzpicture}

\end{tikzpicture}}}

\iffull\end{document}\fi
