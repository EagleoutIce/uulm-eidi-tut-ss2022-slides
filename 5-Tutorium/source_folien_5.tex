\InputIfFileExists{../data/global.src}\relax\relax

\iffull
\title{Klasse globale, potente Entwürfe}
\subtitle{Tutorium fünf}
\date{KW 22}
\addbibresource{references.bib}
\fi
\SetTutoriumNumber{5}

\iffull\begin{document}
\titleframe

\TopicOverview{5}
\fi

\iffull{\SummaryFrame
\begin{frame}[fragile,c]{Kurzwiederholung}
    \begin{itemize}[<+(1)->]
        \itemsep9.25pt
        \item Bei Arrays \textit{muss} Java \info{\say{die erste Ebene}} initialisieren! \begin{itemize}
            \itemsep3pt
            \item Alles wird \say{Null} (\bjava{0}, \bjava{0.0}, \bjava{'\\0'}, \bjava{false}, \bjava{null})
            \item \bjava{new float[2][]}~\textcolor{gray}{\(\to\)} \bjava{\{null, null\}},~~\bjava{new int[2][1]}~\textcolor{gray}{\(\to\)} \bjava{\{\{0\}, \{0\}\}},~\ldots
        \end{itemize}
        \item Methoden werden in Java durch ihre Signatur unterschieden \begin{itemize}
            \itemsep3pt
            \item Dies ist der Name und die Parametertypliste
            \item Im Beispiel \onslide<9->{ist dies \bjava{mega(int, char[])}:}
\AnimateCode{onslide={*\Line{1}\Others{3}\NoLocation\Forever},first slide=8}%
\begin{plainjava}[aboveskip=0pt,belowskip=0pt]
!*\onslide<7->*!public static double mega(int i, char[] r) {
!*\onslide<7->*!    return (double) i + r.length;
!*\onslide<7->*!}!*\onslide<1->*!
\end{plainjava}
\endAnimateCode
        \end{itemize}
    \item<10-> Unterprogramme sind ein wichtiger Abstraktions-Mechanismus
    \item<11-> Seiteneffekte sind ein Problem und sollten \info{wo möglich} vermieden werden!
    \end{itemize}
\end{frame}
}\fi

\SetNextSectionText[.55\linewidth]{TODO.\\---}
\section{Präsenzaufgabe}
\begin{frame}[fragile,c]{Präsenzaufgabe}
\begin{aufgabe}{Show me ya' potency}
\vspace*{-\baselineskip}\relax
\begin{enumerate}
    \item<2-> Implementieren Sie eine Klasse \T{Potenz} für Potenzen. Die Klasse soll zwei private Attribute \T{basis} und \T{potenz} besitzen,
    sowie einen Konstruktor mit zwei Argumenten definieren, der den Attributen Anfangswerte zuweist. \onslide<3->{Zusätzlich
    soll es getter und setter für die Attribute geben. Die Klasse soll eine öffentliche Methode besitzen, die die
    Attribute der Instanz auf die Konsole ausgibt.}
    \item<4-> Legen Sie \info{im selben Ordner} eine zweite Java Datei namens \T{PotenzMain.java} an, die als Programmeinstiegspunkt dienen soll, d.h. hier ist die \T{main} Methode implementiert. \onslide<5->{Instanziieren Sie innerhalb der \T{main} Methode ein Objekt der
    Klasse \T{Potenz} und lassen Sie sich die Attribute des Objekts anzeigen.}
\end{enumerate}
\end{aufgabe}
\begin{tikzpicture}[@O]
\node[above left=.35mm,thick,draw,gray,fill=white,rounded corners=2pt,yshift=\btdmfootheight,scale=.65,align=left] at(current page.south east) {\bbash{javac Potenz.java PotenzMain.java}\\\bbash{java PotenzMain}};
\end{tikzpicture}
\end{frame}

{\newcommand<>\Mark[1]{{\only#2{\color{black}}#1}}
\begin{frame}[fragile,c]{Ich bin Herbert Klassenzüchter!}
\columns[onlytextwidth,c]
\column{.3\linewidth}
\task<2->{Implementieren Sie eine \Mark<3->{Klasse \T{Potenz} für Potenzen}. Die Klasse soll \Mark<5->{zwei private Attribute \T{basis} und \T{potenz}} besitzen,
sowie einen \Mark<8->{Konstruktor mit zwei Argumenten} definieren, der den \Mark<10->{Attributen Anfangswerte zuweist}.\medskip\par
Zusätzlich soll es \Mark<12->{getter} und \Mark<14->{setter} für die Attribute geben. Die Klasse soll eine \Mark<16->{öffentliche Methode} besitzen, die die
Attribute der Instanz auf die Konsole ausgibt.}
\column{.64\linewidth}
\lstfs{9}
\begin{plainjava}[lineskip=.5pt,morekeywords={[3]{Potenz,PotenzMain}}]
!*\onslide<4->*!public class Potenz {
!*\onslide<6->*!   private !*\onslide<7->*!double !*\onslide<6->*!basis;
!*\onslide<6->*!   private !*\onslide<7->*!int !*\onslide<6->*!potenz;
!*\onslide<4->*!
!*\onslide<9->*!   public Potenz(double b, int e) {
!*\onslide<11->*!      this.basis = b;
!*\onslide<11->*!      this.potenz = e;
!*\onslide<9->*!   }
!*\onslide<13->*!   public double getBasis() { return this.basis; }
!*\onslide<13->*!   public int getPotenz() { return this.potenz; }
!*\onslide<15->*!   public void setBasis(double b) { this.basis = b; }
!*\onslide<15->*!   public void setPotenz(int e) { this.potenz = e; }

!*\onslide<17->*!   public void print() {
!*\onslide<17->*!      System.out.println(basis + "^(" + potenz + ")");
!*\onslide<17->*!   }
!*\onslide<4->*!}
\end{plainjava}
\endcolumns
\begin{tikzpicture}[@O]
    \only<18->{\node[above left,yshift=\btdmfootheight] at (current page.south east) {\textattachfile{\curpath Potenz.java}{Potenz.java}\;};}
\end{tikzpicture}
\end{frame}

\begin{frame}[fragile,c]{Major Main}
\columns[onlytextwidth,c]
\column{.3\linewidth}
\task<3->{Legen Sie eine \Mark<4->{zweite Java Datei namens \T{PotenzMain.java}} an, die als Programmeinstiegspunkt dienen soll, d.h. hier ist die \Mark<6->{\T{main} Methode} implementiert.\medskip\par Instanziieren Sie innerhalb der \T{main} Methode \Mark<8->{ein Objekt der
Klasse \T{Potenz}} und lassen Sie sich die \Mark<10->{Attribute des Objekts anzeigen}.}
\column{.64\linewidth}
\lstfs{9}\SetupLstHl
\begin{plainjava}[lineskip=1.5pt,morekeywords={[3]{Potenz,PotenzMain}}]
!*\onslide<2->*!!*\CodeFileMarker{Potenz.java}*!
!*\onslide<2->*!|ihl|public class Potenz {|ihl|
!*\onslide<2->*!|ihl|    public Potenz(double b, int e) { ... }|ihl|
!*\onslide<2->*!|ihl|    ...|ihl|
!*\onslide<2->*!|ihl|    public void print() { ... }|ihl|
!*\onslide<2->*!|ihl|}|ihl|

!*\onslide<5->*!!*\CodeFileMarker{PotenzMain.java}*!
!*\onslide<5->*!public class PotenzMain {
!*\onslide<12->*!   !*\ShowInTheWeb{https://www.online-java.com/oBHh8ndD2C}*!
!*\onslide<7->*!   public static void main(String[] args) {
!*\onslide<9->*!      Potenz p = new Potenz(2.0, 3);
!*\onslide<11->*!      p.print(); !*\onslide<12->*!// :yields: 2.0^(3)
!*\onslide<7->*!   }
!*\onslide<5->*!}
\end{plainjava}
\endcolumns
\begin{tikzpicture}[@O]
    \only<12->{\node[above left,yshift=\btdmfootheight] at (current page.south east) {\textattachfile{\curpath PotenzMain.java}{PotenzMain.java}\;};}
\end{tikzpicture}
\end{frame}
}

\SetNextSectionText{Methoden und OOP\\Abgabe: \DTMDate{2022-05-30}}
\section{Übungsblatt 5}
\subsection{Aufgabe 1}
\begin{lrbox}{\codebox}
\SetupLstHl\lstfs{7}\begin{minipage}{7cm}
\begin{plainjava}
public class Quersumme {
    public static void main(String[] args) {
        int zahl = Integer.parseInt(args[0]);

        if (zahl < 0) {
            System.err.println("Eingabe ungültig!");
            return;
        }

        int quersumme = 0;
        int divRest;
        while (zahl > 0) {
            divRest = zahl % 10;
            quersumme = quersumme + divRest;
            zahl = zahl / 10;
        }
        System.out.println(quersumme);
    }
}
\end{plainjava}
\end{minipage}
\end{lrbox}
\begin{frame}[c]{Aufgabe 1: Methoden mit einer variablen Parameterzahl }
    \task<2->{Legen Sie eine Java Datei namens Quersummen.java und implementieren Sie die folgende Aufgabe innerhalb dieser
    Datei als eine öffentliche statische Methode namens \T{quersummeVonQuersummen}.
    \onslide<3->{Die Methode soll eine beliebige Anzahl an Parametern vom Typ \T{int} erwarten und für jede dieser Zahlen die Quersumme
    berechnen und diese aufsummieren. Zusätzlich soll eine weitere boolean Parameter angeben, ob aus der Summe wiederum die Quersumme berechnet werden soll. Testen Sie Ihre Implementierung mit den angebenden Beispielen.}\medskip

    \onslide<4->{\textbf{Beispiele:}\\
\T{Ja, 123, 92, 57, 30 $\to$ 6 + 11 + 12 + 3 = 32 $\to$ 5}\\
\T{Nein, 12, 9, 4 $\to$ 3 + 9 + 4 = 16}}
    }\vspace*{-2.25\baselineskip}
\onslide<5->{\centerline{\hspace*{15mm}\begin{tikzpicture}
    \pingu[right wing shock,left wing wave, eyes wink,halo,name=calline]
    \node[shape=cloud callout,above right=5mm,xshift=2.85cm,draw,callout absolute pointer={([xshift=12mm]calline-head-left)},callout pointer segments=3,cloud ignores aspect,cloud puffs=25,scale=.45,inner sep=0pt] (@) at(calline-wing-left-tip) {
        \vspace*{-3mm}\hspace*{3mm}\copy\codebox
    };
    \node[opacity=.75,gray] at ([yshift=.25cm]@) {Blatt 3, Aufgabe 3};
\end{tikzpicture}}}
\end{frame}

{
\begin{frame}[fragile]{Meta-Quersummen}
\begin{itemize}[<+(1)->]
    \item<2-> Wir sollen mehrere Quersummen berechnen. Wir haben schon ein Programm für eine:\vfill\lstfs{8}
\begin{tikzpicture}[@O]
    \def\hlcolor{paletteA}
    \onslide<4->{\hlbehindcode{IntParse}{@>}}
    \onslide<8->{\hlbehindcode{return}{@2>}}
    \def\hlcolor{gray}
    \onslide<11->{\hlbehindcode{divrest1}{@3>}}
    \onslide<13->{\hlbehindcode{divrest2}{@4>}
    \hlbehindcode{divrest3}{@5>}}
    \def\hlcolor{paletteA}
    \onslide<15->{\hlbehindcode{return2}{@6>}}
\end{tikzpicture}
\columns[onlytextwidth,c]
\column{.475\linewidth}
\begin{plainjava}
!*\onslide<3->*!public static void main(String[] args) {
!*\onslide<3->*!   !*\Snode{IntParse}*!int zahl = Integer.parseInt(args[0]);!*\Snode{IntParse@}*!
!*\onslide<3->*!   if (zahl < 0) {
!*\onslide<3->*!      System.err.println("Eingabe doof!");
!*\onslide<3->*!      !*\Snode{return}*!return;!*\Snode{return@}*!
!*\onslide<3->*!   }
!*\onslide<3->*!   int quersumme = 0;
!*\onslide<3->*!   !*\Snode{divrest1}*!int divRest;!*\Snode{divrest1@}*!
!*\onslide<3->*!   while (zahl > 0) {
!*\onslide<3->*!      !*\Snode{divrest2}*!divRest = zahl % 10;!*\Snode{divrest2@}*!
!*\onslide<3->*!      quersumme += !*\Snode{divrest3}*!divRest!*\Snode{divrest3@}*!;
!*\onslide<3->*!      zahl = zahl / 10;
!*\onslide<3->*!   }
!*\onslide<3->*!   !*\Snode{return2}*!System.out.println(quersumme);!*\Snode{return2@}*!
!*\onslide<3->*!}
\end{plainjava}
\column{0pt}
\clap{$\to$}
\column{.475\linewidth}
\makeatletter
\begin{plainjava}
!*\onslide<5->*!public static int quersumme(!*\tikzmarknode{IntParseNew}{\sbasic{\skB{int} zahl}}*!) {
!*\onslide<6->*!    if(zahl < 0){
!*\onslide<7->*!        System.err.println("Eingabe doof!");
!*\onslide<9->*!        !*\tikzmarknode{returnNew}{\skA{return}}*! 0;
!*\onslide<6->*!    }
!*\onslide<10->*!    int quersumme = 0;
!*\onslide<12->*!    while(zahl > 0){
!*\onslide<13->*!        quersumme = quersumme + (zahl !*\tikzmarknode{percentNew}{\sbasic{\%}}*! 10);
!*\onslide<14->*!        zahl = zahl / 10;
!*\onslide<12->*!    }
!*\onslide<16->*!    !*\tikzmarknode{return2New}{\skA{return}}*! quersumme;
!*\onslide<5->*!}
\end{plainjava}
\endcolumns
\end{itemize}
\begin{tikzpicture}[@O,K/.style={above,gray,font=\footnotesize\sffamily},every path/.append style={line cap=round}]
    \onslide<5->{\draw[lightgray,-Kite] (@>) to[out=5,in=165,edge node={node[K] {\only<0>{Ein allgemeinerer Eingabeparameter}}}] ([yshift=.5mm]IntParseNew.north);}
    \onslide<9->{\draw[lightgray,-Kite] (@2>) to[out=0,in=175,edge node={node[K,below] {\only<0>{Rückgabetyp!}}}] ([xshift=-1mm]returnNew.west);}

    \onslide<13->{\draw[lightgray,-Circle] (@4>) -- ++(7mm,0) coordinate (@);
    \draw[lightgray] (@3>) to[out=0,in=135] (@);
    \draw[lightgray] (@5>) to[out=0,in=225] (@);
    \draw[lightgray,-Kite] (@) to[out=0,in=182] ([yshift=-.25mm]percentNew.south);}
    \only<0>{\node[K,lightgray,above right,rotate=-1] at(@) {Optionale Vereinfachung};}

    \onslide<16->{\draw[lightgray,-Kite] (@6>) to[out=0,in=175,edge node={node[K,below=.4cm] {\only<0>{Ordentliche Rückgabe}}}] ([xshift=-1mm]return2New.west);}
    \onslide<17->{\node[above left,scale=.8,yshift=\btdmfootheight+.25mm] at(current page.south east) {It is always a great time to be a good \link{https://www.oreilly.com/library/view/97-things-every/9780596809515/ch08.html}{boy or girl scout}!};}
\end{tikzpicture}
\end{frame}

\begin{frame}[fragile]{Beliebige Quersummen brummen}
\SetupLstHl
\begin{itemize}
    \item<2-> Die beliebige Anzahl ints schaffen wir mit \link{https://docs.oracle.com/javase/8/docs/technotes/guides/language/varargs.html}{varargs}:
\begin{plainjava}
!*\onslide<3->*!public class Quersummen {
!*\onslide<4->*!   |ihl|public static int quersumme(int zahl) { ... }|ihl|


!*\onslide<5->*!   public static int !*\tikzmarknode{quersummeVonQuersummen}{quersummeVonQuersummen}*!(boolean !*\tikzmarknode{sum2}{sum2}*!, int... zahlen) {
!*\onslide<6->*!       int quersummenSumme = 0;
!*\onslide<7->*!       for(int zahl : zahlen)
!*\onslide<8->*!           quersummenSumme += quersumme(zahl);
!*\onslide<3->*!
!*\onslide<9->*!       return sum2 ? !*\onslide<10->*!quersumme(quersummenSumme) : !*\onslide<10->*!quersummenSumme!*\onslide<9->*!;
!*\onslide<5->*!  }
!*\onslide<3->*!}
\end{plainjava}
\end{itemize}
\begin{tikzpicture}[@O]
    \onslide<11->{\draw[lightgray,-Kite] (sum2.north) to[bend left] ++(.5,1.15) node[below right,yshift=1.2\dp\strutbox,T,text width=3.4cm] {Hier nur ein kurzer Name, damit es auf die Folie passt};}
    \onslide<12->{\draw[lightgray,-Kite] (quersummeVonQuersummen.300) to[bend right] ++(.5,-.6) node[right,T,align=left,yshift=+1\dp\strutbox] {Signatur?\\\T{quersummeVonQuersummen(boolean, int...)}};}
\end{tikzpicture}
\end{frame}

\begin{frame}[fragile]{Ich bin die Biene!}
\SetupLstHl
\begin{itemize}[<+(1)->]
    \item Jetzt fehlt noch die Methode zum Testen:\medskip
\begin{plainjava}
!*\onslide<3->*!public class Quersummen {
!*\onslide<4->*!    |ihl|public static int quersumme(int zahl) { ... }|ihl|
!*\onslide<4->*!    |ihl|public static int quersummeVonQuersummen(boolean sum2, int... zahlen)|ihl|
!*\onslide<4->*!        |ihl|{ ... }|ihl|


!*\onslide<5->*!    public static void main(String[] args) {
!*\onslide<6->*!        System.out.println(quersummeVonQuersummen(true, 123, 92, 57, 30));
!*\onslide<7->*!        System.out.println(quersummeVonQuersummen(false, 12, 9, 4));
!*\onslide<5->*!    }
!*\onslide<3->*!}
\end{plainjava}
\end{itemize}
\begin{tikzpicture}[@O]
\iffull
    \onslide<8->{\node[above left,xshift=4mm,yshift=1.5mm,scale=.35] (bee) at(current page.south east) {\rotatebox{20}{\copy\beebox}};
    \node[T,left] at (bee.west) {Grüße aus einem \textup{summ}derbaren letzten Semester};}
    \only<9->{\node[above left,yshift=\btdmfootheight,xshift=-2.5cm] at (current page.south east) {\textattachfile{\curpath Quersummen.java}{Quersummen.java}\;};}
\else
    \only<8->{\node[above left,yshift=\btdmfootheight] at (current page.south east) {\textattachfile{\curpath Quersummen.java}{Quersummen.java}\;};}
\fi
\end{tikzpicture}
\end{frame}
}
\subsection{Aufgabe 2}
{
\MakeThePinguExplainIt[text width=7cm]{cap=!hide,headband,cup=!hide,heart=shadeA,right item angle=-120}{Implizite Typkovertierung und Unterprogramme können hier sehr hilfreich sein.}
\begin{frame}{Aufgabe 2: Globale Variablen}
    \task<2->{Legen Sie eine Java Datei namens \T{CharRotation.java} an. Innerhalb dieser Datei sollen Sie die folgende Aufgabe implementieren.\medskip\par
    \onslide<3->{Implementieren Sie eine Methode namens \T{rotiereCharacterArray}, die ein char Array als Parameter erwartet und innerhalb dieses Arrays (\textit{in place}) alle Klein- sowie Großbuchstaben um \T{n} Stellen zyklisch und alphabetisch verschiebt.
    Legen Sie n als statische globale Konstante an. Dabei sollen Klein- und Großbuchstaben erhalten bleiben. Testen Sie Ihre Implementierung mit mindestens einem Beispiel.}\medskip\par
    \onslide<4->{\textbf{Beispiel:}\\
    \T{n = 3, \{'a', 'Z'\} $\to$ \{'d', 'C'\}}
}}
\begin{tikzpicture}[@O]
    \onslide<5->{\node[above left,yshift=\btdmfootheight] at(current page.south east) {\copy\pinguexplainbox};}
\end{tikzpicture}
\end{frame}
}

\begin{frame}[fragile]{Problemreduktion}
    \begin{itemize}[<+(1)->]
        \itemsep5pt
        \item Wir reduzieren das Problem, alle Zeichen zu verschieben, zunächst auf ein Zeichen
        \item Bei Hilfsmethoden stellt sich die Frage, ob sie semantisch alleine sinnvoll sind\begin{itemize}
            \itemsep2pt
            \item Ist es sinnvoll nur ein einziges Zeichen zu rotieren?~\pause Ja
            \item Benötigen wir kontextabhängige Informationen?~\pause Nicht wirklich (\T{n} ist konstant)
            \item Haben wir implizite Annahmen die gelten müssen?~\pause Auch nicht
            \item In dem Fall empfiehlt sich \bjava{public}, sonst eher \bjava{private}
        \end{itemize}
\begin{plainjava}
!*\onslide<11->*!public static char rotiereCharacter(char c){
!*\onslide<11->*!
!*\onslide<11->*!}
\end{plainjava}
        \item<12-> Nun prüfen wir weiter ob es ein Großbuchstaben, ein Kleinbuchstaben oder ein sonstiges Zeichen ist.
    \end{itemize}
\end{frame}

\begin{frame}[fragile]{Hilfsmethodenfreuden}
\SetupLstHl
\begin{onlyenv}<2-8|handout:0>
\begin{plainjava}
!*\onslide<2->*!public static char rotiereCharacter(char c){
!*\onslide<3->*!    if (!*\onslide<4->*!c >= 'a' && c <= 'z'!*\onslide<3->*!) {
!*\onslide<6->*!        return /*?*/;
!*\onslide<3->*!    } else if (!*\onslide<5->*!c >= 'A' && c <= 'Z'!*\onslide<3->*!) {
!*\onslide<6->*!        return /*?*/;
!*\onslide<3->*!    } else {
!*\onslide<6->*!        return c;
!*\onslide<3->*!    }
!*\onslide<2->*!}

!*\onslide<7->*!private static boolean isLowercase(char c) {
!*\onslide<8->*!    return c >= 'a' && c <= 'z';
!*\onslide<7->*!}
\end{plainjava}
\end{onlyenv}
\begin{onlyenv}<9|handout:0>
\begin{plainjava}
public static char rotiereCharacter(char c){
    if (isLowercase(c)) {
        return /*?*/;
    } else if (c >= 'A' && c <= 'Z') {
        return /*?*/;
    } else {
        return c;
    }
}

private static boolean isLowercase(char c) {
    return c >= 'a' && c <= 'z';
}
\end{plainjava}
\end{onlyenv}
\begin{onlyenv}<10|handout:0>
\begin{plainjava}
public static char rotiereCharacter(char c){
    if (isLowercase(c)) {
        return /*?*/;
    } else if (isUppercase(c)) {
        return /*?*/;
    } else {
        return c;
    }
}

private static boolean isLowercase(char c) {
    return c >= 'a' && c <= 'z';
}
private static boolean isUppercase(char c) {
    return c >= 'A' && c <= 'Z';
}
\end{plainjava}
\end{onlyenv}
\begin{onlyenv}<11|handout:1>
\begin{plainjava}
public static char rotiereCharacter(char c){
    if (isLowercase(c)) {
        return /*?*/;
    } else if (isUppercase(c)) {
        return /*?*/;
    } else {
        return c;
    }
}
\end{plainjava}
\lstfs{9}
\begin{plainjava}
|ihl|private static boolean isLowercase(char c) { return c >= 'a' && c <= 'z'; }|ihl|
|ihl|private static boolean isUppercase(char c) { return c >= 'A' && c <= 'Z'; }|ihl|
\end{plainjava}
\end{onlyenv}
\end{frame}

{\AddonFrame
\begin{frame}{Refactoring}
 TODO: eigentlich wird von hinten aufgerollt usw.
\end{frame}
}

\begin{frame}{TODO:}
    TODO: toString machen

    Bauplananalogie etc.
    TODO: auf this eingehen TODO: start überlagern  und soo
    TOPDO: heap-darstellung mit mehr chaos für Chlassen

    TODO: private hilfsmethoden ansprechen
    TODO: Ausblick mit Gültigkeits- und Sichtbarkeitsbereichen
    TODO: Signatur nochmal wiederholen usw
    TODO: überschatten überlagern

    TODO: Klassenentwurf vom letztenmal :D
    TODO: Begriffe fundieren
    TODO: animiere CharRotation stuff mit heap und stack und sooo :eyes:
\end{frame}



\SetNextSectionText{Grundlagen der OOP\\Abgabe: \DTMDate{2022-06-07}}
\section{Aussicht: Übungsblatt 6}
\begin{frame}{}

\end{frame}

\begin{frame}
    Aussicht
\end{frame}

\SetNextSectionText[.55\linewidth]{TODO}
\section{Abschließendes}
{\SummaryFrame
\begin{frame}[t]{Zusammenfassend}
\pause \printBibCommand
\vfill\vfill % double fill for more fraction
\begin{itemize}[<+(1)->]
    \itemsep6.5pt
    \item TODO
\end{itemize}
\end{frame}
}



\outro{\vskip9mm\centering \onslide<2->{\begin{tikzpicture}

\end{tikzpicture}}}

\iffull\end{document}\fi
