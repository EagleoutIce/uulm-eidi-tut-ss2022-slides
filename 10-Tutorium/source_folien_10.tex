\InputIfFileExists{../data/global.tex}\relax\relax

\iffull
\title{Kreisrunde Bäume --- Halin Graphen?}
\subtitle{Datastructures going wild}
\date{KW 28}
\addbibresource{references.bib}
\fi
\SetTutoriumNumber{10}

\iffull\begin{document}
\titleframe

\TopicOverview{6}
\fi

\iffull{\SummaryFrame
\def\treenode[#1](#2)#3#4{\node[draw,rounded corners=2pt,minimum width=1.25cm,minimum height=7mm,align=center,#1] (#3) at (#2) {#4\\[-.35mm]};\draw([yshift=-.5mm]#3.east) -- ([yshift=-.5mm]#3.west) coordinate[pos=.5] (@); \draw(@) -- (#3.south);}
\begin{frame}[c]{Kurzwiederholung}
\small\begin{itemize}[<+(1)->]
   \itemsep12pt
    \item Wir kennen eine Reihe an dynamischen Datenstrukturen: \begin{description}
        \itemsep10pt
        \item[Einfach verkettete Liste:] \pause\hfill\tikzset{W/.style={draw, rounded corners=2pt},K/.style={W, rectangle split, rectangle split parts=2,rectangle split horizontal}}\scalebox{.8}{\begin{tikzpicture}[align-half-base]
        \node[K] (a) at (0,0) {4\nodepart{two}};
        \node[K,right=2mm] (b) at (a.east) {5\nodepart{two}};
        \node[K,right=2mm] (c) at (b.east) {1\nodepart{two}};
        \node[K,right=2mm] (d) at (c.east) {3\nodepart{two}};
        \node[W,right=2mm] (e) at (d.east) {\phantom{2}};
        \pgfinterruptboundingbox
        \node[below=-.33mm,T] at (a.south) {head};
        \node[below=-.33mm,T] at (d.south) {(tail)};
        \endpgfinterruptboundingbox
        \draw[shorten <= .4mm,shorten >= .4mm,] (e.south west) -- (e.north east);
        \draw[Circle-Kite] ([xshift=-1.65ex]a.east) -- (b.west);
        \draw[Circle-Kite] ([xshift=-1.65ex]b.east) -- (c.west);
        \draw[Circle-Kite] ([xshift=-1.65ex]c.east) -- (d.west);
        \draw[Circle-Kite] ([xshift=-1.65ex]d.east) -- (e.west);
    \end{tikzpicture}}
    % this is horrible, but i have no time
    \item[Doppelt verkettete Liste:] \pause\hfill\tikzset{W/.style={draw, rounded corners=2pt},K/.style={W, rectangle split, rectangle split parts=3,rectangle split horizontal}}\scalebox{.8}{\begin{tikzpicture}[align-half-base]
        \node[K] (a) at (0,0) {\nodepart{two}4\nodepart{three}};
        \node[W,left=2mm] (f) at (a.west) {\phantom{2}};
        \node[K,right=2mm] (b) at (a.east) {\nodepart{two}5\nodepart{three}};
        \node[K,right=2mm] (c) at (b.east) {\nodepart{two}1\nodepart{three}};
        \node[K,right=2mm] (d) at (c.east) {\nodepart{two}3\nodepart{three}};
        \node[W,right=2mm] (e) at (d.east) {\phantom{2}};
        \pgfinterruptboundingbox
        \node[below=-.33mm,T] at (a.south) {head};
        \node[below=-.33mm,T] at (d.south) {tail};
        \endpgfinterruptboundingbox
        \draw[shorten <= .4mm,shorten >= .4mm,] (e.south west) -- (e.north east);
        \draw[shorten <= .4mm,shorten >= .4mm,] (f.south west) -- (f.north east);
        \draw[Circle-Kite] ([yshift=1mm,xshift=-1.65ex]a.east) -- ([yshift=1mm]b.west);
        \draw[Circle-Kite] ([yshift=-1mm,xshift=1.65ex]b.west) -- ([yshift=-1mm]a.east);
        \draw[Circle-Kite] ([yshift=1mm,xshift=-1.65ex]b.east) -- ([yshift=1mm]c.west);
        \draw[Circle-Kite] ([yshift=-1mm,xshift=1.65ex]c.west) -- ([yshift=-1mm]b.east);
        \draw[Circle-Kite] ([yshift=1mm,xshift=-1.65ex]c.east) -- ([yshift=1mm]d.west);
        \draw[Circle-Kite] ([yshift=-1mm,xshift=1.65ex]d.west) -- ([yshift=-1mm]c.east);
        \draw[Circle-Kite] ([xshift=-1.65ex]d.east) -- (e.west);
        \draw[Circle-Kite] ([xshift=1.65ex]a.west) -- (f.east);
    \end{tikzpicture}}
    \end{description}
\end{itemize}\medskip
    \tikzset{W/.style={draw, rounded corners=2pt}}%
    \begin{columns}[onlytextwidth,T]
\column{.4\linewidth}
\onslide<6->{\textbf{Einfach verketteter (Binär-)Baum:}\smallskip\par}
\onslide<7->{\scalebox{.8}{\begin{tikzpicture}[align-half-base]
    \treenode[](0,0)a{4};
    \treenode[yshift=-2mm,below left=2mm](a.south)b{5};
    \treenode[yshift=-2mm,below right=2mm](a.south)c{1};
    \treenode[yshift=-2mm,below left=2mm](b.south)d{3};

    \node[W,yshift=-2mm,below left=2mm] (e) at (d.south) {\phantom{2}};
    \node[W,yshift=-2mm,below right=2mm] (f) at (d.south) {\phantom{2}};
    \node[W,yshift=-2mm,below left=2mm] (g) at (c.south) {\phantom{2}};
    \node[W,yshift=-2mm,below right=2mm] (h) at (c.south) {\phantom{2}};
    \node[W,yshift=-2mm,below right=2mm] (i) at (b.south) {\phantom{2}};
    \foreach \a in {e,...,i} {
        \draw[shorten <= .4mm,shorten >= .4mm,] (\a.south west) -- (\a.north east);
    }
    \foreach \fr/\tol/\tor in {a/b/c,b/d/i,c/g/h,d/e/f} {
        \draw[Circle-Kite] ([xshift=-1.25cm/4,yshift=2mm]\fr.south) -- (\tol.north);
        \draw[Circle-Kite] ([xshift=1.25cm/4,yshift=2mm]\fr.south) -- (\tor.north);
    }
    \node[left,T] at (a.west) {head};

\end{tikzpicture}}}
\column{.6\linewidth}
\begin{itemize}
    \itemsep\dimexpr-\smallskipamount-2pt\relax
    \item<8-> Diese erlauben es, abstrakter Datentypen umsetzen:\smallskip
    \item<9->[]\textbf{Liste:} \onslide<10->{Elemente vorne und hinten hinzufügen, Zugriff auf das \(i\)-te Element, ist die Liste leer?,~\ldots}
    \item<11->[]\textbf{Queue (FIFO):} \onslide<12->{Elemente hinzufügen und entfernen (in Einfügereihenfolge), ist die Queue leer?,~\ldots}
    \item<13->[]\textbf{Stack (LIFO):} \onslide<14->{Elemente hinzufügen und entfernen (umgekehrte Einfügereihenfolge), ist der Stack leer?,~\ldots}\smallskip
    \item<15-> All diese abstrakten Datentypen haben ein klares (meist mathematisch definiertes) Verhalten.
\end{itemize}
    \end{columns}
\end{frame}
}\fi

% \SetNextSectionText[.6\linewidth]{TODO}
\section{Präsenzaufgabe}
{
\begin{frame}[fragile,c]{Präsenzaufgabe}
\begin{aufgabe}{Traverse My-Tree}
\begin{onlyenv}<2|handout:0>
    In dieser Aufgabe sollen Sie binären Bäume aus Arrays wachsen lassen und diese anschließend in einem Breitendurchlauf traversieren. Hierfür finden Sie auf Moodle neben diesem  Übungsblatt weitere Java Files (\FileMarkerAttach[p/]{BinaryTree.java}, \FileMarkerAttach[p/]{BinaryTreeMain.java}, \FileMarkerAttach[p/]{IntegerNode.java}), die Sie für Ihre Implementierung verwenden dürfen. Verwenden Sie
    ansonsten keine vorgefertigten dynamischen Datenstrukturen. Für die Darstellung der Knoten können Sie die vorgegeben Klasse \bjava{IntegerNode} verwenden, welche der einfach verketteten Version aus der Vorlesung entspricht.

    Sie sollen die Methode \bjava{public void breadthFirstTraversal()} implementieren, welche den Baum in der Breite durchläuft und die Werte der einzelnen Knoten ausgibt. Hierfür werden Sie die vorgegebene Klasse Queue benötigen, welche \bjava{IntegerNode} Elemente einreihen kann.

    \textbf{Zahlenbeispiel}\quad Im Breitendurchlauf traversiert ergibt sich für diesen Baum folgende Reihenfolge:
    \begin{center}
        \begin{forest}
            for tree={circle,draw}
            [9[1[3][6]][7[5][,phantom]]]
        \end{forest}
    \end{center}
\end{onlyenv}%
\begin{onlyenv}<3->%
    \SetupLstHl % TODO: animation
    \vspace*{-1.25\baselineskip}\par\footnotesize Sie sollen binären Bäume mit einem Breitendurchlauf traversieren. Verwenden Sie
    neben den gegebenen (\FileMarkerAttach[p/]{BinaryTree.java}, \FileMarkerAttach[p/]{BinaryTreeMain.java}, \FileMarkerAttach[p/]{Queue.java}, \FileMarkerAttach[p/]{IntegerNode.java}) keine vorgefertigten dynamischen Datenstrukturen (sie entsprechen der Vorlesung).
    Implementieren Sie die Methode \bjava{public void breadthFirstTraversal()}, welche den Baum im Breitendurchlauf durchläuft und die Werte der einzelnen Knoten ausgibt. Hierfür werden Sie \bjava{Queue} benötigen.\\
    \textbf{Zahlenbeispiel}\quad Im Breitendurchlauf traversiert ergibt sich für diesen Baum folgende Reihenfolge:
    \lstfs{6}\vspace*{-.425\baselineskip}%
\begin{columns}[onlytextwidth,c]
    \column{.16\linewidth}
    \scalebox{.7}{\begin{forest}
        for tree={circle,draw}
        [9,name=9[1,name=1[3,name=3][6,name=6]][7,name=7[5,name=5][,phantom]]]
        \foreach[count=\i from 1] \a in {9,1,7,3,6,5} {
            \ifnum\i>3
                \node[below,paletteA] at(\a.south) {\i};
            \else \node[right,paletteA] at(\a.east) {\i};\fi
        }
    \end{forest}}
    \column{.425\linewidth}
\begin{plainjava}[morekeywords={[3]{BinaryTree, IntegerNode}}]
public class BinaryTree {
  private IntegerNode root;
  public BinaryTree(int[] items) { |ihl|...|ihl| }
  public void breadthFirstTraversal() { !*\faStar*! }
}
public class IntegerNode {
  public IntegerNode(int value) { |ihl|...|ihl| }
  public void setLeftChild(IntegerNode l) { |ihl|...|ihl| }
  public IntegerNode getLeftChild() { |ihl|...|ihl| }
  public void setValue(int value) { |ihl|...|ihl| }
  public int getValue() { |ihl|...|ihl| }
  ...
}
\end{plainjava}
    \column{.435\linewidth}
\begin{plainjava}[morekeywords={[3]{BinaryTree, IntegerNode}}]
public class Queue {
  class Element {
    public Element(IntegerNode node) { |ihl|...|ihl| }
    public void setNextElement(Element n) { |ihl|...|ihl| }
    public Element getNextElement() { |ihl|...|ihl| }
    public IntegerNode getNode() { |ihl|...|ihl| }
  }
  public Queue() { |ihl|...|ihl| }
  public void enqueue(IntegerNode node) { |ihl|...|ihl| }
  public IntegerNode dequeue() { |ihl|...|ihl| }
  public int getLength() { |ihl|...|ihl| }
  public boolean isEmpty() { |ihl|...|ihl| }
}
\end{plainjava}
    \end{columns}
\end{onlyenv}\vspace*{-1.1\baselineskip}%
\end{aufgabe}
\end{frame}
}

\begin{frame}{Die schematische Struktur}
    \begin{itemize}
        \item Pseudocode
        \item Dann überführung mit Queue
        \item Beispiel mit stack
    \end{itemize}
\end{frame}

\begin{frame}[fragile,c]{Let it be code}
\begin{minted}{java}
public void breadthFirstTraversal() {
    Queue queue = new Queue();
    queue.enqueue(root);

    while (!queue.isEmpty()) {
        IntegerNode node = queue.dequeue();

        if (node.getLeftChild() != null)
            queue.enqueue(node.getLeftChild());

        if (node.getRightChild() != null)
            queue.enqueue((node.getRightChild()));

        System.out.println(node.getValue());
    }
}
\end{minted}
\end{frame}

\SetNextSectionText{Dynamische Datenstrukturen\\Abgabe: \DTMDate{2022-07-11}}
\section{Übungsblatt 10}
\subsection{Aufgabe 1}
{\taskenum
\begin{frame}[c]{Aufgabe 1: Bäume}
    \taskblock<2->
Betrachten Sie den folgenden Binärbaum:\medskip
\begin{columns}[onlytextwidth,c]
\column{.4\linewidth}
\centering
\begin{forest}
    for tree={circle,draw}
    [9[6[3][2]][7[5][,phantom]]]
\end{forest}
\column{.6\linewidth}
\begin{enumerate}
    \item Welchen Verzweigungsgrad hat der Baum? Begründen Sie Ihre Antwort \textit{kurz}.
    \item Welche Tiefe hat der Baum?
    \item Wie viele Knoten hat der Baum und wie viele davon sind Blätter?
    \item Formt der Baum einen Max-Heap? Begründen Sie Ihre Antwort \textit{kurz}.
    \item In welcher Reihenfolge werden die Knoten besucht, wenn der Baum im Tiefendurchlauf traversiert wird?
    \item In welcher Reihenfolge werden die Knoten besucht, wenn der Baum im Breitendurchlauf traversiert wird?
\end{enumerate}
\end{columns}
    \endtaskblock
\end{frame}
}

\subsection{Aufgabe 2}
\begin{frame}{Self-Overriding Ring Buffer}
    \taskblock<2->TODO
    \endtaskblock
\end{frame}

\iffull
\SetNextSectionText{Dynamische Datenstrukturen II\\Abgabe: \DTMDate{2022-07-18}}
\section{Aussicht: Übungsblatt 11}
\subsection{Aufgabe 1}
\begin{frame}{Aufgabe 1: Graphen}
\begin{itemize}[<+(1)->]
    \item
\end{itemize}
\end{frame}

% \SetNextSectionText[.55\linewidth]{TODO}
\section{Abschließendes}
{\SummaryFrame
\begin{frame}[fragile,t]{Zusammenfassend}
% \printBibCommand
\lstfs{9}
\vfill\vfill % double fill for more fraction
\begin{itemize}[<+(1)->]
    \itemsep8pt
    \item
\end{itemize}
\end{frame}
}

\outro{\vskip9mm\centering \onslide<2->{\scalebox{1.35}{\begin{tikzpicture}

\end{tikzpicture}}}}

\iffull\end{document}\fi
