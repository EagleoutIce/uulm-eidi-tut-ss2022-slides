\InputIfFileExists{../data/global.tex}\relax\relax

\iffull
\title{Kreisrunde Bäume --- Halin Graphen?}
\subtitle{Datastructures going wild}
\date{KW 28}
\addbibresource{references.bib}
\fi
\SetTutoriumNumber{10}

\iffull\begin{document}
\titleframe

\TopicOverview{6}
\fi

\iffull{\SummaryFrame
\begin{frame}[c]{Kurzwiederholung\hfill{Sortierverfahren}}
\onslide<2->{\centering
\def\arraystretch{1.225}
\begin{tabular}{l*2{l}>{\footnotesize}{l}}
\toprule
              & \multicolumn{2}{c}{Laufzeit}  & \\
    \cmidrule{2-3}
    & \onslide<2->{best} & \onslide<2->{worst} & \onslide<2->{\normalsize Ansatz} \\
\midrule
    \onslide<3->{\tikzmarknode{itstart}{Bubble}}      & \onslide<3->{\(\O(n^2)\)\textcolor{gray}{, \(\O(n)\)}}       & \onslide<3->{\(\O(n^2)\)} & \onslide<4->{Vertausche benachbarte El., solange unsortiert.} \\
    \onslide<5->{Insertion} & \onslide<5->{\(\O(n)\)}             & \onslide<5->{\(\O(n^2)\)}      & \onslide<6->{Sortiere 1. unsortiertes El. in sortierten Teil ein.}\\
    \onslide<7->{\tikzmarknode{itend}{Selection}} & \onslide<7->{\(\O(n^2)\)}    & \onslide<7->{\(\O(n^2)\)}      & \onslide<8->{Kleinstes unsortiertes El. an Ende des sortierten Teils.} \\
    % Quick-, merge und Heapsort können auch O(n)
    \onslide<9->{\tikzmarknode{rekstart}{Merge}}    & \onslide<9->{\(\O(n \log n)\)} & \onslide<9->{\(\O(n \log n)\)}  & \onslide<10->{Aufteilen bis einel., wiederholtes mergen sortierter Teill.}\\
    \onslide<11->{Quick}                             & \onslide<11->{\(\O(n \log n)\)} & \onslide<11->{\(\O(n^2)\)}    & \onslide<12->{Pivot $\to$ Ende, \(l\) sol. \(<\), \(r\) sol. \(\geq\). Treffen \(\to\) tausche Pivot.}\\
    \onslide<13->{\tikzmarknode{rekend}{Heap}}       & \onslide<13->{\(\O(n \log n)\)} & \onslide<13->{\(\O(n\log n)\)} & \onslide<14->{Baue 4Heap, entferne wiederholt Wurzel, heapify.}\\
\bottomrule
\end{tabular}\medskip\par
\only<0|handout:0>{\scriptsize\textit{Dies folgt den Implementationen der Vorlesung. Bereits leichte Modifikationen (wie: prüfe zuerst ob die Liste bereits sortiert ist) können die Daten verändern (beispielsweise einen best-case von \(\O(n)\) im Falle von Bubblesort).}}}
\begin{tikzpicture}[@O]
    \onslide<15->{\draw[decoration={brace,mirror},decorate] ([xshift=-1.25mm]itstart.north west) to[edge node={node[left=1mm] {\rotatebox{90}{iterativ}}}] ([xshift=-1.25mm]itend.south west);}
    \onslide<16->{\draw[decoration={brace,mirror},decorate] ([xshift=-1.25mm]rekstart.north west) to[edge node={node[left=1mm] {\rotatebox{90}{rekursiv}}}] ([xshift=-1.25mm]rekend.south west);}
\end{tikzpicture}
\end{frame}
\def\treenode[#1](#2)#3#4{\node[draw,rounded corners=2pt,minimum width=1.25cm,minimum height=7mm,align=center,#1] (#3) at (#2) {#4\\[-.35mm]};\draw([yshift=-.5mm]#3.east) -- ([yshift=-.5mm]#3.west) coordinate[pos=.5] (@); \draw(@) -- (#3.south);}
\begin{frame}[c]{Kurzwiederholung\hfill{Datentypen Advanced}}
\small\begin{itemize}[<+(1)->]
   \itemsep12pt
    \item Wir kennen eine Reihe an dynamischen Datenstrukturen: \begin{description}
        \itemsep10pt
        \item<3->[Einfach verkettete Liste:] ~\onslide<4->{\hfill\tikzset{W/.style={draw, rounded corners=2pt},K/.style={W, rectangle split, rectangle split parts=2,rectangle split horizontal}}\scalebox{.8}{\begin{tikzpicture}[align-half-base]
        \node[K] (a) at (0,0) {4\nodepart{two}};
        \node[K,right=2mm] (b) at (a.east) {5\nodepart{two}};
        \node[K,right=2mm] (c) at (b.east) {1\nodepart{two}};
        \node[K,right=2mm] (d) at (c.east) {3\nodepart{two}};
        \node[W,right=2mm] (e) at (d.east) {\phantom{2}};
        \pgfinterruptboundingbox
        \node[below=-.33mm,T] at (a.south) {head};
        \node[below=-.33mm,T] at (d.south) {(tail)};
        \endpgfinterruptboundingbox
        \draw[shorten <= .4mm,shorten >= .4mm,] (e.south west) -- (e.north east);
        \draw[Circle-Kite] ([xshift=-1.65ex]a.east) -- (b.west);
        \draw[Circle-Kite] ([xshift=-1.65ex]b.east) -- (c.west);
        \draw[Circle-Kite] ([xshift=-1.65ex]c.east) -- (d.west);
        \draw[Circle-Kite] ([xshift=-1.65ex]d.east) -- (e.west);
    \end{tikzpicture}}}
    % this is horrible, but i have no time
    \item<5->[Doppelt verkettete Liste:] ~\onslide<6->{\hfill\tikzset{W/.style={draw, rounded corners=2pt},K/.style={W, rectangle split, rectangle split parts=3,rectangle split horizontal}}\scalebox{.8}{\begin{tikzpicture}[align-half-base]
        \node[K] (a) at (0,0) {\nodepart{two}4\nodepart{three}};
        \node[W,left=2mm] (f) at (a.west) {\phantom{2}};
        \node[K,right=2mm] (b) at (a.east) {\nodepart{two}5\nodepart{three}};
        \node[K,right=2mm] (c) at (b.east) {\nodepart{two}1\nodepart{three}};
        \node[K,right=2mm] (d) at (c.east) {\nodepart{two}3\nodepart{three}};
        \node[W,right=2mm] (e) at (d.east) {\phantom{2}};
        \pgfinterruptboundingbox
        \node[below=-.33mm,T] at (a.south) {head};
        \node[below=-.33mm,T] at (d.south) {tail};
        \endpgfinterruptboundingbox
        \draw[shorten <= .4mm,shorten >= .4mm,] (e.south west) -- (e.north east);
        \draw[shorten <= .4mm,shorten >= .4mm,] (f.south west) -- (f.north east);
        \draw[Circle-Kite] ([yshift=1mm,xshift=-1.65ex]a.east) -- ([yshift=1mm]b.west);
        \draw[Circle-Kite] ([yshift=-1mm,xshift=1.65ex]b.west) -- ([yshift=-1mm]a.east);
        \draw[Circle-Kite] ([yshift=1mm,xshift=-1.65ex]b.east) -- ([yshift=1mm]c.west);
        \draw[Circle-Kite] ([yshift=-1mm,xshift=1.65ex]c.west) -- ([yshift=-1mm]b.east);
        \draw[Circle-Kite] ([yshift=1mm,xshift=-1.65ex]c.east) -- ([yshift=1mm]d.west);
        \draw[Circle-Kite] ([yshift=-1mm,xshift=1.65ex]d.west) -- ([yshift=-1mm]c.east);
        \draw[Circle-Kite] ([xshift=-1.65ex]d.east) -- (e.west);
        \draw[Circle-Kite] ([xshift=1.65ex]a.west) -- (f.east);
    \end{tikzpicture}}}
    \end{description}
\end{itemize}\medskip
    \tikzset{W/.style={draw, rounded corners=2pt}}%
    \begin{columns}[onlytextwidth,T]
\column{.4\linewidth}
\onslide<7->{\textbf{Einfach verketteter (Binär-)Baum:}\smallskip\par}
\onslide<8->{\scalebox{.8}{\begin{tikzpicture}[align-half-base]
    \treenode[](0,0)a{4};
    \treenode[yshift=-2mm,below left=2mm](a.south)b{5};
    \treenode[yshift=-2mm,below right=2mm](a.south)c{1};
    \treenode[yshift=-2mm,below left=2mm](b.south)d{3};

    \node[W,yshift=-2mm,below left=2mm] (e) at (d.south) {\phantom{2}};
    \node[W,yshift=-2mm,below right=2mm] (f) at (d.south) {\phantom{2}};
    \node[W,yshift=-2mm,below left=2mm] (g) at (c.south) {\phantom{2}};
    \node[W,yshift=-2mm,below right=2mm] (h) at (c.south) {\phantom{2}};
    \node[W,yshift=-2mm,below right=2mm] (i) at (b.south) {\phantom{2}};
    \foreach \a in {e,...,i} {
        \draw[shorten <= .4mm,shorten >= .4mm,] (\a.south west) -- (\a.north east);
    }
    \foreach \fr/\tol/\tor in {a/b/c,b/d/i,c/g/h,d/e/f} {
        \draw[Circle-Kite] ([xshift=-1.25cm/4,yshift=2mm]\fr.south) -- (\tol.north);
        \draw[Circle-Kite] ([xshift=1.25cm/4,yshift=2mm]\fr.south) -- (\tor.north);
    }
    \node[left,T] at (a.west) {head};

\end{tikzpicture}}}
\column{.6\linewidth}
\begin{itemize}
    \itemsep\dimexpr-\smallskipamount-1.65pt\relax
    \item<9-> Diese erlauben es, abstrakter Datentypen umsetzen:\smallskip
    \item<10->[]\textbf{Liste:} \onslide<11->{Elemente vorne und hinten hinzufügen, Zugriff auf das \(i\)-te Element, ist die Liste leer?,~\ldots}
    \item<12->[]\textbf{Queue (FIFO):} \onslide<13->{Elemente hinzufügen und entfernen (in Einfügereihenfolge), ist die Queue leer?,~\ldots}
    \item<14->[]\textbf{Stack (LIFO):} \onslide<15->{Elemente hinzufügen und entfernen (umgekehrte Einfügereihenfolge), ist der Stack leer?,~\ldots}\smallskip
    \item<16-> All diese abstrakten Datentypen haben ein klares (meist mathematisch definiertes) Verhalten.
\end{itemize}
    \end{columns}
\end{frame}
}\fi

% \SetNextSectionText[.6\linewidth]{TODO}
\section{Präsenzaufgabe}
{
\begin{frame}[fragile,c]{Präsenzaufgabe}
\begin{aufgabe}{Traverse My-Tree}
\begin{onlyenv}<1-2|handout:0>
    \onslide<2->{In dieser Aufgabe sollen Sie binären Bäume aus Arrays wachsen lassen und diese anschließend in einem Breitendurchlauf traversieren. Hierfür finden Sie auf Moodle neben diesem  Übungsblatt weitere Java Files (\FileMarkerAttach<2>[p/]{BinaryTree.java}, \FileMarkerAttach<2>[p/]{BinaryTreeMain.java}, \FileMarkerAttach<2>[p/]{IntegerNode.java}), die Sie für Ihre Implementierung verwenden dürfen. Verwenden Sie
    ansonsten keine vorgefertigten dynamischen Datenstrukturen. Für die Darstellung der Knoten können Sie die vorgegeben Klasse \bjava{IntegerNode} verwenden, welche der einfach verketteten Version aus der Vorlesung entspricht.

    Sie sollen die Methode \bjava{public void breadthFirstTraversal()} implementieren, welche den Baum in der Breite durchläuft und die Werte der einzelnen Knoten ausgibt. Hierfür werden Sie die vorgegebene Klasse Queue benötigen, welche \bjava{IntegerNode} Elemente einreihen kann.

    \textbf{Zahlenbeispiel}\quad Im Breitendurchlauf traversiert ergibt sich für diesen Baum folgende Reihenfolge:
    \begin{center}
        \begin{forest}
            for tree={circle,draw}
            [9[1[3][6]][7[5][,phantom]]]
        \end{forest}
    \end{center}}
\end{onlyenv}%
\begin{onlyenv}<3->%
    \SetupLstHl % TODO: animation
    \vspace*{-1.25\baselineskip}\par\footnotesize Sie sollen binären Bäume mit einem Breitendurchlauf traversieren. Verwenden Sie
    neben den gegebenen (\FileMarkerAttach[p/]{BinaryTree.java}, \FileMarkerAttach[p/]{BinaryTreeMain.java}, \FileMarkerAttach[p/]{Queue.java}, \FileMarkerAttach[p/]{IntegerNode.java}) keine vorgefertigten dynamischen Datenstrukturen (sie entsprechen der Vorlesung).
    Implementieren Sie die Methode \bjava{public void breadthFirstTraversal()}, welche den Baum im Breitendurchlauf durchläuft und die Werte der einzelnen Knoten ausgibt. Hierfür werden Sie \bjava{Queue} benötigen.\\
    \onslide<4->{\textbf{Zahlenbeispiel}\quad Im Breitendurchlauf traversiert ergibt sich für diesen Baum folgende Reihenfolge:}
    \lstfs{6}\vspace*{-.55\baselineskip}%
\begin{columns}[onlytextwidth,c]
    \column{.16\linewidth}
    \onslide<4->{\scalebox{.7}{\begin{forest}
        for tree={circle,draw}
        [9,name=9[1,name=1[3,name=3][6,name=6]][7,name=7[5,name=5][,phantom]]]
        \foreach[count=\i from 1] \a in {9,1,7,3,6,5} {
            \ifnum\i>3
                \node[below,paletteA] at(\a.south) {\i};
            \else \node[right,paletteA] at(\a.east) {\i};\fi
        }
    \end{forest}}}
    \column{.425\linewidth}
\begin{plainjava}[morekeywords={[3]{BinaryTree, IntegerNode}}]
!*\onslide<5->*!public class BinaryTree {
!*\onslide<5->*!  private IntegerNode root;
!*\onslide<5->*!  public BinaryTree(int[] items) { |ihl|...|ihl| }
!*\onslide<5->*!  public void breadthFirstTraversal() { !*\faStar*! }
!*\onslide<5->*!}
!*\onslide<9->*!public class IntegerNode {
!*\onslide<9->*!  public IntegerNode(int value) { |ihl|...|ihl| }
!*\onslide<9->*!  public void setLeftChild(IntegerNode l) { |ihl|...|ihl| }
!*\onslide<9->*!  public IntegerNode getLeftChild() { |ihl|...|ihl| }
!*\onslide<9->*!  public void setValue(int value) { |ihl|...|ihl| }
!*\onslide<9->*!  public int getValue() { |ihl|...|ihl| }
!*\onslide<9->*!  |ihl|...|ihl|
!*\onslide<9->*!}!*\onslide<1->*!
\end{plainjava}
    \column{.435\linewidth}
\begin{plainjava}[morekeywords={[3]{BinaryTree, IntegerNode}}]
!*\onslide<6->*!public class Queue {
!*\onslide<7->*!  class Element {
!*\onslide<7->*!    public Element(IntegerNode node) { |ihl|...|ihl| }
!*\onslide<7->*!    public void setNextElement(Element n) { |ihl|...|ihl| }
!*\onslide<7->*!    public Element getNextElement() { |ihl|...|ihl| }
!*\onslide<7->*!    public IntegerNode getNode() { |ihl|...|ihl| }
!*\onslide<7->*!  }
!*\onslide<8->*!  public Queue() { |ihl|...|ihl| }
!*\onslide<8->*!  public void enqueue(IntegerNode node) { |ihl|...|ihl| }
!*\onslide<8->*!  public IntegerNode dequeue() { |ihl|...|ihl| }
!*\onslide<8->*!  public int getLength() { |ihl|...|ihl| }
!*\onslide<8->*!  public boolean isEmpty() { |ihl|...|ihl| }
!*\onslide<6->*!}!*\onslide<1->*!
\end{plainjava}
    \end{columns}
\end{onlyenv}\vspace*{-1.02\baselineskip}%
\end{aufgabe}
\end{frame}
}

\setbox\pinguA=\hbox{\scalebox{.7}{\begin{forest}
    for tree={circle,draw}
    [9,name=9[1,name=1[3,name=3][6,name=6]][7,name=7[5,name=5][,phantom]]]
    \foreach[count=\i from 1] \a in {9,1,7,3,6,5} {
        \ifnum\i>3
            \node[below,paletteA] at(\a.south) {\i};
        \else \node[right,paletteA] at(\a.east) {\i};\fi
    }
\end{forest}}}

\iffull{%
\immediate\write18{wget https://media.githubusercontent.com/media/EagleoutIce/Episode-Traversierung/gh-pages/preview-01.png -O logoTraversal-\jobname.png}
\AddonFrame
\begin{frame}[c]{Wääärbuuuung}
    \hypertarget<2>{episode-trav-jmp}{}\begin{center}
        \scalebox{.75}{\begin{tikzpicture}[align-base]
            \onslide<2->{\draw[thick,darkgray,rounded corners=2.5pt,path picture={\node at(path picture bounding box.center) {\href{https://media.githubusercontent.com/media/EagleoutIce/Episode-Traversierung/gh-pages/noanim_traversal.pdf}{\includegraphics[width=8.5cm,height=4.788cm,keepaspectratio]{logoTraversal-\jobname.png}}};}] (0,0) rectangle (8.5cm,4.788cm);}
        \end{tikzpicture}}\qquad\onslide<2->{\fancyqr{https://github.com/EagleoutIce/Episode-Traversierung}}
    \end{center}
\end{frame}}\fi

% TODO: animation?
\MakeThePinguExplainIt[text width=7.65cm,yshift=-.33cm]{cap=!hide,glasses=!hide,eyes wink,strawhat,blush,right item angle=-112}{Es lässt sich auch mit \bjava{if(node == null)} beim Entnehmen prüfen.}
\begin{frame}[fragile,c]{Let it be code}
\begin{minted}[morekeywords={[3]{BinaryTree, IntegerNode}}]{java}
!*\CodeFileMarkerAttach<11->[p-sol/]{BinaryTree.java}\AddCodeFileMarkerAttach<11->[p-sol/]{BinaryTreeMain.java} \AddCodeFileMarkerAttach<11->[p-sol/]{IntegerNode.java}\AddCodeFileMarkerAttach<11->[p-sol/]{Queue.java}*!
!*\onslide<2->*!public void breadthFirstTraversal() {
!*\onslide<3->*!    Queue queue = new Queue();
!*\onslide<4->*!    queue.enqueue(root);
!*\onslide<2->*!
!*\onslide<5->*!    while (!queue.isEmpty()) {
!*\onslide<6->*!        IntegerNode node = queue.dequeue();
!*\onslide<7->*!        if (node.getLeftChild() != null)
!*\onslide<8->*!            queue.enqueue(node.getLeftChild());
!*\onslide<9->*!        if (node.getRightChild() != null)
!*\onslide<9->*!            queue.enqueue(node.getRightChild());
!*\onslide<10->*!        System.out.println(node.getValue());
!*\onslide<5->*!    }
!*\onslide<2->*!}
\end{minted}
\begin{tikzpicture}[@O]
\onslide<2->{\node[below left,yshift=-1.33cm] at(current page.north east) {\copy\pinguA};}
\onslide<0|handout:1>{\node[above left,xshift=5mm,scale=.8,yshift=\btdmfootheight-5mm] at(current page.south east) {\copy\pinguexplainbox};}% copy for animations
\end{tikzpicture}
\end{frame}

\SetNextSectionText{Dynamische Datenstrukturen\\Abgabe: \DTMDate{2022-07-11}}
\section{Übungsblatt 10}
\subsection{Aufgabe 1}
{\taskenum
\begin{frame}[c]{Aufgabe 1: Bäume}
    \taskblock<2->
Betrachten Sie den folgenden Binärbaum:\medskip
\begin{columns}[onlytextwidth,c]
\column{.4\linewidth}
\centering
\onslide<2->{\begin{forest}
    for tree={circle,draw}
    [9[6[3][1]][7[5][,phantom]]]
\end{forest}}
\column{.6\linewidth}
\onslide<3->{\begin{enumerate}
    \item Welchen Verzweigungsgrad hat der Baum? Begründen Sie Ihre Antwort \textit{kurz}.
    \item Welche Tiefe hat der Baum?
    \item Wie viele Knoten hat der Baum und wie viele davon sind Blätter?
    \item Formt der Baum einen Max-Heap? Begründen Sie Ihre Antwort \textit{kurz}.
    \item In welcher Reihenfolge werden die Knoten besucht, wenn der Baum im Tiefendurchlauf traversiert wird?
    \item In welcher Reihenfolge werden die Knoten besucht, wenn der Baum im Breitendurchlauf traversiert wird?
\end{enumerate}}
\end{columns}
    \endtaskblock
\end{frame}

\setbox\pinguA=\hbox{\scalebox{.75}{\begin{forest}
    for tree={circle,draw}
    [9[6[3][1]][7[5][,phantom]]]
\end{forest}}}
\begin{frame}{Grundlagen Fragen}
\begin{enumerate}[<+(1)->]
    \itemsep8pt
    \item \task{Welchen Verzweigungsgrad hat der Baum? Begründen Sie Ihre Antwort \textit{kurz}.}
        \onslide<3->{\(2\), alle Knoten haben \textit{höchstens} zwei Kinder.}
    \item<4-> \task{Welche Tiefe hat der Baum?}\pause
        \onslide<5->{\(3\), der längste Pfad von einem Knoten zur Wurzel ist \(2\) lang.}
    \item<6-> \task{Wie viele Knoten hat der Baum und wie viele davon sind Blätter?}
        \onsldie<7->{\(6\) Knoten \info{\(\{3, 6, 9, 1, 7, 5\}\)}, \(3\) davon sind Blätter \info{Knoten ohne Kinder, \(\{3, 1, 5\}\)}}
    \item \task{Formt der Baum einen Max-Heap? Begründen Sie Ihre Antwort \textit{kurz}.}\pause
        Ja, die Elternknoten simd immer kleiner als alle ihre Kinder.
\end{enumerate}
\begin{tikzpicture}[@O]
    \node[below left,yshift=-1.33cm] at(current page.north east) {\copy\pinguA};
\end{tikzpicture}
\end{frame}

\MakeThePinguExplainIt[text width=6.65cm,yshift=-1.75cm]{cap=!hide,glasses=!hide,eyes wink,conical hat,blush,horse left,right item angle=-52,cup=!hide,horse left}{Neben Pre-, In- und Post-Order gibt es für spezifische Bäume (wie Binärbäume) auch noch weitere Versionen, die uns aber erstmal egal sind.}
\begin{frame}{Grundlagen Fragen\hfill II}
\begin{enumerate}[<+(1)->]
    \setcounter{enumi}{4}
    \itemsep8pt
    \item \task{In welcher Reihenfolge werden die Knoten besucht, wenn der Baum im Tiefendurchlauf\\traversiert wird?} \begin{description}[Post-Order:]
        \item[Prä-Order:]  \(9\), \(6\), \(3\), \(1\), \(7\), \(5\)
        \item[In-Order:]   \(3\), \(6\), \(1\), \(9\), \(5\), \(7\)
        \item[Post-Order:] \(3\), \(1\), \(6\), \(5\), \(7\), \(9\)
    \end{description}
    \onslide<6->{\infoblock{Unter Angabe der Variante (Prä-, In- oder Post-Order) reicht hier auch nur eine Version.}}
    \item<7-> \task{In welcher Reihenfolge werden die Knoten besucht, wenn der Baum im Breitendurchlauf traversiert wird?}
    \onslide<8->{\(9\), \(6\), \(7\), \(3\), \(1\), \(5\).}
\end{enumerate}
\begin{tikzpicture}[@O]
    \node[below left,yshift=-1.33cm] at(current page.north east) {\copy\pinguA};
    \onslide<9->{\node[above left,xshift=5mm,scale=.8,yshift=\btdmfootheight] at(current page.south east) {\copy\pinguexplainbox};}% copy for animations
\end{tikzpicture}
\end{frame}
}

\subsection{Aufgabe 2}
{\taskenum
\begin{frame}{Aufgabe 2: Self-Overriding Ring Buffer}
\begin{onlyenv}<handout:1|2-4>
    \vspace*{-.5\baselineskip}\taskblock<2->
    In dieser Aufgabe sollen Sie einen Ring Buffer implementieren. Hierbei handelt es sich um eine dynamische
    Datenstruktur, ähnlich zu einer \T{DoublyLinkedList} aus der Vorlesung, jedoch verweist hier jedes Element auf seinen
    Vorgänger und Nachfolger, wodurch sie einen Ring formen. Zusätzlich hat ein Ring Buffer eine fest vorgegebene
    Kapazität. Hierfür werden intern zwei Referenzen \T{head} und \T{tail} verwendet, die so Beginn und Ende des belegten Bereichs markieren. Alle Positionen bis zum \T{tail} sind beginnend beim \T{head} Zeiger belegt. Wir betrachten ein
    modifiziertes Ring Buffer, bei welchem alte Elemente überschrieben werden, falls der Buffer voll ist. Der Ring Buffer soll nach dem First--In--First--Out Prinzip arbeiten: neue Elemente werden an \T{tail} angefügt und es werden vom \T{head} aus Elemente abgearbeitet. Ein Beispiel finden Sie im Folgenden.\vspace*{-.33\baselineskip}
\onslide<3->{\begin{center}
    \begin{tikzpicture}[align-half-base]
        \draw (0,0) circle[radius=1cm];
        \draw (0,0) circle[radius=.825cm];
        \foreach[count=\i] \a in {0,20,...,359} {
            \draw (\a:.825cm) -- (\a:1cm);
            \coordinate (@\i-out) at (\a+10:1cm);
        }

        \draw[Kite-] (@3-out) -- ++(.33,.6) -- ++(.25,0) node[right] {Tail};
        \draw[Kite-] (@3-out) -- ++(.33,.25) -- ++(.25,0) node[right] {Head};

        \node at(0,0) {i)};
    \end{tikzpicture}\qquad\begin{tikzpicture}[align-half-base]
        \fill[shadeA] (60:.825cm) -- (60:1cm) arc (60:-20:1cm) -- (-20:.825cm) arc(-20:60:.825cm);
        \draw (0,0) circle[radius=1cm];
        \draw (0,0) circle[radius=.825cm];
        \foreach[count=\i] \a in {0,20,...,359} {
            \draw (\a:.825cm) -- (\a:1cm);
            \coordinate (@\i-out) at (\a+10:1cm);
        }
        \draw[Kite-] (@17-out) -- ++(.33,-.25) -- ++(.25,0) node[right] (tail) {Tail};
        \draw[Kite-] (@3-out) -- ++(.33,.25) -- ++(.25,0) node[right] (head) {Head};
        \node at(0,0) {ii)};
        \draw[|-|] (tail) to[bend right=10] (head);
    \end{tikzpicture}\qquad\begin{tikzpicture}[align-half-base]
        \fill[shadeA] (0:.825cm) -- (0:1cm) arc (0:360:1cm) -- (360:.825cm) arc(360:0:.825cm);
        \draw[pattern=north west lines,pattern color=gray] (60:.825cm) -- (60:1cm) arc (60:0:1cm) -- (0:.825cm) arc(0:60:.825cm);
        \draw[|-|] (60:1.25cm) to[bend left=30] (0:1.25cm);
        \draw (0,0) circle[radius=1cm];
        \draw (0,0) circle[radius=.825cm];
        \foreach[count=\i] \a in {0,20,...,359} {
            \draw (\a:.825cm) -- (\a:1cm);
            \coordinate (@\i-out) at (\a+10:1cm);
        }
        \draw[Kite-] (@18-out) -- ++(.33,-.6) -- ++(.25,0) node[right] {Head};
        \draw[Kite-] (@18-out) -- ++(.33,-.25) -- ++(.25,0) node[right] {Tail};
        \node at(0,0) {iii)};
    \end{tikzpicture}\medskip\par
\scalebox{.8}{\parbox{.6\linewidth}{\raggedright\tiny i) Der \T{RingBuffer} ist leer und \T{head}, sowie \T{tail} zeigen auf dasselbe Element.\quad ii) Teilweise belegt, nachdem vier Elemente eingefügt wurden und \T{head}, sowie \T{tail} nun auf verschiedenen Elemente zeigen.\quad iii) Der Buffer ist voll (\T{tail} hat \T{head} \say{überrundet}) und wird nun überschreieben.\only<4->{\quad Die Elemente sind wie eine \T{DoublyLinkedList} verbunden und haben zusätzlich eine Verknüpfung er beiden äußeren Elemente (unten rechts).}}}\hfill\tikzset{W/.style={draw, rounded corners=2pt},K/.style={W, rectangle split, rectangle split parts=3,rectangle split horizontal}}\onslide<4->{\scalebox{.8}{\begin{tikzpicture}[align-half-base]
    \node[K] (a) at (0,0) {\nodepart{two}7\nodepart{three}};
    \node[K,right=2mm] (b) at (a.east) {\nodepart{two}9\nodepart{three}};
    \node[K,right=2mm] (c) at (b.east) {\nodepart{two}2\nodepart{three}};
    \node[K,right=2mm] (d) at (c.east) {\nodepart{two}4\nodepart{three}};
    \pgfinterruptboundingbox
    \node[below=-.33mm,T] at (a.south) {head};
    \node[below=-.33mm,T] at (d.south) {tail};
    \endpgfinterruptboundingbox
    \draw[Circle-Kite] ([yshift=1mm,xshift=-1.65ex]a.east) -- ([yshift=1mm]b.west);
    \draw[Circle-Kite] ([yshift=-1mm,xshift=1.65ex]b.west) -- ([yshift=-1mm]a.east);
    \draw[Circle-Kite] ([yshift=1mm,xshift=-1.65ex]b.east) -- ([yshift=1mm]c.west);
    \draw[Circle-Kite] ([yshift=-1mm,xshift=1.65ex]c.west) -- ([yshift=-1mm]b.east);
    \draw[Circle-Kite] ([yshift=1mm,xshift=-1.65ex]c.east) -- ([yshift=1mm]d.west);
    \draw[Circle-Kite] ([yshift=-1mm,xshift=1.65ex]d.west) -- ([yshift=-1mm]c.east);
    \draw[Circle-Kite,rounded corners=1pt] ([xshift=-1.65ex]d.east) -- ++(0,.33) -| ([xshift=.65ex,yshift=-1mm]a.west);
    \draw[Circle-Kite,rounded corners=1pt] ([xshift=1.65ex]a.west) -- ++(0,.5) -| ([xshift=-.65ex,yshift=-1mm]d.east);
\end{tikzpicture}}}
\end{center}}
\endtaskblock
\end{onlyenv}
\begin{onlyenv}<handout:2|5->
\def\intaskfont{\scriptsize}
\begin{enumerate}[<+(1)->]
    \item<6-> \task{Nehmen Sie die Klassendefinition für \only<6->{\textattachfile{\curpath a2/Element.java}{Element}}, die Sie bereits aus der Vorlesung kennen. Erweitern
    Sie diese um eine private Referenz für den Vorgänger sowie passende getter und setter. Die Elemente sollen \bjava{int} halten.}
    \item<7-> \task{Um den Ring Buffer zu implementieren, definieren Sie zunächst die entsprechende Klasse \T{RingBuffer}. Der
   überladene Konstruktor soll die Kapazität als Parameter übernehmen und die Datenstruktur anlegen, indem
   der Kapazität entsprechend viele Elemente (siehe a)) angelegt und verknüpft werden. \T{head} und \T{tail} zeigen
   nun auf das erste Element (obige Abb. i). Wird als Kapazität ein negativer Wert übergeben, soll eine
   \T{NegativeArraySizeException} ausgelöst werden.}
    \item<8-> \task{Die Methode \T{public int peek()} der Klasse \T{RingBuffer} soll den Wert an der Position zurückliefern, auf die
    \T{head} zeigt. Sollte der Buffer leer sein, soll eine \T{NoSuchElementException} ausgelöst werden.}
    \item<9-> \task{Die Methode \T{public void put(int value)} der Klasse \T{RingBuffer} soll den übergebenen Wert an der
    Position im Ring Buffer speichern, auf die \T{tail} zeigt und \T{tail} auf den Nachfolger verschoben werden (obige Abb. ii). Sollte der Buffer komplett gefüllt sein, soll wieder von vorne begonnen werden und alte Elemente
    überschrieben werden (obige Abb. iii). In diesem Fall soll auch \T{head} auf den Nachfolger verschoben, damit
    \T{head} immer auf das älteste Element im Buffer zeigt und so die FIFO Eigenschaft gewahrt bleibt.}
    \item<10-> \task{Die Methode \T{public int remove()} der Klasse \T{RingBuffer} soll den Wert an der Position zurückliefern, auf
    die \T{head} zeigt und \T{head} um eine Stelle zurückverschoben werden. Diese Methode simuliert also ein Abarbeiten
    der Elemente im Buffer. Sollte der Buffer leer sein, soll eine NoSuchElementException ausgelöst werden.}
\end{enumerate}
\end{onlyenv}
\end{frame}

\begin{frame}[fragile,c]{Extend the Element}
\begin{enumerate}[<+(1)->]
    \item \task{Nehmen Sie die Klassendefinition für \T{Element}, die Sie bereits aus der Vorlesung kennen. Erweitern
    Sie diese um eine private Referenz für den Vorgänger sowie passende getter und setter. Die Elemente sollen \bjava{int} halten. \vspace*{-.25\baselineskip}}
\end{enumerate}
\SetupLstHl\lstfs{9}
\begin{onlyenv}<handout:0|3>
\begin{minted}[escapeinside={/*}{*/}]{java}
/*\CodeFileMarkerAttach<9->[a2-sol/]{Element.java}*/
class Element {
    private int value;
    private Element next;
    public Element() { this.next = null; }
    public void setNextElement(Element nextElement) { this.next = nextElement; }
    public Element getNextElement() { return this.next; }
    public void setValue(int value) { this.value = value; }
    public int getValue() { return this.value; }
}/*\onslide<2->*/
\end{minted}
\end{onlyenv}
\begin{onlyenv}<4->
\begin{minted}[escapeinside={/*}{*/}]{java}
/*\CodeFileMarkerAttach<9->[a2-sol/]{Element.java}*/
/*\onslide<1->*/|ihl|class Element {|ihl|
/*\onslide<1->*/|ihl|    private int value; private Element next;|ihl|
/*\onslide<5->*/    private Element previous;
/*\onslide<1->*/|ihl|    public Element() { this.next = null;  /*\onslide<6->*/|ihl|this.previous = null;|ihl| /*\onslide<1->*/}|ihl|
/*\onslide<1->*/|ihl|    public void setNextElement(Element n) { this.next = n; }|ihl|
/*\onslide<1->*/|ihl|    public Element getNextElement() { return this.next; }|ihl|
    /*\onslide<7->*/public void setPreviousElement(Element p) {
    /*\onslide<7->*/    this.previous = p;
    /*\onslide<7->*/}
    /*\onslide<8->*/public Element getPreviousElement() {
    /*\onslide<8->*/    return this.previous;
    /*\onslide<8->*/}
/*\onslide<1->*/|ihl|    public void setValue(int value) { this.value = value; }|ihl|
/*\onslide<1->*/|ihl|    public int getValue() { return this.value; }|ihl|
/*\onslide<1->*/|ihl|}|ihl|/*\onslide<1->*/
\end{minted}
\end{onlyenv}
\end{frame}

\begin{frame}[fragile]{Start des RingBuffers}
\begin{enumerate}[<+(1)->]
    \setcounter{enumi}{1}
    \item \task{Um den Ring Buffer zu implementieren, definieren Sie zunächst die entsprechende Klasse \T{RingBuffer}. Der
    überladene Konstruktor soll die Kapazität als Parameter übernehmen und die Datenstruktur anlegen, indem
    der Kapazität entsprechend viele Elemente (siehe a)) angelegt und verknüpft werden. \T{head} und \T{tail} zeigen
    nun auf das erste Element (obige Abb. i). Wird als Kapazität ein negativer Wert übergeben, soll eine
    \T{NegativeArraySizeException} ausgelöst werden}
\end{enumerate}
\SetupLstHl
\begin{minted}[escapeinside={/*}{*/}]{java}
/*\CodeFileMarkerAttach<8->[a2-sol/]{RingBuffer.java}*/
/*\onslide<3->*/public class RingBuffer {
/*\onslide<4->*/    private Element head;
/*\onslide<4->*/    private Element tail;
/*\onslide<5->*/    private boolean isEmpty;
/*\onslide<3->*/
/*\onslide<6->*/    public RingBuffer(int capacity) {
/*\onslide<7->*/        |ihl|...|ihl|
/*\onslide<6->*/    }
/*\onslide<3->*/}/*\onslide<1->*/
\end{minted}
\end{frame}
\begin{frame}[fragile]{Der Konstruktor}
\SetupLstHl
\begin{minted}[escapeinside={/*}{*/}]{java}
/*\onslide<2->*/|ihl|public class RingBuffer {|ihl|
/*\onslide<2->*/    |ihl|private Element head, tail; private boolean isEmpty;|ihl|
/*\onslide<3->*/    |ihl|RingBuffer(int capacity) {|ihl|
    /*\onslide<4->*/    if (capacity == 0) { this.isEmpty = false; return; }
    /*\onslide<5->*/    else if (capacity < 0) { throw new NegativeArraySizeException(); }
    /*\onslide<6->*/
    /*\onslide<6->*/    Element current = new Element();
    /*\onslide<7->*/    this.head = this.tail = current;
    /*\onslide<7->*/    this.isEmpty = true;
    /*\onslide<8->*/    current = fillBuffer(capacity, current);
    /*\onslide<9->*/    this.head.setPreviousElement(current);
    /*\onslide<10->*/    current.setNextElement(head);
/*\onslide<3->*/    |ihl|}|ihl|
/*\onslide<2->*/|ihl|}|ihl|/*\onslide<1->*/
\end{minted}
\end{frame}

\begin{frame}[fragile,c]{Der Konstruktor\hfill II}
\SetupLstHl\lstfs{9}
\begin{minted}[escapeinside={/*}{*/}]{java}
/*\onslide<2->*/|ihl|public class RingBuffer {|ihl|
/*\onslide<2->*/    |ihl|private Element head, tail; private boolean isEmpty;|ihl|
/*\onslide<3->*/    |ihl|RingBuffer(int capacity) { ... }|ihl|

    /*\onslide<4->*/private Element fillBuffer(int capacity, Element current) {
    /*\onslide<5->*/    for (int i = 1; i < capacity; i++) {
    /*\onslide<6->*/        Element next = new Element();
    /*\onslide<7->*/        current.setNextElement(next);
    /*\onslide<7->*/        next.setPreviousElement(current);
    /*\onslide<8->*/        if (i == capacity - 1)
    /*\onslide<8->*/            next.setNextElement(head);
    /*\onslide<9->*/        current = next;
    /*\onslide<5->*/    }
    /*\onslide<10->*/    return current;
    /*\onslide<4->*/}
/*\onslide<2->*/|ihl|}|ihl|/*\onslide<1->*/
\end{minted}
\end{frame}

\begin{frame}[fragile]{Modern Peek}
\begin{enumerate}[<+(1)->]
    \setcounter{enumi}{2}
    \item \task{Die Methode \T{public int peek()} der Klasse \T{RingBuffer} soll den Wert an der Position zurückliefern, auf die
    \T{head} zeigt. Sollte der Buffer leer sein, soll eine \T{NoSuchElementException} ausgelöst werden.}
\end{enumerate}
\SetupLstHl
\begin{minted}[escapeinside={/*}{*/}]{java}
/*\onslide<3->*/public class RingBuffer {
/*\onslide<3->*/    |ihl|private Element head, tail; private boolean isEmpty;|ihl|
/*\onslide<3->*/    |ihl|RingBuffer(int capacity) { ... }|ihl|
/*\onslide<3->*/
/*\onslide<4->*/    public int peek() {
/*\onslide<5->*/        if (this.isEmpty)
/*\onslide<6->*/            throw new NoSuchElementException();
/*\onslide<5->*/        else
/*\onslide<7->*/            return head.getValue();
/*\onslide<4->*/    }
/*\onslide<3->*/}/*\onslide<1->*/
\end{minted}
\end{frame}

\begin{frame}[fragile]{Put Put Put}
\begin{enumerate}[<+(1)->]
    \setcounter{enumi}{3}
    \item \task{Die Methode \T{public void put(int value)} der Klasse \T{RingBuffer} soll den übergebenen Wert an der
    Position im Ring Buffer speichern, auf die \T{tail} zeigt und \T{tail} auf den Nachfolger verschoben werden (obige Abb. ii). Sollte der Buffer komplett gefüllt sein, soll wieder von vorne begonnen werden und alte Elemente
    überschrieben werden (obige Abb. iii). In diesem Fall soll auch \T{head} auf den Nachfolger verschoben, damit
    \T{head} immer auf das älteste Element im Buffer zeigt und so die FIFO Eigenschaft gewahrt bleibt.}
\end{enumerate}
\SetupLstHl
\begin{minted}[escapeinside={/*}{*/}]{java}
/*\onslide<3->*/public class RingBuffer {
/*\onslide<3->*/    |ihl|private Element head, tail; private boolean isEmpty;|ihl|
/*\onslide<3->*/    |ihl|RingBuffer(int capacity) { ... }|ihl|
/*\onslide<4->*/    public void put(int value) {
/*\onslide<5->*/        if (this.tail == this.head && !this.isEmpty)
/*\onslide<5->*/            this.head = this.head.getNextElement();
/*\onslide<6->*/        this.tail.setValue(value);
/*\onslide<7->*/        this.tail = this.tail.getNextElement();
/*\onslide<8->*/        this.isEmpty = false;
/*\onslide<4->*/    }
/*\onslide<3->*/}/*\onslide<1->*/
\end{minted}
\end{frame}

\begin{frame}[fragile]{Operation Removal}
\begin{enumerate}[<+(1)->]
    \setcounter{enumi}{4}
    \item \task{Die Methode \T{public int remove()} der Klasse \T{RingBuffer} soll den Wert an der Position zurückliefern, auf
    die \T{head} zeigt und \T{head} um eine Stelle zurückverschoben werden. Diese Methode simuliert also ein Abarbeiten
    der Elemente im Buffer. Sollte der Buffer leer sein, soll eine NoSuchElementException ausgelöst werden.}
\end{enumerate}
\SetupLstHl
\begin{minted}[escapeinside={/*}{*/}]{java}
/*\onslide<3->*/public class RingBuffer {
/*\onslide<3->*/    |ihl|private Element head, tail; private boolean isEmpty;|ihl|
/*\onslide<3->*/    |ihl|RingBuffer(int capacity) { ... }|ihl|
/*\onslide<4->*/    public int remove() {
/*\onslide<5->*/        if (this.isEmpty)
/*\onslide<5->*/            throw new NoSuchElementException();
/*\onslide<6->*/        int value = this.head.getValue();
/*\onslide<7->*/        this.head = this.head.getNextElement();
/*\onslide<8->*/        this.isEmpty = this.head == tail;
/*\onslide<9->*/        return value;
/*\onslide<4->*/    }
/*\onslide<3->*/}/*\onslide<1->*/
\end{minted}
\end{frame}
}
\iffull
\SetNextSectionText{Dynamische Datenstrukturen II\\Abgabe: \DTMDate{2022-07-18}}
\section{Aussicht: Übungsblatt 11}
\subsection{Aufgabe 1}
\begin{frame}{Aufgabe 1: Graphen}
\begin{center}
\begin{minipage}{.3\linewidth}% igrate to columns?
\centering\onslide<2->{\begin{tikzpicture}[kreis/.style={inner sep=4pt,circle,draw,font=\sffamily,outer sep=1pt}]
\node[kreis] (a) at (0,0) {1};
\node[kreis] (b) at (2,0) {2};
\node[kreis] (c) at (2,-2) {3};
\node[kreis] (d) at (0,-2) {4};
\draw[-Kite] (a) edge (b) (b) edge (c) (c) edge (a) (b) edge (d) (d) edge (b) (c) edge (d);
\end{tikzpicture}}
\end{minipage}\quad\onslide<3->{\begin{minipage}{.25\linewidth}%
    \raisebox{2pt}{\(\bordermatrix{~ & \textcolor{gray}{1} & \textcolor{gray}{2} & \textcolor{gray}{3} & \textcolor{gray}{4} \cr
    \textcolor{gray}{1} & 0 & 1 & 0 & 0\cr
    \textcolor{gray}{2} & 0 & 0 & 1 & 1\cr
    \textcolor{gray}{3} & 1 & 0 & 0 & 1\cr
    \textcolor{gray}{4} & 0 & 1 & 0 & 0\cr
    }\)}
\end{minipage}}\quad\onslide<4->{\begin{minipage}{.25\linewidth}
    \begin{enumerate}
        \item \(\to 2\)
        \item \(\to 3 \to 4\)
        \item \(\to 1 \to 4\)
        \item \(\to 2\)
    \end{enumerate}
\end{minipage}}\end{center}\bigskip\par
\begin{itemize}
    \itemsep6pt
    \item<5-> Je nach Kontext bietet sich die Adjazenzmatrix (links) oder die Adjazenzliste (rechts) mehr an
    \item<6-> Für die Durchläufe markiert \(S\) den Startknoten
    \item<7-> Auch hier hilft die \hyperlink{episode-trav-jmp}{\rawlink{Episode}}
\end{itemize}
\end{frame}

\begin{frame}{Aufgaben 2 und 3}
    \begin{itemize}[<+(1)->]
        \itemsep12pt
        \item Die Implementation eines Stacks sollte nach Schema \(F\) funktionieren
        \item Bei der Baum-Traversierung nicht abschrecken lassen \begin{itemize}
            \item Bei Problemen erst einen kleinen Baum mit nur einer Operation probieren
            \item Gerne auch Unterstützung für weitere Funktionen!
        \end{itemize}
    \end{itemize}
\end{frame}
\fi

% \SetNextSectionText[.55\linewidth]{TODO}
\section{Abschließendes}

{\SummaryFrame
\begin{frame}[fragile,t]{Zusammenfassend}
% \printBibCommand
\lstfs{9}
\vfill\vfill % double fill for more fraction
\begin{itemize}[<+(1)->]
    \itemsep8pt
    \item Meldet euch für die Übungsleistung an \begin{itemize}
        \item Dies geht in \link{https://campusonline.uni-ulm.de/qislsf/}{campusonline}
        \item Die Prüfung heißt \say{10850 Praktische Informatik -  Übung}
    \end{itemize}
    \item Meldet euch auch für die Prüfung an
    \item Feedback bezüglich des Tutoriums wird immer gern gesehen \begin{itemize}
        \item Ist hier eine anonyme Umfrage erwünscht?
    \end{itemize}
    \item Die Betrachtung von Algorithmen und Datenstrukturen ist elementar!
\end{itemize}
\end{frame}
}


\outro{\vskip9mm\centering \onslide<2->{\scalebox{1.75}{
\begin{tikzpicture}
    \doggo[collar=purple,height delta=-.5mm]
\end{tikzpicture}~~%
\reflectbox{\begin{tikzpicture}
    \doggo[height delta=.1mm,body=black!50!gray!85!doggo@brown]
\end{tikzpicture}}
}}}

\iffull\end{document}\fi
