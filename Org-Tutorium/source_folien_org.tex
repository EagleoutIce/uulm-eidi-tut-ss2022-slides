\InputIfFileExists{../data/global.src}\relax\relax
\InputIfFileExists{../data/overview.tex}\relax\relax

\iffull
\title[Allgemeine Informationen]{Ja sind sie denn schon wieder da?}
\subtitle{Organisatorisches Tutorium}
\date{KW 17}
\addbibresource{references.bib}
\fi
% \SetTutoriumNumber{Org}
\setfootmarker{EidI~\faAngleRight\ Org. Tutorium}
\titlenumber{Org}
\setauthormailsubject{Organisatorisches\%20Tutorium}

% \documentclass{article}
% \usepackage{fetamont}
% \begin{document}
% \ffmwfamily\fontseries{eb}\selectfont Gauß
% \end{document}

\iffull
\begin{document}
\begin{frame}[plain,c]
\null\hfill\begin{minipage}{.8\linewidth}
    \rule{0cm}{1.25cm}\llap{\color{gray}Theorie\quad}\only<3->{\LoadOverview{1}{}}\hfill
    %\only<12->{\LoadOverview{1}{-active}}
    \onslide<11->{\LoadOverview{8}{}}\medskip\\
    \llap{\color{gray}Grundlagen\quad}\onslide<4->{\LoadOverview{2}{}}\onslide<5->{\LoadOverview{3}{}}\onslide<6->{\LoadOverview{4}{}}\medskip\\
    \llap{\color{gray}Fortgeschritten\quad}\hfill\onslide<7->{\LoadOverview{5}{}}\hfill\onslide<8->{\LoadOverview{6}{}}\onslide<9->{\LoadOverview{7}{}}\hfill\onslide<10->{\LoadOverview{9}{}}
\end{minipage}\par
\end{frame}

\titleframe
\fi

\SetNextSectionText{Schön ist das, was ohne Begriff allgemein gefällt.\\--- Immanuel Kant~\cite[p.~61]{kant1924kritik}}
\section{Allgemeines}


\savebox\pinguA{\tikz\pingu[wings shock,eyes shock,heart=gray!25!white];}
\savebox\pinguB{\tikz\pingu[wings wave,eyes wink,mask,heart=gray!25!white,halo];}
\subsection{Hygiene}
\begin{frame}[c]{Hygienekonzept}
\begin{itemize}[<+(1)->]
    \itemsep=\fill
    \item Bisher keine besonderen Regelungen.
    \item Dennoch gilt die Empfehlung:
\end{itemize}
\vfill
\begin{center}
    \scalebox{.8}{\begin{tikzpicture}
        \contourlength{2pt}
        \onslide<4->{\node[fill=btdm@border@down,minimum size=4cm,rounded corners=5pt] (wrongy) at (0,0) {\copy\pinguA};
        \node[gray,scale=1.35] at(wrongy.south east) {\huge\contour{btdm@background}{\faTimes}};}
        \onslide<5->{\node[fill=btdm@border@down,minimum size=4cm,rounded corners=5pt,right=1.5cm] (righty) at (wrongy.east) {\raisebox{1.4pt}{\copy\pinguB}};
        \node[gray,xshift=-1mm,yshift=1mm,scale=1.35,btdm@primary] at(righty.south east) {\huge\contour{btdm@background}{\faCheck}};}
    \end{tikzpicture}}
\end{center}
\vspace*{-.75\baselineskip}
\end{frame}

\begin{frame}[t]{Folien und Tutorien}
    \begin{itemize}[<+(1)->]
        \itemsep=18pt
        \item Folien von Florian Sihler (\link{mailto:florian.sihler@uni-ulm.de}{florian.sihler@uni-ulm.de})
        \item Mitschriften, Zusammenfassungen und vieles mehr im Cloudstore: \begin{itemize}[<1->]
            \item \url{https://cloudstore.uni-ulm.de/apps/circles/}
            \item Kreis: \say{Mitschriebe Informatik} (ganz eingeben)
        \end{itemize}
        \item Bei erbrachter Leistung: 3 ECTS (LP)
    \end{itemize}
\end{frame}


\savebox\pinguA{\tikz\pingu[left wing hug,eyes wink];}
\savebox\pinguB{\tikz\pingu[right wing hug,eyes shiny];}
\subsection{Übungsblätter}
\begin{frame}[t]{Übungsblätter\hfill I}
\begin{itemize}[<+(1)->]
    % \itemsep=\fill % yielded wrong results after closing of itemize
    \item Mindestens \qty{50}\percent\ der Punkte pro Blatt\begin{itemize}
        \itemsep=2.5pt
        \item Maximal 2 der 13 regulären Blätter dürfen nicht bestanden werden
        \item Aktuell keine Zusatzaufgaben geplant\vfill
    \end{itemize}
    \item Tutoriumsteilnahme verpflichtend für Punkte\vfill
    \item Abgabe in Zweiergruppen (Montags, 12:00\;Uhr) \only<handout:0|5>{\smash{\raisebox{-2pt}{\resizebox*!{\baselineskip}{\usebox\pinguA\!\usebox\pinguB}}}}\only<6->{\smash{\raisebox{-2pt}{\resizebox*!{\baselineskip}{\usebox\pinguA\hskip8em\usebox\pinguB}}}}\par
    \infoblock{Nur für die, die die Übungsleistung brauchen (Abgabegruppen!). Die Anderen schicken einfach eine Mail an mich.}\vfill
    \item<7-> Erlaubte Formate: {\begin{itemize}
        \itemsep=2.5pt
        \item<8-> Bei einer Datei: PDF, Plaintext (\T{.txt}), Java-Quellcode
        \item<9-> Bei mehreren: Archiv (ZIP,~\ldots) verwenden
        \item<10-> Wir möchten keine \T{.class}-Dateien, nur \T{.java} und dergleichen.
    \end{itemize}}\vfill
\end{itemize}
\end{frame}

\begin{frame}[t]{Übungsblätter\hfill II}
    \begin{itemize}[<+(1)->]
        \itemsep10.5pt
        \item Programme \emph{müssen} lauffähig sein (kompilierbar und mit \bjava{main}-Methode)\pause
            \infoblock{Oder entsprechend kommentiert, wo es (wieso) Probleme gibt. Sonst: 0 Punkte auf die Aufgabe.}
        \item Wir fordern Antwortsätze für Text und Rechenaufgaben.
        \item \emph{Jeder} muss die Abgabe vorstellen können.\pause\infoblock{Unabhängig davon, wer die Aufgabe im Team gemacht hat. Sonst: 0 Punkte auf die Aufgabe.}
        \item Im Falle einer Krankheit:\pause~Nachricht an Tutor\gend in mit zugehöriger Entschuldigung!
    \end{itemize}
    \vfill
    \begin{center}
        \pause\bfseries Lest die \link{https://www.uni-ulm.de/einrichtungen/zuv/dez1/recht-und-organisation/satzungen-und-ordnungen/studium-promotion-habilitation/studien-und-pruefungsordungen-bachelor-master-staatsexamen/}{\textbf{FSPO}}!\\
        \pause\bfseries Verwendet die Uni-Mail!
    \end{center}
\end{frame}

\iffull
\begin{frame}[t]{Präsenzaufgaben}
    \begin{itemize}[<+(1)->]
        \itemsep10pt
        \item Stift und Papier
        \item Jede Woche, unbepunktet
        \item Darf aber gerne abgegeben werden!
    \end{itemize}
\end{frame}
\fi

\begin{frame}{Klausur}
    \begin{itemize}
        \itemsep16pt
        \item Alle Informationen werden rechtzeitig bekanntgegeben
        \item Früh genug anmelden!
        \item Termine: \begin{itemize}
            \item Erstklausur am 10. August 2022
            \item Zweitklausur am 7. Oktober 2022
        \end{itemize}
        \item Es wird eine Probeklausur in der Vorlesung geben.\pause\infoblock{Der Termin dafür steht noch aus.}
    \end{itemize}
\end{frame}

\iffull
% guard against problems with wget clearing the file even if there is no connection
\immediate\write18{wget https://media.githubusercontent.com/media/EagleoutIce/Episode-Heaps/gh-pages/preview-01.png -O logoHeaps-\jobname.png}
\immediate\write18{wget https://media.githubusercontent.com/media/EagleoutIce/Episode-Recursion/gh-pages/preview-01.png -O logoRecursion-\jobname.png}
\immediate\write18{wget https://media.githubusercontent.com/media/EagleoutIce/Episode-Traversierung/gh-pages/preview-01.png -O logoTraversal-\jobname.png}
% and more :D

{%
\AddonFrame
\subsection{Zusätzliches}
\begin{frame}[t]{Zusätzliches}
    \begin{itemize}[<+(1)->]
        \itemsep9pt
        \item Es \textsl{empfiehlt} sich die Abgaben in \LaTeX\ ($2_{\textstyle\varepsilon}$) zu machen
        \item Dafür wurden erklärende Bonusblöcke erstellt. Darunter:\pause{}
        \vspace*{-.75\baselineskip}\begin{multicols}{3}
            \begin{enumerate}[<1->]
                \item \LaTeX{} (2-3 Tutorien)
                \item Richtig Kommentieren
                \item Linux (\& Kommandozeile)
                \item Git --- Versionsverwaltung
                \item (Unit-)Testen
                \item \ldots\par
            \end{enumerate}
        \end{multicols}\vspace*{-.75\baselineskip}
        \item Mit dem letzten Semester gibt es auch tolle Episoden. Darunter:\\\smallskip
% hella hail the copy and paste
\resizebox{.85\linewidth}!{\begin{tikzpicture}[align-base]
    \onslide<2->{\draw[thick,darkgray,rounded corners=2.5pt,path picture={\node at(path picture bounding box.center) {\href{https://media.githubusercontent.com/media/EagleoutIce/Episode-Traversierung/gh-pages/noanim_traversal.pdf}{\includegraphics[width=8.5cm,height=4.788cm,keepaspectratio]{logoTraversal-\jobname.png}}};}] (0,0) rectangle (8.5cm,4.788cm);}
\end{tikzpicture}\quad\begin{tikzpicture}[align-base]
    \onslide<2->{\draw[thick,darkgray,rounded corners=2.5pt,path picture={\node at(path picture bounding box.center) {\href{https://media.githubusercontent.com/media/EagleoutIce/Episode-Recursion/gh-pages/noanim_rekursion.pdf}{\includegraphics[width=8.5cm,height=4.788cm,keepaspectratio]{logoRecursion-\jobname.png}}};}] (0,0) rectangle (8.5cm,4.788cm);}
\end{tikzpicture}\quad\begin{tikzpicture}[align-base]
    \onslide<2->{\draw[thick,darkgray,rounded corners=2.5pt,path picture={\node at(path picture bounding box.center) {\href{https://media.githubusercontent.com/media/EagleoutIce/Episode-Heaps/gh-pages/noanim_heap.pdf}{\includegraphics[width=8.5cm,height=4.788cm,keepaspectratio]{logoHeaps-\jobname.png}}};}] (0,0) rectangle (8.5cm,4.788cm);}
\end{tikzpicture}\qquad\textbullet~\textbullet~\textbullet}
        \item Die (Lösungs-)Folien werden wochenweise veröffentlicht.
    \end{itemize}
\end{frame}
}
\fi

\SetNextSectionText[.7\linewidth]{As long as there were no machines, programming was no problem at all; when we had a few weak computers, programming became a mild problem, and now we have gigantic computers, programming had become an equally gigantic problem. In this sense the electronic industry has not solved a single problem, it has only created them.\\--- Edsger W. Dijkstra~\cite[p.~861]{dijkstra1972humble}}
\section{Java \& Editoren}
\def\os#1{\;\textsuperscript{\tiny\color{btdm@text}\def\x##1{\textcolor{gray!80!lightgray}{##1}}#1}}
\subsection{Was es für Editoren gibt}
\begin{frame}[t]{How to Editor}
    \begin{itemize}[<+(1)->]
        \itemsep16pt
        \item Text-Editor mit Syntaxfreude
        \item Keine IDE!
        \item Klassisch: \begin{itemize}
            \item \link{https://notepad-plus-plus.org/}{Nodepad++}\os{\faWindows, \x{\faApple}, \x{\faLinux}}
            \item \link{https://code.visualstudio.com/Download}{VS-Code}\os{\faLinux, \faApple, \faWindows} (ohne Extensions)
            \item \link{https://atom.io/}{\textsc{atom}}\os{\faLinux, \faApple, \faWindows}
        \end{itemize}
        \item Advanced: \begin{itemize}
            \item \link{https://www.nano-editor.org/}{nano}\os{\faLinux, \faApple, \x{\faWindows}}
            \item \link{https://www.vim.org/}{vim}\os{\faLinux, \faApple, \x{\faWindows}}
        \end{itemize}
    \end{itemize}
    \pause \begin{tikzpicture}[overlay, remember picture]
        \node[above left=1mm,yshift=\btdmfootheight,gray] at(current page.south east) {{\footnotesize\sbseries Und wenn eine IDE, dann\ldots}\qquad\raisebox\depth{\fancyqr[color=gray,height=2cm,level=H,image=\large\faYoutubePlay]{https://www.youtube.com/watch?v=X34ZmkeZDos}}};
    \end{tikzpicture}
\end{frame}

\subsection{Wie Java geht}
\begin{frame}[t]{How to Java}
    \begin{itemize}[<+(1)->]
        \itemsep7.5pt
        \item Linux: \begin{itemize}
            \item Wenn \T{apt}-basiert: \bbash{sudo apt install default-jdk} (\T{openjdk-14-jdk},~\ldots)
            \item Auch sonst über den jeweiligen Paketmanager
            \item Oder per \link{https://www.oracle.com/java/technologies/javase-downloads.html}{download}.
        \end{itemize}
        \item Windows: \begin{itemize}
            \item Download \& Installation der \link{https://www.oracle.com/java/technologies/javase-downloads.html}{gewünschten Version}.
            \item \T{PATH}-Variable anpassen! Dafür gibt es etliche \link{https://explainjava.com/set-java-path-and-java-home-windows/}{Anleitungen}:
                \begin{enumerate}
                    \item \say{Systemumgebungsvariablen bearbeiten} bzw. \say{Edit the system environment variables}
                    \item \say{bin}-Ordner der JDK zum \T{PATH} hinzufügen. \info{Zum Beispiel: \say{C:\textbackslash Program Files\textbackslash Java\textbackslash jdk-15\textbackslash bin}}
                \end{enumerate}
        \end{itemize}
        \item MacOS: \begin{itemize}
            \item Download \& Installation der \link{https://www.oracle.com/java/technologies/javase-downloads.html}{gewünschten Version}.
            \item Sonst unterstützt auch \link{https://brew.sh/}{Homebrew} ein paar Versionen.
        \end{itemize}
        % \item Es existieren \link{https://jdk.java.net/16/}{open source} Implementationen.
    \end{itemize}
\end{frame}

\SetNextSectionText{Gefühl von Grenze darf nicht heißen: hier bist du zu Ende, sondern: hier hast du noch zu wachsen.\\--- Emil Gött~\cite{gott1984zettelspruche}}
\section{Abschließendes}
{\SummaryFrame
\begin{frame}[t]{Zusammenfassend}
\pause \printBibCommand
\vfill\vfill % double fill for more fraction
\begin{itemize}[<+(1)->]
    \itemsep16pt
    \item Wöchentliche Übungsblätter in Zweiergruppen \begin{itemize}
        \item {\qty{50}\percent} der Punkte und Anwesenheit im Tutorium um ein Blatt zu Bestehen.
        \item Mindestens~11 der 13~Blätter müssen bestanden werden (exklusive Blatt~0).
        \item Beide müssen die Abgabe vorstellen können.
        \item Programme müssen lauffähig sein.
    \end{itemize}
    \item Bei Fragen darf sich jederzeit \emph{gerne} gemeldet werden
\end{itemize}
\end{frame}
}

\outro{\begin{tikzpicture}% remember picture does not work :C
\IfBtdmDarkmode{}{\pgfinterruptboundingbox
\only<3->{\fill[btdm@primary] (0,0) circle [radius=2*\paperwidth];}
\endpgfinterruptboundingbox}
\end{tikzpicture}%
\null\vfill
% use fill and stuff to get it done
\centering
\only<2|handout:0>{\begin{tikzpicture}[scale=1.5]
    \pingu[wings wave,name=saphira,eyes wink,pants=cprimary,monocle right]
    \pgfinterruptboundingbox
    \onslide<2->{\path[postaction={decorate},decoration={text along path, text={|\huge|Motivation!},text align={fit to path}}] (saphira-bill) ++ (90+55:45pt) arc (90+55:90-55:45pt);}
    \endpgfinterruptboundingbox
\end{tikzpicture}}%
\only<3->{%
\IfBtdmDarkmode{}{\BtdmMakeNextFullBottom}%inverts bottom
\begin{tikzpicture}[scale=1.5]
    \pingu[wings raise,monocle right,name=saphira,eyes wink,devil horns,left eye devil,right eye angry,bill=angry,glow thick=white,heart=pingu@purple!60!black]
    \pgfinterruptboundingbox
    \path[postaction={decorate},decoration={text along path, text={|\huge\bfseries\IfBtdmDarkmode{\color{btdm@primary}}{\color{white}}|Motivation!},text align={fit to path}}] (saphira-bill) ++ (90+55:45pt) arc (90+55:90-55:45pt);
    \endpgfinterruptboundingbox
\end{tikzpicture}%
}%
}
% {\setbeamercolor{background canvas}{bg=cprimary!15!black}
% \begin{frame}[plain,c]
% \null\vfill
% \end{frame}}
\iffull\end{document}\fi