\InputIfFileExists{../data/global.src}\relax\relax

\iffull
\title{Crossover Zahnbehandlungen}
\subtitle{Wurzelziehen Drei}
\date{KW 20}
\addbibresource{references.bib}
\fi
\SetTutoriumNumber{3}

\iffull\begin{document}
\titleframe

\TopicOverview{3}
\fi

\iffull{\SummaryFrame
\def\k{\textcolor{gray}{\(\rightarrow\)}}
\def\K{\textcolor{gray}{\(\longrightarrow\)}}
\def\a{\textcolor{codeouthl}{a}}
\def\b{\textcolor{codeouthl}{b}}
\setbeamerfont{description item}{shape=\itshape,series=\mdseries,size=\footnotesize}
\setbeamerfont{description body}{size=\footnotesize}
\setbeamerfont{itemize/enumerate subbody}{size=\footnotesize}
\begin{frame}[fragile,c]{Kurzwiederholung}
    \begin{itemize}[<+(1)->]
        \itemsep9.25pt
        \item Algorithmen können wir auf diverse Eigenschaften untersuchen: \begin{description}[Totale Korrektheit:]
            \item[Determiniertheit:] die gleiche Eingabe erzeugt immer die gleiche Ausgabe\hfill\tikz[align-half-base,every path/.append style={line cap=round},gray]{\node[circle,scale=.4,fill] (a) at (0,0) {}; \node[circle,scale=.4,fill] (b) at (.7,0) {}; \pgfonlayer{background}\draw[codeouthl](a) to[bend left=30.5] (b); \draw[codeouthl](a) to[bend right=30.5] (b); \draw[codeouthl](a) to[bend left=10] (b); \draw[codeouthl](a) to[bend right=10] (b); \endpgfonlayer\node[scale=.85,circle,fill=white,opacity=.3,text opacity=1,inner sep=0pt] at(current bounding box.center) {?};}
            \item[Determinismus:] bei gleicher Eingabe folgt stets die gleiche Schrittfolge \info{\textcolor{gray}{~\(\Rightarrow\)~Determiniert}}\hfill\tikz[align-half-base,every path/.append style={line cap=round},gray]{\node[circle,scale=.4,fill] (a) at (0,0) {}; \node[circle,scale=.4,fill] (b) at (.7,0) {}; \pgfonlayer{background}\draw[codeouthl](a) to[bend left=30.5] (b); \draw[codeouthl](a) to[bend right=30.5] (b); \draw[gray](a) to[bend left=10] (b); \draw[codeouthl](a) to[bend right=10] (b);\endpgfonlayer}
            \item[Totale Korrektheit:] Terminiert und partiell korrekt \info{\(\text{Terminiert} \implies \text{gewünschte Ausgabe}\)}
        \end{description}
        \item Manche von Javas primitiven Datentypen können implizit konvertiert werden \begin{itemize}
            \item \bjava{byte}~\k~\bjava{short}~\k~\bjava{int}~\k~\bjava{long}~\k~\bjava{double}~~und~~\bjava{char}~\k~\bjava{int}
        \end{itemize}
        \item Javas Präzedenzregeln geben an, wie Operatoren implizit geklammert werden \begin{itemize}
            \item \textcolor{codeouthl}{(stark)}~~~\T{\a++}~\K~\T{!\a}, \T{++\a}~\K~\T{\a\,/\,\b}, \T{\a\,*\,\b}~\K~\T{\a\,+\,\b}, \T{\a\,-\,\b}~\K~\T{\a\,==\,\b}~\K~\T{\a\,\&\&\,\b}~\K~\T{\a\,|\!|\,\b}~~~\textcolor{codeouthl}{(schwach)}%\smallskip
            \item Bei Ganzzahldivisionen werden Nachkommastelle abgeschnitten \info{\bjava{5/2} ergibt \bjava{2}}
            \item Bei unterschiedlichen Typen (\bjava{5.0/2}) wird der \say{kleinere} implizit \say{angehoben} (\bjava{5.0/2.0})
            \item \bjava{byte} oder \bjava{short} werden dabei immer mindestens zu \bjava{int} \info{\say{promoted}/widening, \link{https://docs.oracle.com/javase/specs/jls/se17/html/jls-5.html\#jls-5.6}{JLS17~5.6}}
        \end{itemize}
        % für Ganzzahlen (\bjava{byte}, \bjava{short}, \bjava{int}, \bjava{long}), Fließkommazahlen (\bjava{float}, \bjava{double}), Zeichen (\bjava{char}) und Wahrheitswerte (\bjava{boolean})
%        \begin{itemize}
% TODO:            \item Variablen dekl, init, zws
%        \end{itemize}
        \item Wenn verschachtelte Schleifen voneinander abhängen, hilft Gauß \info{\raisebox{1.5pt}{\(\scriptstyle\sum_{i = \text{1}}^N i\;{}={} \frac{N \cdot (N + \text{1})}{\text{2}}\)}}% TODO: Schleifen die voneinander abhängen
% \item TODO: Kommandozeilenparameter
    \end{itemize}
    \onslide<1->
\end{frame}
}\fi

\SetNextSectionText[.55\linewidth]{My volition shrinks from the painful task of recalling my humiliation; yet, like a second Prometheus, I will endure this and worse, if by any means I may arouse in the interiors of Plane and Solid Humanity a spirit of rebellion against the Conceit which would limit our Dimensions to Two or Three or any number short of Infinity.\\---~Edwin A. Abbott~\cite{abbott1987flatland}}
\section{Präsenzaufgabe}
\subsection{The simple solution}
\begin{frame}[fragile,c]{Präsenzaufgabe}
\begin{aufgabe}{Von doppelten Methoden}
    \small\onslide<2->{Mehrdimensionale Arrays, also Arrays, deren Elemente selbst wiederum Arrays sein können, können wir als Matrizen interpretieren, beispielsweise ist \bjava{array[1][3]} das Element der zweiten Zeile in der vierten Spalte.} \onslide<3->{Legen Sie eine Java Datei namens \T{PositiveEintraege.java} an und implementieren Sie folgende Teilaufgaben innerhalb dieser Datei \info{oder bearbeiten Sie die Aufgabe auf einem Blatt Papier}.}
\begin{enumerate}
    \itemsep-2pt
    \item<4-> Initialisieren Sie ein zweidimensionales Array mit \(3 \times 3\) Elementen vom Typ \bjava{double} mit gültigen Werten in der \bjava{main}-Methode.
    \item<5-> Implementieren Sie eine Methode \bjava{public static int anzahlPositive}\allowbreak\bjava{(double[][] matrix)}, die die Anzahl der positiven Eintrage (\(> 0\)) bestimmt und zurückgibt. Für ungültige Matrizen soll die Methode \bjava{-1.0} zurückgeben.
    \item<6->[\color{gray}3.] \color{gray}Testen Sie die Implementierung aus b) mit ihrem Array aus a).
\end{enumerate}
\onslide<1->
\end{aufgabe}
\end{frame}

\begin{frame}[c,fragile]{An Array a day}
\begin{itemize}
    \item<2-> Wir basteln uns eine Klasse und Initialisieren froh umher.
\end{itemize}
\DoAnimations
\begin{plainjava}
$3->$public class PositiveEintraege {
$4->$    public static void main(String[] args) {
$5->$        double[][] mat = {
$6->$              { -1.0, 21.0, 3.0 }, $7->\text{\tikzmarknode{r1}{\T{\HStrut}}}$
$6->$              $6->\text{\tikzmarknode{@1}{\sbasic{\HStrut\{ \snum{-1.0}, \snum{42.0}, \snum{3.0} \}}}}$, $8->\text{\tikzmarknode{r2}{\T{\HStrut}}}$
$6->$              { -1.0, 84.0, 3.0 }  $9->\text{\tikzmarknode{r3}{\T{\HStrut}}}$
$5->$       };
$4->$     }
$3->$}
\end{plainjava}
\begin{tikzpicture}[overlay,remember picture]
    \onslide<7->{
        \node[T,right,yshift=1pt] at(r1.east) {\T{mat[0]}~\(\to\) \T{\{-1.0, 21.0, 3.0\}}};
    }
    \onslide<8->{
        \hlcode[.5pt]{@1}{@K1}
        \node at(@1) {\sbasic{\HStrut\{ \snum{-1.0}, \snum{42.0}, \snum{3.0} \}}};
        \node[T,right,outer xsep=0pt,inner sep=0pt] (@r) at(r2.east) {\sbasic{\normalsize\HStrut}\T{mat[1]}};
        \node[T,right,outer xsep=0pt,inner sep=0pt] at (@r.east) {\sbasic{\normalsize\HStrut}\T{[2]}\(\to\) \T{3.0}};
        \pgfonlayer{background}
            \onslide<8->{\hlcode[.5pt]{@r}{@K2}}
        \endpgfonlayer
    }
\end{tikzpicture}
\end{frame}

{\MakeThePinguExplainIt[text width=5.75cm,yshift=-5mm]{cap=!hide,cup=!hide,construction helmet=btdm@primary!60!btdm@background}{Was heißt \say{ungültig?} Das werden wir voerst zurückstellen!}
\begin{frame}[c,fragile]{Hey guy's, i did some methods}
\task<2->{Implementieren Sie eine Methode \T{anzahlPositive(double[][])}, die die Anzahl der positiven Eintrage (\(> 0\)) bestimmt und zurückgibt. Für \only<3->{\expandafter\textbf}{ungültige} Matrizen soll die Methode \T{-1.0} zurückgeben.}
\DoAnimations\SetupLstHl
\begin{plainjava}
$5->$public class PositiveEintraege {
$5->$    |ihl|public static void main(String[] args) { $6->$:ldots:$5->$ }|ihl|
$7->$    public static int anzahlPositive(double[][] matrix) {
$8->$        int anzahl = 0;
$9->$        for(int row = 0; row < matrix.length; row++) {
$10->$            for(int col = 0; col < matrix[row].length; col++) {
$11->$                if(matrix[row][col] > 0)  anzahl = anzahl + 1;
$10->$            }
$9->$        }
$12->$        return anzahl;
$7->$   }
$5->$}$1->$
\end{plainjava}
\begin{tikzpicture}[overlay,remember picture]
    \onslide<4->{\node[left=-4mm,xshift=4mm,scale=.8,yshift=\btdmfootheight] at(current page.-29) {\copy\pinguexplainbox};}% copy for animations
\end{tikzpicture}
\end{frame}
}

\begin{frame}{Complex Guardians}
\begin{itemize}[<+(1)->]
    \itemsep11pt
    \item Arrays sind in Java \emph{komplexe} Datentypen \info{vs. Javas primitiver Datentypen}
    % \item Solche sind Referenzdatentypen \info{dazu später mehr!}
    \item Neben einem \say{gültigen} Wert, können diese auch den Wert \bjava{null} haben \begin{itemize}
        \item Mit \bjava{double[][] matrix = null;} liefert \bjava{matrix[1]} eine \bjava{NullPointerException}
        \item Prüfen können wir dies beispielsweise mit \bjava{matrix == null}
    \end{itemize}
    \item Allgemein sollte nie \bjava{null} übergeben oder zurückgegeben werden
    \item Neben \bjava{null}, könnte man auch noch weitere Dinge abprüfen \begin{itemize}
        \item Gibt es überhaupt eine Zeile \bjava{matrix.length == 0}?
        \item Gibt es überhaupt eine Spalte \bjava{matrix[row].length == 0}?
        \item Und viele mehr\ldots\ In jedem Fall sollte man diese mit einem Kommentar absichern
    \end{itemize}
    \item Mit komplexen Datentypen werden wir später noch viel Freude haben!
\end{itemize}
\end{frame}

\begin{frame}[c,fragile]{Die Suche nach der positiven Anzahl}
\DoAnimations\lstfs{10}
\begin{plainjava}
$2->$public static int anzahlPositive(double[][] matrix) {
$4->$    if (matrix == null || matrix.length == 0) {
$5->$        System.out.println("Matrix ungültig!");
$6->$        return -1;
$4->$    }

$3->$    int anzahl = 0;
$3->$    for (int row = 0; row < matrix.length; row++) {
$3->$        for (int col = 0; col < matrix[row].length; col++) {
$3->$            if (matrix[row][col] > 0)
$3->$                anzahl = anzahl + 1;
$3->$        }
$3->$    }
$3->$    return anzahl;
$2->$}
\end{plainjava}
\begin{tikzpicture}[overlay,remember picture]
    \only<7->{\node[above left,yshift=\btdmfootheight] at (current page.south east) {\textattachfile{\curpath PositiveEintraege.java}{PositiveEintraege.java}\;};}
\end{tikzpicture}
\end{frame}

\subsection{The crazy overengineer}
\begin{frame}
    TODO: MEthoden
    TODO: for-each anstelle von for-loop
\end{frame}

\section{Abschließendes}
{\SummaryFrame
\begin{frame}[t]{Zusammenfassend}
\pause \printBibCommand
\vfill\vfill % double fill for more fraction
\begin{itemize}[<+(1)->]
    \itemsep11pt
    % \item TODO: get bib to work here
    \item
\end{itemize}
\end{frame}
}

\outro{\vskip6mm\centering\begin{tikzpicture}[scale=1.85]
\end{tikzpicture}}


\iffull\end{document}\fi
