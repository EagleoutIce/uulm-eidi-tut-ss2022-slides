\PassOptionsToClass{handout}{beamer}
\def\thpath{./data}
\RequirePackage{color-palettes}
\DefinePalette{Compact}
{Türkis,türkisfarben: RGB(21, 128, 112)}
{Türkis,türkisfarben: RGB(21, 128, 112)}
{Hellblau,hellbläulich: RGB(21, 110, 130)}
% {Blau,bläulich: RGB(21, 92, 148)}
{Grün,grünlich: RGB(21, 150, 90)}
\UsePalette{Compact}
\def\btdmopts{color=paletteA}%, footfade
\def\btdmmode{lightmode-fill}
\input{data/global.src}
\UsePalette{Compact}
% re-inforce the palette in case compact changes
\usepackage{fmtcount}

\colorlet{MaterialHeaderColor}{paletteB!75!btdm@background}%
\colorlet{NextMaterialHeaderColor}{paletteB!65!btdm@background}%
\attachfilesetup{color=paletteB}
\fullfalse
\def\maxtut{0}
\titlenumber{0--\maxtut}
\title[Alle Tutorien 0--\maxtut]{Kompaktversion Tutorien\\\small Von pinguinreduzierten Tutorien 0--\maxtut}
\date{\sffamily\today}


\let\oldinputif\InputIfFileExists

\makeatletter
\usepackage{pgfcalendar}
\edef\semesterstartyear{2022}
\def\startkwofsemester{16} % used to detect year breaks
\newcount\@kwconverter
\newcount\@firstweekday
\DTMsetdatestyle{german}
\def\getstartdayofweek#1#2{% #1 is kw, #2 is day offset
      % do we have a wraparoundi?
      \ifnum#1<\startkwofsemester \edef\@year{\the\numexpr\semesterstartyear+1}%
      \else\let\@year\semesterstartyear\fi
      \pgfcalendardatetojulian{\@year-1-1}{\@kwconverter}\pgfcalendarjuliantoweekday\@kwconverter\@firstweekday
            % we remove the first weekday as an offset
            \edef\@day{\the\numexpr#1*7-\number\@firstweekday+#2\relax}%
      \pgfcalendardatetojulian{\@year-1-1+\@day}{\@kwconverter}%
      \pgfcalendarjuliantodate{\@kwconverter}\@year\@month\@day
      \DTMdisplaydate\@year\@month\@day\m@ne\relax
}

\def\getKwRange#1{\getstartdayofweek{#1}0\,---\,\getstartdayofweek{#1}6}

% sub | title | path | file | num | KW
\newcommand\Load[5][]{%
\ifx!\detokenize{#1}!\def\@short{#2}\else\def\@short{#1}\def\beamer@shorttitle{#1}\fi
\SetNextSectionText{\getKwRange{#5}\\KW~#5}% 3 instead of two as this seems to pop more
\expandafter\section\expandafter[\@short]{#2}%
\bgroup\let\section\subsection
\let\subsection\subsubsection
% sub sub stays subsub?
% \let\subsubsection\paragraph
\gpreto\input@path{{#3/}}%
\def\curpath{#3/}% for rBash and rExecute
% we block global.src form being loaded
% we keep the first for input
\def\InputIfFileExists##1##2##3{\relax \global\let\InputIfFileExists\oldinputif}%
% forbid additional loading:
\renewcommand*\usepackage[2][]{\typeout{gobble: ##1|##2}}
\renewcommand*\RequirePackage[2][]{\typeout{gobble: ##1|##2}}
\def\printBibCommand{\let\vfill\relax}%
\let\titleframe\relax % no extra titlepage
% get the date:
\typeout{Loading: #3/source_folien_#4.tex}
\begin{refsection}[#3/references.bib]% allow for sub bibliographies
\@input{#3/source_folien_#4.tex}
\end{refsection}
\egroup
}

% update prepath
\setbeamercolor{section in toc}{fg=darkgray}

% we do not want footers
\BtdmNoSubsInFooter
\btdmsetfootlinethreshold{10}

\begin{document}
% \Titlepage{0--\maxtut}
\titleframe
\setcounter{tocdepth}{1}
{\def\ImpT#1{\textit{\color{btdm@primary}#1}}
\makeatletter\setfootmarker{\btdm@footbottomleft@title}% this way we get the titlemarker again
\begin{frame}[c]{}
\strut\vspace*{3em}\vfill\centering
% TODO: custom foot left bottom
      \bfseries\color{btdm@primary!70!white}
         Diese \ImpT{Kompaktversion} ist dazu gedacht die wichtigsten Kommentare und Lösungen zu sammeln.
         Sie ist allerdings \ImpT{ohne jede Garantie auf Vollständigkeit} aufzufassen. Zusätzliche Inhalte finden sich in den Folien zu den einzelnen Tutorien.\bigskip\par
         Liebe Grüße, Flo
\vfill\strut
\end{frame}
}
\setcounter{tocdepth}{5}
\def\sheet{\texorpdfstring{\\}{ --- }}
\Load[Org]{Organisatorisches Tutorium}{Org-Tutorium}{org}{17}
\Load[0]{Blatt 0\sheet Hello World}{0-Tutorium}{0}{17}
\Load[Tutorium 1]{Blatt 1\sheet Algorithmenentwurf und -analyse}{1-Tutorium}{1}{18}
% \Load[Tutorium 2]{Blatt 2\sheet Raten, Typen \& Booleans}{2-Tutorium}{2}{45}
% \Load[Tutorium 3]{Blatt 3\sheet Programmfluss \& Schleifen}{3-Tutorium}{3}{46}
% \Load[Tutorium 4]{Blatt 4\sheet Schleifen, Vokale \& Prim}{4-Tutorium}{4}{47}
% \Load[Tutorium 5]{Blatt 5\sheet Arrays, Matrizen \& Vektoren}{5-Tutorium}{5}{48}
% \Load[Tutorium 6]{Blatt 6\sheet Methoden \& Tic-Tac-Toe}{6-Tutorium}{6}{49}
% \Load[Tutorium 7]{Blatt 7\sheet Überladen \& Überschatten}{7-Tutorium}{7}{50}
% \Load[Tutorium 8]{Blatt 8\sheet Klassen \& Taxis}{8-Tutorium}{8}{2}
% \Load[Tutorium 9]{Blatt 9\sheet Rekursion}{9-Tutorium}{9}{3}
% \Load[Tutorium 10]{Blatt 10\sheet Sortierverfahren}{10-Tutorium}{10}{4}
% \Load[Tutorium 11]{Blatt 11\sheet Stacks und Buffer}{11-Tutorium}{11}{5}
% \Load[Tutorium 12]{Blatt 12\sheet Bäume und Minimax}{12-Tutorium}{12}{6}
% \Load[Tutorium 13]{Blatt 13\sheet Vererbung}{13-Tutorium}{13}{7}
\end{document}