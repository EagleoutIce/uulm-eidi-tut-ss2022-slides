\InputIfFileExists{../data/global.src}\relax\relax

\iffull
\title[Nulltes Tutorium -- Übungsblatt 0]{Mein Compiler und ich}
\subtitle{Tutorium ZeroHero}
\date{KW 17}
\addbibresource{references.bib}
\fi
\SetTutoriumNumber{0}

\pingudefaultsappend{eyes=shiny,wings=grab,glasses opacity=.97}

\iffull\begin{document}
\begin{frame}[plain,c]
\null\hfill\onslide<2->{\begin{minipage}{.8\linewidth}
    \rule{0cm}{1.25cm}\llap{\color{gray}Theorie\quad}\only<-3|handout:0>{\LoadOverview{1}{}}\only<4->{\LoadOverview{1}{-active}}
    \hfill
    \onslide<11->{\LoadOverview{8}{}}\medskip\\
    \llap{\color{gray}Grundlagen\quad}\LoadOverview{2}{}\LoadOverview{3}{}\LoadOverview{4}{}\medskip\\
    \llap{\color{gray}Fortgeschritten\quad}\hfill\LoadOverview{5}{}}\hfill\LoadOverview{6}{}\LoadOverview{7}{}\hfill\LoadOverview{9}{}
\end{minipage}}\par
\end{frame}

\titleframe
\fi


\SetNextSectionText{\say{My dear Watson, try a little analysis yourself,} said he, with a touch of impatience. \say{You know my methods. Apply them, and it will be instructive to compare results.}\\--- Conan Doyle~\cite[chp.~VI]{doyle2010sign}}
\section{Präsenzaufgabe}
{\def\f#1->#2\;{\parbox{7.5em}{\(S = \text{\T{#1}}\)\hfill}\(~~{\color{gray}\longrightarrow}~~\hbox to3.5em{\say{#2}\hfill}\)}
\setbeamertemplate{enumerate item}{\alph{enumi})}
\begin{frame}[c]{Präsenzaufgabe}
\begin{aufgabe}{Das habe ich schon gesehen!}
\vspace*{-.825\baselineskip}\null\hfill\smash{\raisebox{-\height+.15\baselineskip}{\only<4->{\tikz\node[draw=lightgray,rounded corners=2pt,fill=btdm@background,text=btdm@text,inner sep=5.5pt,very thick,outer sep=0pt]{\minipage{.6\linewidth}
\itemize
    \itemsep4pt
    \item<5-> \f ababababa->Nein\; \info{Enthält ein \T{b}}
    \item<6-> \f aaaaaaaaa->Ja\; \info{Hat \T{a} und kein \T{b}}
    \item<7-> \f cdefg->Nein\; \info{Enthält kein \T{a}}
    \item<8-> \f klmohna->Ja\; \info{Hat \T{a} und kein \T{b}}
\enditemize
\endminipage};}}\kern-2.75mm}\par{\only<-3|handout:0>{\parshape 1
0pt \linewidth}\only<4->{\parshape 7
0pt .365\linewidth
0pt .365\linewidth
0pt .365\linewidth
0pt .365\linewidth
0pt .365\linewidth
0pt .365\linewidth
0pt \linewidth}
% \raggedright
\onslide<2->{Entwerfen Sie einen Algorithmus um herauszufinden, ob in einer Zeichenkette \(S\) das Zeichen \T{a}, aber nicht das Zeichen \T{b} vorkommt. Als Ergebnis soll der Algorithmus ausgeben, ob die Bedingung zutrifft (\say{Ja}) oder nicht (\say{Nein}).} \onslide<3->{Ein paar Beispiele\ldots}%
\medskip\par}
    \only<9->{\relax Führen Sie die Schritte der Algorithmenentwicklung durch:\vspace*{-\topsep}
    \begin{multicols}{3}
        \begin{enumerate}
            \item<10-> Problemspezifikation
            \item<11-> Problemabstraktion
            \item<12-> Algorithmenentwurf
            \item<13-> Korrektheitsnachweis% bzw. Verifikation}
            \item<14-> Aufwandsanalyse\vspace*{-.35\topsep}
        \end{enumerate}
    \end{multicols}}
    \onslide<1->
\end{aufgabe}
\end{frame}
}

{
\def\g{\color{gray}\itshape}
\setbeamertemplate{enumerate item}{\g\alph{enumi})}
\begin{frame}{Problemspezifikation und -abstraktion}
    \onslide<+(1)->\begin{center}
        Es gelte \(i, j \in \{1, \ldots, n\} \subseteq \N\)
    \end{center}\vfill
    % \vskip0pt plus1filll\relax
    \begin{enumerate}[<+(1)->]
        \itemsep=16pt
        \item {\g Spezifikation: Definiere alle relevante Begriffe unmissverständlich}
        \begin{itemize}[<+(1)->]
            \itemsep1.5pt
            \item Eine \say{Zeichenkette} \(S\) ist eine indizierte Liste \((s_1, \ldots, s_n)\) an Zeichen.
            \item Ein \say{Zeichen} ist ein Symbol \(s_i\) aus einem Alphabet \(\Sigma\).
            \item Mit \say{Zeichen \(x\) kommt in \(S\) vor} meinen wir: \(\exists i:~s_i = x\)
            % \item Das \say{nicht-vorkommen} bezeichne hier die aussagenlogische Negation.
        \end{itemize}
        \item {\g Abstraktion: Definiere die Eingabe und die gewünschte Ausgabe}
        \begin{description}[Gegeben:]
            \itemsep1.5pt
            \item<+(1)->[Gegeben:] Eine Zeichenkette \(S = s_1, \ldots, s_n\)
            \item<+(1)->[Gesucht:] Ausgabe \say{Ja} genau dann, wenn \(\exists i: s_i = \text{\T{a}}\) und \(\forall j: s_j \neq \text{\T{b}}\), sonst \say{Nein}
        \end{description}
    \end{enumerate}
\end{frame}

\SetAlgoVlined
\SetKwIF{If}{ElseIf}{Else}{Wenn}{tue:\quad}{}{Sonst:}{}
\SetKwFor{For}{für}{tue:}{ende}
\SetKwInput{KwIn}{Eingabe}
\SetKw{KwTo}{bis einschließlich}
\SetKw{KwStep}{in Schritten von}
\SetKw{KwIs}{ist}
\SetKw{KwAnd}{und}
\SetKw{KwNot}{nicht}
\def\a{\text{\textit{enthältA}}}
\def\b{\text{\textit{enthältB}}}
\begin{frame}[c]{Algorithmenentwurf}
\setcounter{algocf}{0}
\pause\begin{algorithm}[H]
\PreCode
    \KwIn{Eingabe als Zeichenkette $S$ mit Zeichen $S=s_1, \ldots, s_n$}\medskip
\StartCode
    \onslide<3->{$\a \gets Nein$\;}
    \onslide<4->{$\b \gets Nein$\medskip\;}
    \onslide<5->{\For{i $\gets$ 1 \KwTo n \KwStep 1}{
        \onslide<6->{\lIf{\(s_i\) \textup{ist Buchstabe \T{a}}}{$\a\gets Ja$}}
        \onslide<7->{\lIf{\(s_i\) \textup{ist Buchstabe \T{b}}}{$\b\gets Ja$}}
    }}\medskip
    \onslide<8->{\lIf{\a\ \textup{ist} Ja \textup{aber} \b\ \textup{ist} Nein}{Gebe aus: \say{Ja}}}
    \onslide<9->{\lElse{Gebe aus: \say{Nein}}}\medskip
    \only<10->{\caption{Ein \say{Zeichenkettenfilter}}}
\end{algorithm}
\end{frame}

\SetKw{KwTo}{bis}
\SetKw{KwStep}{Schrittgröße}
\SetKwIF{If}{ElseIf}{Else}{Wenn}{\!:~}{}{sonst:}{}
\SetKwFor{For}{für}{\!:~}{ende}
\def\a{\text{\textit{hatA}}}
\def\b{\text{\textit{hatB}}}
\SetNlSty{textbf}{Z}{:}
\newsavebox\presalgobox
\setbox\presalgobox=\hbox{\color{gray}\minipage{6cm}\footnotesize\begin{algorithm}[H]
    $\a \gets Nein,\quad \b \gets Nein$\;
    \For{i $\gets$ 1 \KwTo n \KwStep 1}{
        \lIf{\(s_i = \text{\T{a}}\)}{$\a\gets Ja$}
        \lIf{\(s_i = \text{\T{b}}\)}{$\b\gets Ja$}
    }
    \lIf{\(\a = Ja\), \(\b = Nein\)}{\say{Ja}}
    \lElse{\say{Nein}}
\end{algorithm}\endminipage}

\def\a{\text{\textit{enthältA}}}
\def\b{\text{\textit{enthältB}}}
\begin{frame}{Korrektheitsnachweis, Verifikation}
\begin{enumerate}[<+(3)->]
    \setcounter{enumi}{3}
    \item<3-> {\g Verifikation: Zeige die totale Korrektheit (der Algorithmus\\  terminiert und ist partiell Korrekt):}\medskip
    \begin{description}[Partiell Korrekt:]
        \itemsep8pt
        \item<4->[Termination:] \(Z1\) und \(Z3\)--\(Z6\) sind Elementaroperationen.\\ Diese Terminieren per Definition.\medskip\par
        \onslide<5->{Die Schleife in \(Z2\) terminiert, da \(i\) streng monoton ansteigt und damit in endlicher Zeit \(i \leq n\) verletzt.}
        \item<6->[Partiell Korrekt:] Wir merken uns mit \a\ und \b, ob der jeweilige Buchstabe gefunden wurde (\(Z2\)--\(Z4\)). \onslide<7->{Wir geben \say{Ja} nur dann aus, wenn wir ein \T{a}, aber kein \T{b} finden (\(Z5\)). Sonst geben wir \say{Nein} aus.}
    \end{description}
\end{enumerate}
\tikzpicture[overlay,remember picture]
    \onslide<2->{\node[below left,xshift=2.5mm,yshift=-1.35cm,scale=.8,align=center,gray] at(current page.north east) {\copy\presalgobox\\\itshape\small Eine verkürzte Version};}
\endtikzpicture
\end{frame}
\iffull
\MakeThePinguExplainIt[text width=6.65cm,yshift=-1.75cm]{cap=!hide,halo,glasses=!hide,eyes shiny,heart=shadeA,body type=chubby,right item angle=-52}{Invarianten mögen vielleicht gruselig klingen, sind aber nur Aussagen, die zum Beispiel im Rahmen von Schleifen \say{immer} (also in jedem Durchlauf) gelten.}
\begin{frame}{Verifikation --- Eine kleine Vertiefung}
   \begin{itemize}[<+(1)->]
    \item Der Alternative, \say{weniger schwammige} Beweis der partiellen Korrektheit: \begin{itemize}
        \itemsep5pt
        \item Man überzeuge sich leicht, dass folgende Invariante für alle \(i \in \{1, \ldots, n\}\) gelte: \begin{equation*}
            \onslide<4->{\a = \begin{cases}
                Ja, & \exists j \in \{1, \ldots, i\}: s_j = \text{\T{a}}\\
                Nein,\quad & \text{sonst.}
            \end{cases}} \qquad \onslide<5->{\b = \begin{cases}
                Ja, & \exists j \in \{1, \ldots, i\}: s_j = \text{\T{b}}\\
                Nein,\quad & \text{sonst.}
            \end{cases}}\vspace*{-4mm}
        \end{equation*}
        \item<6-> Dies kann man beispielsweise mit vollständiger Induktion zeigen.
        \item<7-> Anschließend folgt der Beweis durch Abgleich der Bedingungen.
    \end{itemize}
   \end{itemize}
   \begin{tikzpicture}[overlay,remember picture]
    \onslide<8->{\node[above left,xshift=5mm,scale=.8,yshift=\btdmfootheight] at(current page.south east) {\copy\pinguexplainbox};}% copy for animations
    \end{tikzpicture}
\end{frame}
\fi


\begin{frame}{Aufwandsanalyse}
\begin{itemize}
    \itemsep14pt
    \item<3-> Mögliche ist eine tabellarische oder textuelle Erfassung.
    \item<4-> Wir machen es textuell: \begin{itemize}
        \itemsep7pt
        \item<5-> \(Z1\) enthält zwei Zuweisungen, die Bedingung in~\(Z5\) enthält zwei\\Vergleiche und in jedem Fall eine Ausgabe. %\\ (als Elementaroperation).
        \item<6-> Die Schleife in \(Z2\) enthält zwei Vergleiche und im schlechtesten Fall eine Zuweisung (so ist für \(\text{\T{a}} \neq \text{\T{b}}\) höchstens eine beider Bedingungen trifft zu).
        \item<7-> Die Schleife wird immer genau \(n\)-mal durchlaufen.
    \end{itemize}
    \item<8-> So erhalten wir im best- (keine \(a\)'s und \(b\)'s) und worst-case (nur \(a\)'s und \(b\)'s):\vspace*{-3.5mm} \begin{align*}
        \onslide<9->{\text{best-case}} &\onslide<9->{= 2 + n \cdot (2 + 0) + 2 + 1 = 2n + 5 \in \O(n)}\\
        \onslide<10->{\text{worst-case}} &\onslide<10->{= 2 + n \cdot (2 + 1) + 2 + 1 = 3n + 5 \in \O(n)}
    \end{align*}
\end{itemize}
\tikzpicture[overlay,remember picture]
    \onslide<2->{\node[below left,xshift=2.5mm,yshift=-1.35cm,scale=.8,align=center,gray] at(current page.north east) {\copy\presalgobox\\\itshape\small Eine verkürzte Version};}
\endtikzpicture
\end{frame}
}

\MakeThePinguExplainIt[text width=6.65cm,yshift=-1.75cm]{cap=!hide,hat,hat coronal=paletteA,hat ribbon=paletteA,glasses=!hide,eyes wink,cup=!hide,heart=shadeA,right item angle=-50}{Noch werden wir uns solchen Annahmen zärtlich nähern. Mit der Zeit wird es aber immer mehr auch darum gehen, \textit{was} man eigentlich annehmen darf.}
\begin{frame}[t]{Schweigsame Annahmen}
    \onslide<2->{\color{btdm@text}Für das Programm \say{Zeichenkettenfilter} und seine Analyse gelten die Annahmen:
    \begin{enumerate}[<+(1)->]
        \item<3-> die Länge \(n\) von \(S\) sei endlich und (in endlicher Zeit) berechenbar%, hier durch \textit{\(n \gets{}\) length(\(S\))},
        \item<4-> wir können die Zeichen \(s_i\) aus \(\Sigma\) vergleichen.
    \end{enumerate}}
    \vspace*{.65cm}
    \onslide<5->{Ob \(Z5\) wirklich immer zwei Vergleiche hat, hängt davon ab, wie das logische \say{und} funktioniert. \onslide<6->{Wertet es für \say{\(\text{falsch} \land \langle?\rangle\)} den zweiten Operand nicht mehr aus, haben wir Falle eines \(S\) ohne \(a\)'s, einen Aufwand von \(2n + 4\).}}
\begin{tikzpicture}[overlay,remember picture]
\onslide<7->{\node[above left,xshift=5mm,scale=.8,yshift=\btdmfootheight] at(current page.south east) {\copy\pinguexplainbox};}% copy for animations
\end{tikzpicture}
\end{frame}

\iffull
{\AddonFrame
\def\a{\text{\textit{Answer}}}
\def\b{\text{\textit{B}}}
% ugly hell
\makeatletter\def\boxlot#1#2{\text{\phantom{#2}\llap{\@solLstHL{fill=#1,draw=#1}{#2}}}}
\IfBtdmDarkmode{\def\boxme#1#2{\only<-12|handout:0>{\text{#2}}\only<13->{\boxlot{#1!12!btdm@background}{\text{\phantom{#2}\llap{#2}}}}}}{\def\boxme#1#2{\only<-12|handout:0>{#2}\only<13->{\boxlot{#1}{#2}}}}
\begin{frame}{Alternativer Ansatz\hfill I}
\pause\begin{algorithm}[H]
\PreCode
    \KwIn{String $S$ of chars $S[i]$ and length $N$}\medskip
\StartCode
    \onslide<+(1)->{$i  = 0$\;}
    \onslide<+(1)->{$\a = false$\;}
    \onslide<+(1)->{$\b = false$\medskip\;}
    \onslide<+(1)->{\While{$i \leq N$}{
        \onslide<+(1)->{\lIf{\boxme{shadeA}{\(S[i]\)~\T{==}~\('a'\)} \KwAnd\ \boxme{shadeA}{\(\b\)~\T{==}~\(false\)}}{$\a= true$}}
        \onslide<+(1)->{\lIf{{\boxme{lightgray!50!white!70!shadeB}{\(S[i]\)~\T{==}~\('b'\)}}}{\boxme{lightgray!90!btdm@background}{$\b= true$; $\a = false$;} Exit while early}}
        \onslide<+(1)->{Erhöhe~\(i\) um \(1\)\;}
    }}\medskip
    \onslide<+(1)->{\lIf{\(\a\)~\T{==}~\(true\)}{Output: \say{Ja}}}
    \onslide<+(1)->{\lElse{Output: \say{Nein}}}
\end{algorithm}
\begin{tikzpicture}[overlay,remember picture]
    \onslide<+(1)->{\node[above left,yshift=\btdmfootheight+5mm,xshift=-3.5mm,align=left,gray,font=\footnotesize\sffamily\slshape] at (current page.south east) {Die Laufzeit beläuft sich auf:\\% no cause this seems nicer align
        \upshape\(\phantom{{}={}}3 + n \cdot \bigl(2 + \max(2 + 1 + 1,\quad \llap{\boxme{shadeA}{$2$}} {+} \boxme{lightgray!50!white!70!shadeB}{$1$} {+} \boxme{lightgray!90!btdm@background}{$2$}) \bigr) + 1\) \\
        \({}={} 4 + n \cdot 7 \in \O(n)\)};}
\end{tikzpicture}
\end{frame}
{\def\xs{.75cm}
\def\BF#1{\only<-19|handout:0>{#1}\only<20->{\textbf{\color{paletteA}#1}}}
\begin{frame}{Alternativer Ansatz\hfill II}
    \begin{center}
        {\footnotesize\onslide<2->{\color{gray}Given the String \(S = s^0 s^1 s^2\ldots\)}\onslide<3->{ and let \(i \in \N\), \(Res \in \{\times, \checkmark\}\)}}\medskip
    \end{center}
\centering\resizebox{.925\linewidth}!{\begin{tikzpicture}[rounded corners=1pt,mh/.style={minimum height=2\baselineskip+3mm,rounded corners=1pt,align=center,draw},m/.style={gray,font=\small\sffamily\slshape}]
    \onslide<4->{\node[rounded rectangle,draw] (s) {Start};}
    \onslide<5->{\node[right=\xs,mh] (2) at(s.east) {Set \(i\) to \(1\)\\Set \(Res\) to $\times$}; \draw[-Kite] (s) -> (2); \onslide<19->{\node[below=-.5mm,m]  at(2.south) {$\BF2$};}}%
    % rectangle split did not want to work
    \onslide<6->{\node[align=center,right=\xs,inner xsep=5pt,mh] (3) at(2.east) {Calculate length\\of \(\abs{S}\) as \(N \in \N_0\)}; \draw([xshift=3pt]3.north west)--([xshift=3pt]3.south west); \draw([xshift=-3pt]3.north east)--([xshift=-3pt]3.south east); \draw[-Kite] (2) -> (3); \onslide<19->{\node[below=-.5mm,m]  at(3.south) {$\BF1$};}}

    \onslide<7->{\node[right=\xs,mh,minimum width=2.5cm,diamond,aspect=1.5,draw] (4) at (3.east) {\strut\(i \leq N\)};\draw[-Kite] (3) -> (4); \onslide<19->{\pgfinterruptboundingbox\node[below right,m]  at(4.south) {$\BF1$};\node[above,m] at(4.north) {$\BF{n}$};\endpgfinterruptboundingbox}
    }

    \onslide<8->{\node[draw,trapezium,trapezium left angle = 80,trapezium right angle = 100,trapezium stretches,mh,right=\xs] (5) at(4.east) {Print \(\langle Res \rangle\):\\ \(\times\mapsto \text{\say{Nein}}\), \(\checkmark \mapsto \text{\say{Ja}}\)};
        \node[above right,xshift=-1.33mm] at (4.east) {false};
        \draw[-Kite] (4) -> (5);
        \onslide<19->{\node[below=-.5mm,m]  at(5.south) {$\BF1$};}
    }

    \onslide<10->{\node[below=\xs-2mm,mh,minimum width=2.5cm,diamond,aspect=1.5,draw] (6) at (4.south) {\strut\(s^i = \text{\say{a}}\)};
        \node[below left]  at (4.south) {true};
        \draw[-Kite] (4) -> (6);
        \onslide<19->{\node[below=-.5mm,m]  at(6.south) {$\BF1$};}
    }

    \onslide<11->{
        \node[mh] (7) at(6.east-|5.south) {Set \(Res\) to $\checkmark$};%
        \node[above right,xshift=-1.33mm] at (6.east) {true};
        \draw[-Kite] (6) -> (7);
        \onslide<19->{\node[below left,m] at (7.south) {$\BF1$};}
    }

    \onslide<14->{\node[mh,minimum width=2.5cm,diamond,aspect=1.5,draw] (8) at (6.west-|3.south) {\strut\(s^i = \text{\say{b}}\)};
        \draw[-Kite] (6) -> (8);
        \node[above left,xshift=1.33mm] at (6.west) {false};
        \onslide<19->{\node[below right,m]  at(8.south) {$1$};}
    }
    \onslide<16->{
        \node[mh] (9) at(8.west-|2.south) {Set \(Res\) to $\times$};%
        \node[above left,xshift=1.33mm] at (8.west) {true};
        \draw[-Kite] (8) -> (9);
        \onslide<19->{\node[below left,m]  at(9.south) {$1$};}
    }

    \onslide<12->{\node[mh,below=\xs-2mm] (10) at(6.south) {Increment \(i\) by \(1\)};%
        \draw[-Kite] (7) |- (10.5);
        \onslide<19->{\node[below=-.5mm,m] (di) at(10.south) {$\BF1$};}
    }

    \onslide<9->{
        \node[rounded rectangle,draw,right=\xs] (e) at(5.east) {End};
        \draw[-Kite] (5) -> (e);
    }

    \onslide<15->{\draw[-Kite] (8) |- (10);\node[below left] at (8.south) {false};}
    \onslide<13->{\draw[-Kite] (10.-5) -- ([xshift=-3mm]5.east|-10.-5) |- ([yshift=5mm]7.north) -| (4.south east);}

    \onslide<17->{\draw[-Kite] (9) |- ([yshift=-4mm]10.south) -| ([xshift=5mm]5.-11) -- (5.-11);}
\end{tikzpicture}}
\pause
\begin{tikzpicture}[overlay,remember picture]
    \onslide<18->{\node[above left,yshift=\btdmfootheight+1mm,xshift=-3.5mm,align=left,gray,font=\footnotesize\sffamily\slshape] at (current page.south east) {\(2 + 1 + n \cdot (2 + 2) + 1  = 4 + n \cdot 4 \in \O(n)\)};}
\end{tikzpicture}
% TODO: Note about increment und pingu mit wir mögen textuell mehr
\end{frame}}

\begin{frame}{Weiter Kommentare}
    \begin{itemize}[<+(1)->]
        \itemsep9.5pt
        \item Ob wir das Inkrement mitzählen oder nicht, hängt von der Definition ab.
        \item Analog könnte das Berechnen der Länge auch in \(\O(n)\) oder vergleichbar sein.
        \item Deswegen wichtig: Annahmen in Textform festhalten!
        \item Spielt das überhaupt eine Rolle?\pause \infoblock{Ja und nein. Im Allgemeinen Fall nein, weil es uns mehr darum geht ob die Algorithmen konstant (\(\O(1)\)), polynomiell (\(\O(n^2)\)) oder schlimmer sind (\(\O(2^n)\), \(\O(n!)\),~\ldots). Dennoch kann es sinnvoll sein um Algorithmen der gleichen Wachstumsklasse zu vergleichen oder um Randentscheidungen zu treffen (vergleiche \(2^n\) vs. \(n^{4000}\) für \(n = 10\)). Die Nützlichkeit der Definitionen hängt vom konkreten Fall ab.}
        \item Wir präferieren strukturiert-textuelle Beschreibungen gegenüber Programmablaufplänen.
    \end{itemize}
\end{frame}
}
\fi

\iffull
\SetNextSectionText[.35\linewidth]{Avoid syntactic elements from the target programming language\\--- Steve McConnell, \cite[p.~54]{10.5555/151071}}
\section{Bonus: Pseudocode}
\setbox\pinguA=\hbox{\raisebox{-1pt}{\scalebox{.15}{\tikz{\pingu}}}}
\def\tpingu{\copy\pinguA}
\def\tagent{\pingu[sunglasses,wings wave,heart=paletteA,eyes wink]}
{\SetAlgoVlined
\begin{frame}[t]{How-To Pseudocode}
    \begin{itemize}[<+(1)->]
        \itemsep5pt
        \item \textbf{Konsistent bleiben}
        \item Menschenlesbare Notation \info{meist Programmiersprachen-Mathe-Gemisch}
        \item Solange \emph{klar} und \emph{eindeutig}, frei gestaltbar
        \item Beispiel:\par
% \setcounter{algocf}{2}
\begin{algorithm}[H]
    \PreCode
    \onslide<6->{\KwIn{Eingabe als Liste von $n$ Pingus: $\tpingu_1, \ldots \tpingu_n$}}
    \StartCode
    \onslide<7->{\Comment{Jeder Pinguin ist klasse!}
    $i \gets 1$\;}
    \onslide<8->{\While{$i \leq n$}{
        \onslide<9->{\lIf{$\tpingu_i$ ist klasse!}{
            \onslide<10->{\KwRet{$\tpingu_i$}}
        }}
        \onslide<11->{$i \gets i + 1$\;}
    }}
    \onslide<12->{\KwRet{\raisebox{-1pt}{\scalebox{0.15}{\tikz{\tagent}}}}\Comment*[l]{Wenn kein klasse Pingu gefunden.}}
    % \caption{Einen \say{klasse} Pinguin finden}
\end{algorithm}
    \end{itemize}
\end{frame}
}

\begin{frame}[c]
\null\vfill\centering\begin{tikzpicture}[scale=2.5,every node/.style={transform shape}]
    \tagent
\end{tikzpicture}\vfill\null
\end{frame}

\begin{frame}[t]{Pseudocode: do's and dont's}
    \begin{itemize}[<+(1)->]
        \itemsep7.5pt
        \item \emph{Do:} Gerne eine an Python oder C angelehnte Syntax verwenden.
        \item \emph{Do:} mathematisch bleiben, also $n \in \N$, $n \in [0,\infty)$ oder \say{$\T{Zeichenkette}~n$}.
        \item \emph{Don't:} Spracheigene \say{syntactic-sugar} Funktionen oder Definitionen verwenden.\pause{} Also: \emph{kein} \bjava{int}, \bjava{String} oder \bjava{double}.
        \item \emph{Don't:} Einfach nur Java- oder C-Code.
        \item \emph{Don't:} Zu allgemein Formulieren:\smallskip\pause
\setcounter{algocf}{3}
\begin{algorithm}[H]
\PreCode
    \KwIn{Eingabe als Liste von $n$ Pingus: $\tpingu_1, \ldots \tpingu_n$}
\StartCode
    \pause sucheKlassePingu($\tpingu_1, \ldots \tpingu_n$, \raisebox{-1pt}{\scalebox{0.15}{\tikz{\tagent}}})\;\smallskip
    \caption{Einen \say{klasse} Pinguin finden}
\end{algorithm}
    \end{itemize}
\end{frame}

\begin{frame}[fragile,c]{Pseudocode, Beispiele\hfill I}
\begin{plainvoid}[morecomment={[l]//}]
!*\onslide<2->*!In: !*$\tpingu_1, \ldots, \tpingu_n$*!
!*\onslide<3->*!Out: awesome-pengu, if none: mega-pengu
!*\onslide<4->*!iterate for every !*$\tpingu_i$*! in !*$\tpingu_1, \ldots, \tpingu_n$*!:
!*\onslide<5->*!    if !*$\tpingu_i$*! is awesome: !*\onslide<6->*! return !*$\tpingu_i$*! // Pengu found
!*\onslide<7->*!return !*\raisebox{-1pt}{\scalebox{0.15}{\tikz{\tagent}}}*! // Not found
\end{plainvoid}
\end{frame}

\begin{frame}[fragile,c]{Pseudocode, Beispiele\hfill II}
\begin{plainvoid}[morecomment={[s]\[\]}]
!*\onslide<2->*!Sei A die Liste an n Pingus, indexiert mit !*$\tpingu_i$*!.
!*\onslide<3->*!Mache nun für jeden Pingu !*$\tpingu_i$*! aus A: [Durchsuche Pingus]
!*\onslide<4->*!    Wenn !*$\tpingu_i$*! klasse ist:
!*\onslide<5->*!            Gebe !*$\tpingu_i$*! zurück. [Pingu gefunden]
!*\onslide<6->*!Wenn kein Pingu super war:
!*\onslide<7->*!    Gebe Mega-Pingu (!*\raisebox{-1pt}{\scalebox{0.15}{\tikz{\tagent}}}*!) zurück. [Standardwert: Nichts gefunden]
\end{plainvoid}
\end{frame}
\fi

\SetNextSectionText{Organisatorisches und Einführung\\Abgabe: \DTMDate{2022-04-25}\medskip\\Einzelabgabe, kein reguläres Übungsblatt.}
\section{Übungsblatt 0}
\subsection{Aufgabe 1}
\begin{frame}[t]{Aufgabe 1: Java Compiler und Laufzeitumgebung}
    \task<2->{Installieren Sie die für Ihr Betriebssystem aktuelle Version des \link{https://www.oracle.com/java/technologies/downloads/}{Java Development Kits}. Diese werden Sie auch noch im Lauf der Veranstaltung für die Bearbeitung der Programmieranteile der  Übungsblätter benötigen. Bestimmen und notieren Sie anschließend die Versionsnummern.}\vfill

    % i do not want number highlight for this
    \soldisablenumhl\def\smallerXLST{\lstfs[1.2]{8}}% bracket parser
    \begin{itemize}
        \item<3-> In der Konsole: \rbash[:\smallerXLST\onslide<4->]{java -version}
        \item<5-> Sowie: \rbash[:\smallerXLST\onslide<6->]{javac -version}
    \end{itemize}
\end{frame}

\subsection{Aufgabe 2}
\iffull\rExecute{javac HelloWorld.java}\fi
\begin{frame}[t]{Aufgabe 2: Erste Schritte in Java}
    \task<2->{Speichern Sie den Code in einer Textdatei namens \T{HelloWorld.java} ab.}\vfill
    \begin{itemize}[<+(1)->]
        \item<3-> Tipp, tipp, tipp, \ldots \onslide<4->{\ijava{HelloWorld.java}}
    \end{itemize}
    \task<5->{Übersetzen Sie das Programm mittels \bbash{javac} und führen Sie den erzeugten Java-Bytecode mit \bbash{java} aus.}\vfill
    \begin{itemize}[<+(1)->]
        \item<6-> Wir tun wie geheißen: \cbash{javac HelloWorld.java}%
        \item<7-> Und dann auch für die Ausgabe: \iffull\Rbash[:\onslide<8->]{java HelloWorld}{java HelloWorld}\else\cbash{java HelloWorld}.\fi
    \end{itemize}
\end{frame}

\iffull
\SetNextSectionText{Algorithmenentwurf und -analyse\\Abgabe: \DTMDate{2022-05-02}}
\section{Aussicht: Übungsblatt 1}
\MakeThePinguExplainIt[text width=6.65cm]{cap=!hide,headphones,glasses=!hide,cup=!hide,glasses round,eyes shiny,heart=shadeC,body type=legacy,right item angle=-52}{Mit der Zeit werden wir mehr Eigenschaften zur Charakterisierung betrachten.}
\begin{frame}{Ein \say{Algorithmus}, was ist das?}
    \begin{itemize}[<+(1)->]
        \itemsep13pt
        \item \emph{Eindeutige} Handlungsvorschrift zur Lösung eines Problems.\pause\infoblock{Die ausführbar und reproduzierbar ist.}
        \item Endlich viele, \emph{wohldefinierte} \info{also nicht mehrdeutige} und elementare Einzelschritte.
        \item Der Algorithmus muss nach einer endlichen Zeit zum Ende kommen (\say{terminieren}).% \vfill
        % \item Mit der Zeit werden wir mehr Eigenschaften zur Charakterisierung betrachten.% TODO: => Tutorium 1 (Determinismus, Korrektheit,\ldots)
    \end{itemize}
\begin{tikzpicture}[overlay,remember picture]
\onslide<6->{\node[above left,xshift=5mm,scale=.8,yshift=\btdmfootheight] at(current page.south east) {\copy\pinguexplainbox};}% copy for animations
\end{tikzpicture}
\end{frame}
\begin{frame}{Die Spezifikation}
    \onslide<2->{\begin{center}
    \begin{tikzpicture}
        \IfBtdmDarkmode{% TODO
            \node[block,paletteA,draw,thick] (A) at(0,0) {Algorithmus};
        }{
            \node[block,text=paletteA,thick,draw=gray] (A) at(0,0) {\sbseries Algorithmus};
        }
        \onslide<3->{\node (V) at(-5,0) {Vorbedingung};
        \draw[-Kite] (V) -- (A);}
        \onslide<4->{\node (N) at(5,0) {Nachbedingung};
        \draw[-Kite] (A) -- (N);}
        \onslide<1->
    \end{tikzpicture}
    \end{center}}
    \begin{itemize}
        \itemsep10.5pt
        \item<5-> Suche die größte, ganze Zahl in einem beliebigen (ganzzahligen, nicht-leeren) Array.
        \item<6-> Information: Die Regeln zum Pseudocode gelten hier nach wie vor!
        \item<7-> Problemspezifikation:
        \begin{description}[Ganzzahliges Array:]
            \item<8->[Ganzzahliges Array:] Tupel \(t = (t_1, t_2, \ldots, t_m) \in \Z^m\) der Größe \(m \geq 1\) mit \info{\(t_i \in \Z\) für alle \(t_i\)}
            \item<9->[Beliebig:] Die Elemente können unsortiert vorliegen \info{z.B. \(t_1 \leq t_2 > t_3\)}.
            \item<10->[Größte:] Die größte, ganze Zahl \(z \in t\) mit \(z = \max(t)\).
        \end{description}
        \onslide<11->{Ordentlich wäre rein mathematische\textor axiomatisch belegte Formalisierungen zu nutzen. \onslide<12->{In dieser Veranstaltung beschreiten wir einen Mittelweg.}}
    \end{itemize}
\end{frame}

\begin{frame}{Die Abstraktion}
    \begin{center}
    \begin{tikzpicture}
        \IfBtdmDarkmode{% TODO
            \node[block,paletteA,draw,thick] (A) at(0,0) {Algorithmus};
        }{
            \node[block,text=paletteA,thick,draw=gray] (A) at(0,0) {\sbseries Algorithmus};
        }
        \node (V) at(-5,0) {Vorbedingung};
        \node (N) at(5,0) {Nachbedingung};
        \draw[-Kite] (V) edge (A) (A) edge (N);
    \end{tikzpicture}
    \end{center}
    \begin{itemize}[<+(1)->]
        \itemsep10.5pt
        \item \textcolor{gray}{Suche die größte, ganze Zahl in einem beliebigen (ganzzahligen, nicht-leeren) Array.}
        \item Problemabstraktion:
        \begin{description}[Gegeben:]
            \item[Gegeben:] Endliches unsortiertes Array $t$ an ganzen Zahlen \(n \in \Z\).
            \item[Gesucht:] Maximales ganzzahliges Element \(x = \max(t)\).
        \end{description}
        \onslide<+(1)->{In diesem Fall ist die Abstraktion nahezu offensichtlich. \onslide<+(1)->{Es dient der Veranschaulichung des Vorgehens \Laughey.}}
    \end{itemize}
\end{frame}

% das hier ist wirklich nur für das formale.
\begin{frame}[t]{Entwurf und Korrektheit}
    \begin{itemize}[<+(1)->]
        \item Algorithmenentwurf: \info{Annahme eines 0-indizierten Arrays mit Zugriff \T{a[$i$]} für das \(i\)-te Element \(t_i\).}\\
        \begin{enumerate}\itemsep0pt
            \item Setze \(max = \text{\T{a[$0$]}}\).
            \item Setze \(i = 1\).
            \item \label{alg:loop}Solange \(i < m\):\\
                \pause{\quad Wenn (\(max < \text{\T{a[$i$]}}\)):}\pause{~~\(max = \text{\T{a[$i$]}}\).}\\
                \pause{\quad Inkrementiere \(i\) um \(1\).}
            \item<+(1)-> Lösung ist \(max\).
        \end{enumerate}
        \item Korrektheitsnachweis: \begin{description}[Partiell korrekt:]
            \itemsep3pt
            \item[Terminiert:] Die Schleife aus \ref{alg:loop}. terminiert, da \(i\) pro Iteration um \(1\) inkrementiert wird und damit streng monoton wächst \onslide<+(1)->{\info{also sicher irgendwann \(i \geq m\)}.}
            \item<+(1)->[Partiell korrekt:] Hier lässt sich leicht zeigen, dass für jeden Schritt in \ref{alg:loop}. gilt, dass \(max \geq (t_1, \ldots, t_i)\). \onslide<+(1)->{Für \(m = 1\) ist weiter \(max = \text{\T{a[$0$]}} = \max\bigl(t_0\bigr)\).}
        \end{description}
    \end{itemize}
\end{frame}

\newsavebox\algobox
\savebox\algobox{\parbox\linewidth{\setbeamercolor{enumerate item}{fg=gray}\scriptsize\begin{enumerate}\itemsep0pt
    \item \color{gray}\label{alg:a}Setze \(max = \text{\T{a[$0$]}}\).
    \item \color{gray}\label{alg:b}Setze \(i = 1\).
    \item \color{gray}\label{alg:loop2}Solange \(i < m\): \\
        \quad Wenn (\(max < \text{\T{a[$i$]}}\)):~~\(max = \text{\T{a[$i$]}}\). \\
        \quad Inkrementiere \(i\) um \(1\).
    \item \color{gray}Lösung ist \(max\).
\end{enumerate}}}

\begin{frame}{Die Aufwandsanalyse}
\begin{itemize}[]
    \item Algorithmenentwurf: \info{Annahme eines 0-indizierten Arrays mit Zugriff \T{a[$i$]} für das \(i\)-te Element \(t_i\).}\\
    \usebox\algobox\vfill
    \item<+(1)-> Aufwandsanalyse:\pause\ Wir machen dies als strukturierten Text: \begin{itemize}
        \item<+(1)-> \ref{alg:a}. und \ref{alg:b}. entsprechen jeweils einer Elementaroperation.
        \item<+(1)-> \ref{alg:loop2}. wird genau \(m - 1\) mal ausgeführt. Sie enthält sicher einen Vergleich und ein Inkrement. Eventuell eine Zuweisung.\pause\ Für unseren Fall also drei Elementaroperationen.
    \end{itemize}
        \pause Damit ist der Gesamtaufwand \(1 + 1 + (m - 1) \cdot \bigl( 3 \bigr) = 2 + 3m - 3 = 3m - 1\).\medskip\\
        \pause Für komplexere Szenarien können sich die Analysen auch komplexer gestalten.
\end{itemize}
\end{frame}
\fi

\SetNextSectionText{He who chooses the beginning of a road chooses the place it leads to. It is the means that determine the end.\\--- Harry Emerson Fosdick, \cite[p.~111]{fosdick1941living}}
\section{Abschließendes}
{\SummaryFrame
\begin{frame}[t]{Zusammenfassend}
\pause \printBibCommand
\vfill\vfill % double fill for more fraction
\begin{itemize}[<+(1)->]
    \itemsep14pt
    \item Für Pseudocode wünschen wir uns \begin{itemize}
        \itemsep1pt
        \item eine konsistente und menschenlesbare Notation,
        \item mathematische Definitionen,
        \item aber nichts sprachspezifisches (\bjava{int}) oder zu allgemeines (\say{löst Problem}).
    \end{itemize}
    \item Algorithmen und deren Konstruktion: \begin{itemize}
        \itemsep1pt
        \item Eine eindeutige, endliche Beschreibung wohldefinierter Elementaroperationen, deren schrittweise Ausführung durch einen Prozessor möglich und endlich ist.
        \item \def\t{~~\faAngleRight~~}Spezifikation\t Abstraktion\t Entwurf\t Verifikation\t Aufwandsanalyse
        % \item Als Frage für die Woche: Warum und wann sind all diese Schritte von Bedeutung?
    \end{itemize}
\end{itemize}
\end{frame}
}

\outro{\vskip6mm\centering\begin{tikzpicture}[scale=2.5]
    \only<2->{\pingu[right wing grab,headband=purple!90!green,cup=purple!90!green,name=saphira, left eye wink,body type=chubby]}
\end{tikzpicture}}

\iffull\end{document}\fi
